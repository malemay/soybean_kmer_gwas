\documentclass[12pt]{article}

\usepackage[utf8]{inputenc} % declare encoding as utf8
\usepackage[T1]{fontenc} % declare encoding as utf8
\usepackage{graphicx} % to enable \includegraphics
\usepackage{flafter} % to make sure floats appear after their position in text
\usepackage{pdflscape} % allow some pages to be in landscape orientation
\usepackage{layout} % for debugging purposes, looking at page layout
\usepackage[margin=1in]{geometry} % leaves more space for text body
\usepackage[hidelinks]{hyperref} % hyperlinks from list of figures/tables
\usepackage[labelfont=bf]{caption} % format captions and make links go to figure instead of caption
\usepackage{csvsimple} % read and display csv files as tables
\usepackage{booktabs} % prettier tables
\usepackage{threeparttable} % easy addition of footnotes below tables
\usepackage{textcomp} % makes the \textdegree command available
\usepackage{tocloft} % formatting the lists of tables and figures
\usepackage{xcolor} % enable text coloring
\usepackage{calc} % makes \widthof available to set the width of the last column
\usepackage{array} % using command \arraybackslash to restore the definition of \\ after \centering
\usepackage{longtable} % allow tables to span more than one page
\usepackage{threeparttablex} % threeparttable for longtable
\usepackage{capt-of} % put caption on next page
\usepackage{ccaption}
\usepackage{listings}
\usepackage{datatool} % allows fetching key-value pairs from a text file
\usepackage{apacite}
\usepackage{setspace}

\title{Supplementary material to ``\textit{k}-mer-based GWAS enhances the discovery of causal variants and candidate genes in soybean''}
\date{}

\author{Marc-André Lemay \and
Maxime de Ronne \and
Richard Bélanger \and
François Belzile}

% Remove indent for new paragraphs
\setlength{\parindent}{0pt}

% Adjust line spacing to 1.5
\onehalfspacing

% Setting the separator to use with datatool
\DTLsetseparator{ = }
\graphicspath{ {../figures/} } % tell Latex to look for figures in figures/

% Defining new environments as I will always want the figures centered
\newenvironment{cfigure}
{\begin{figure} \centering}
{\end{figure}}

\newenvironment{lsfigure}
{\begin{landscape} \begin{figure} \centering}
{\end{figure} \end{landscape}}

\setcounter{totalnumber}{1} % Allow a maximum of one float (table/figure) per page

\setlength{\parskip}{2ex}

% Defining a command to generate a table of phenotypic data for a single gwas
\newcommand{\gwastableone}[2]{
	\begin{table}
		\caption[Phenotypic data used for the GWAS analysis of #2]%
		{Phenotype frequency and numerical coding of phenotypic data used for the GWAS analysis of #2.
		``NA'' indicates a missing value in the source data.
		Dashes indicate that this phenotype was not used in the GWAS analysis.}
		\label{#1_gwas_table}

		\centering
		\begin{threeparttable}\small

			\csvreader[head to column names,
			/csv/separator=semicolon,
			tabular = llc,
			table head = \toprule Phenotype & Frequency & GWAS \\ \midrule,
			table foot = \bottomrule]%
			{../tables/#1_gwas_table.csv}%
			{}%
			{\Value & \Frequency & \GWASa}

		\end{threeparttable}

	\end{table}
	\clearpage
}

% Defining a command to generate a table of phenotypic data for two gwas
\newcommand{\gwastabletwo}[2]{
	\begin{table}
		\caption[Phenotypic data used for the GWAS analyses of #2]%
		{Phenotype frequency and numerical coding of phenotypic data used in two GWAS analyses of #2.
		``NA'' indicates a missing value in the source data.
		Dashes indicate that this phenotype was not used in the GWAS analysis.}
		\label{#1_gwas_table}

		\centering
		\begin{threeparttable}\small

			\csvreader[head to column names,
			/csv/separator=semicolon,
			tabular = llcc,
			table head = \toprule Phenotype & Frequency & GWAS \#1 & GWAS \# 2\\ \midrule,
			table foot = \bottomrule]%
			{../tables/#1_gwas_table.csv}%
			{}%
			{\Value & \Frequency & \GWASa & \GWASb}

		\end{threeparttable}

	\end{table}
	\clearpage
}

% Defining a command to generate a table of phenotypic data for three gwas
\newcommand{\gwastablethree}[2]{
	\begin{table}
		\caption[Phenotypic data used for the GWAS analyses of #2]%
		{Phenotype frequency and numerical coding of phenotypic data used in three GWAS analyses of #2.
		``NA'' indicates a missing value in the source data.
		Dashes indicate that this phenotype was not used in the GWAS analysis.}
		\label{#1_gwas_table}

		\centering
		\begin{threeparttable}\small

			\csvreader[head to column names,
			/csv/separator=semicolon,
			tabular = llccc,
			table head = \toprule Phenotype & Frequency & GWAS \#1 & GWAS \# 2 & GWAS \# 3\\ \midrule,
			table foot = \bottomrule]%
			{../tables/#1_gwas_table.csv}%
			{}%
			{\Value & \Frequency & \GWASa & \GWASb & \GWASc}

		\end{threeparttable}

	\end{table}
	\clearpage
}

% Defining a command for manhattan plot figures
\newcommand{\manhattanplot}[3]{
	\begin{cfigure}
		\includegraphics[width = 6.5in]{#1_manhattan}
		\caption[Manhattan plots for #2]{Manhattan plots and associated Q-Q plots
		generated from the GWAS analysis of #2 on #3 samples using three genotype datasets :
		(a) Platypus (SNPs and indels), (b) Paragraph (SVs), (c) \emph{k}-mers presence/absence.
		The x-axis shows the position along the reference assembly version 4 of Williams82.
		Each point shows the $-\log_{10} (p)$ associated with a particular marker or \emph{k}-mer.
		Horizontal dotted lines indicate the 5\% family-wise error-rate significance threshold determined from a randomization approach.
		Vertical dotted lines indicate the position of signals associated with the trait.
		Documented loci are colored according to whether they were found (green) or not (red) by a particular approach.
		The ``Gm'' prefix has been left out of chromosome names for simplicity.
		The dashed red line in Q-Q plots represents the identity line ($y = x$)
		}
		\label{#1_manhattan}
	\end{cfigure}

	\clearpage}

% Defining a new command for signal plot figures
\newcommand{\signalplot}[4]{
	\begin{cfigure}
		\includegraphics[width = 5.8in]{#1_signal}
		\caption[Zoomed-in Manhattan plots of the #3 locus]{Zoomed-in Manhattan plots of signals detected by the GWAS analysis of #2 at the #3 locus using three genotype datasets:
		(b) Platypus (SNPs and indels), (c) Paragraph (SVs), (d) \emph{k}-mers presence/absence.
		Panel (a) shows gene models over the genomic interval.
		Horizontal dotted lines indicate the 5\% family-wise error-rate significance threshold determined from a randomization approach.
		#4
		Gray shaded rectangles indicate the region delimited by the top 5\% (SVs) or top 1\% (SNPs/indels and \emph{k}-mers) associations in the signal region.
		In the case of SNPs/indels, blue points denote markers used in the original analysis, whereas red points denote markers that had originally been pruned but whose
		\emph{p}-values were computed after signal discovery.
		}
		\label{#1_signal}
	\end{cfigure}

	\clearpage}

% Defining a new command for gene plot figures
\newcommand{\geneplot}[5]{
	\begin{cfigure}
		\includegraphics[width = 5.8in]{#1_gene}
		\caption[Zoomed-in Manhattan plots of the #3 locus gene]%
		{Zoomed-in Manhattan plots generated from the GWAS analysis of #2 at the #4 associated with the #3 locus using three genotype datasets:
		(b) Platypus (SNPs and indels), (c) Paragraph (SVs), (d) \emph{k}-mers presence/absence.
		#5
		Horizontal dotted lines indicate the 5\% family-wise error-rate significance threshold determined from a randomization approach.
		In the case of SNPs/indels, blue points denote markers used in the original analysis, whereas red points denote markers that had originally been pruned but whose
		\emph{p}-values were computed after signal discovery.
		Panel (a) shows gene models over the plotting interval.
		Exons are represented by rectangles whereas introns are represented by horizontal lines.
		Coding sequences are shown in blue and the direction of transcription is indicated by arrows.
		}
		\label{#1_gene}
	\end{cfigure}

	\clearpage}

% Defining a new command for k-mer plot figures
\newcommand{\kmerplot}[5]{
	\begin{cfigure}
		\includegraphics[width = 5.8in]{#1_kmers}
		\caption[Variants associated with significant \emph{k}-mers at the #3 locus]%
		{Identification of #5 underlying significant \emph{k}-mers at the #2 gene associated with the #3 locus for #4.
		(a) Gene model of #2. Exons are represented by rectangles whereas introns are represented by horizontal lines.
		Coding sequences are shown in blue and the direction of transcription is indicated by arrows.
		The red rectangle identifies the region that is zoomed-in in panel (b).
		(b) Nucleotide sequences of haplotypes observed in at least five samples across the dataset.
		Individual nucleotides are colored according to the $-\log_{10}(p)$ of the most significant \emph{k}-mer overlapping them.
		Dashes indicate gaps in haplotype sequence alignment whereas vertical lines indicate differences in sequence between two haplotypes.
		(c) Contingency table of the phenotypes and haplotypes observed in the dataset. Haplotypes correspond to those shown in panel (b).}
		\label{#1_kmers}
	\end{cfigure}

	\clearpage}

% Defining a new command for LD plot figures
\newcommand{\ldplot}[3]{
	\begin{cfigure}
		\includegraphics[width = 5.8in]{#1_ld}
		\caption[LD plot of significant \emph{k}-mers for #2]{Pairwise LD among #3 significant \emph{k}-mers identified for #2.
		\emph{k}-mers are sorted along the y-axis according to their putative position along the reference assembly version 4 of Williams82,
		as identified by ``Gm'' chromosome labels.
		Sequences that lack a ``Gm'' label (top of the y-axis) represent unanchored scaffolds.
		\emph{k}-mers are represented in the same order along the x- and y-axis.
		The colored rectangles drawn below the x-axis represent the $-\log_{10} (p)$ of each \emph{k}-mer.}
		\label{#1_ld}
	\end{cfigure}

	\clearpage}

% END OF PREAMBLE

\begin{document}


\maketitle

% Format figure and table numbers to follow S1, S2, ...
\renewcommand\thefigure{S\arabic{figure}} 
\setcounter{figure}{0}
\setcounter{table}{0}
\renewcommand\thetable{S\arabic{table}} 

% Reading the contents of the variables.txt file first
\DTLloaddb[noheader, keys={key,value}]{variables}{variables.txt}
% Defining a new command to fetch the value from the key
\newcommand{\getvar}[1]{\DTLfetch{variables}{key}{#1}{value}}

% Turning page numbering off for the supplements
\setcounter{page}{1}

% Detailed methods for some aspects of the paper
\section*{Detailed methods}
\label{annexe-sv-gwas-detailed-methods}

\subsection*{Discovery of structural variation from Illumina data}
\label{sv-gwas-sv-discovery-illumina}

We used the Illumina whole-genome sequencing (WGS) data from the 389 selected \emph{G. max} accessions
for SV discovery using methods similar to those described by
\shortciteA{lemay2022}.  Briefly, we used four different SV discovery programs or
pipelines: 

\begin{enumerate}

\item We used AsmVar \shortcite<version of 2015-04-16, >[]{liu2015} to call SVs
	from \emph{de novo} assemblies generated with SOAPdenovo2 v. 2.04
		\shortcite{luo2012} and aligned to the reference genome using LAST
		v. 1047 \shortcite{kielbasa2011}.
\item We ran Manta v. 1.6.0 \shortcite{chen2016} in randomly selected batches: 77 batches of
	5 samples and 1 batch of 4 samples. We used the candidate SVs
		identified from each of the batches for downstream analyses.
\item We used smoove v. 0.2.4 \shortcite{pedersen2019} to obtain a dataset of candidate SVs
	using the following sequence of commands:
	\texttt{smoove call}, \texttt{smoove merge}, \texttt{smoove
		genotype} and \texttt{smoove paste}.
		For three (3) of the 389 samples, we were unable
		to obtain candidate variants due to a reproducible
		error (segmentation fault) when running \texttt{smoove
		call}.
	\item We used SvABA v. 1.1.3 \shortcite{wala2018} to call SVs from each of the
	samples separately. We classified the SVs called as breakends by SvABA
		as either deletions, duplications or inversions, and converted
		them to explicit sequence using a custom R script. Variants
		called as indels by SvABA were used as is for downstream
		analyses.
\end{enumerate}

We filtered the SVs called by each tool to remove those meeting any of the following conditions:

\begin{itemize}
	\item smaller than 50 bp or larger than 500 kb in size
	\item located on unanchored scaffolds or organellar genomes
	\item classified as unresolved breakends (\texttt{SVTYPE=BND})
	\item containing at least one N in the alternate allele sequence
\end{itemize}

Next, we converted all filtered variants to a sequence-explicit format using
\texttt{bayesTyperTools convertAllele} \shortcite{sibbesen2018} and normalized them
using bcftools \texttt{norm} v. 1.10.2-105 \shortcite{li2009-samtools}.

A single VCF file was generated from the variants called using Illumina data by
running SVmerge \shortcite{wong2010} in two steps. In the first step, we merged
very similar variants among the ones called by the same tool using a
sliding window of 5 bp. Next, we used the output of the first merging steps
for all four tools and merged similar variants across the four datasets using
a sliding window of 15 bp. This approach was used because a
sliding window of 15 bp on the outputs of all tools without pre-merging
resulted in a computationally intractable number of pairwise comparisons.

\subsection*{Discovery of structural variation from Oxford Nanopore sequencing data}
\label{sv-gwas-sv-discovery-ont}

We used SVs discovered among 17 Canadian soybean cultivars sequenced using
Oxford Nano\-pore Technologies (ONT) by \shortciteA{lemay2022} as an additional
source of candidate SVs. Although this set of cultivars does not overlap the
ones analyzed in this study, it is likely that many of the SVs occurring
in Canadian germplasm also occur in the germplasm used for this study given that modern soybean
cultivars are derived from a rather small set of progenitors. We merged the SVs detected in the
long-read (ONT) data with the ones discovered in the short-read (Illumina) data
using SVmerge and systematically favored SVs discovered by Illumina whenever
SVs from the two sets were to be merged.  Our rationale for favoring variants
discovered from the Illumina data was that whenever both technologies discover
the same variants, the breakpoints of SVs discovered from Illumina data are
likely to be more precise given the higher basecalling accuracy.

\subsection*{Discovery of structural variation from high-quality assemblies}
\label{sv-gwas-sv-discovery-assembly}

In addition to Illumina data from 389 accessions, we also used
high-quality genome assemblies of 29 \emph{G. max} or \emph{Glycine soja}
accessions to call SVs. We
retrieved 26 high-quality genomes assembled by \shortciteA{liu2020-pangenome} as well
as the genome of the cultivar Zhonghuang 13 (ZH13) from the Genome Warehouse
repository (\url{https://ngdc.cncb.ac.cn/gwh/}). Furthermore, we retrieved the genomes
of \emph{G. soja} accession W05 and \emph{G. max} cultivar Lee from SoyBase
\shortcite{grant2010}.

We used methods based on \shortciteA{liu2020-pangenome} to call SVs from the
genome assemblies.  First, we used the nucmer program of the MUMmer suite of
tools v. 4.0.0rc1 \shortcite{marcais2018} with option \texttt{-c 1000} to align
each of the assemblies to the Williams82 assembly. Next, we used the
delta-filter program of the MUMmer suite to keep only one-to-one alignments
between any genome and the Williams82 reference using the \texttt{-1} option.
The filtered alignments were used to call SVs with a version of the svmu
program \shortcite{chakraborty2019} that we forked from the original repository
(\url{https://github.com/malemay/svmu}) in order to improve execution time and
memory usage.

We used a set of tools that we assembled in an R package called svmutools
(\url{https://github.com/malemay/svmutools}) to convert the svmu output to VCF
format. We limited ourselves to processing deletions and insertions because of
the complexity of resolving other types of variants from whole-genome
assemblies.  Copy number variants and inversions called by svmu were therefore
excluded from downstream analyses. The
resulting set of variants was filtered to remove those smaller than 50 bp or larger than 500
kb and those with any N in either the reference or alternate allele sequence.
Finally, we normalized the VCF files using \texttt{bcftools norm} and merged
them using SVmerge to obtain a single file with variants discovered through
comparison of high-quality genome assemblies.

\subsection*{Genotyping structural variation using Paragraph}
\label{sv-gwas-sv-genotyping-paragraph}

We used Paragraph v. 2.4a \shortcite{chen2019}
to generate a set of SV genotypes from the Illumina data of the 389 accessions.
The set of SVs used as input for Paragraph was generated using SVmerge by
merging variants discovered from the Illumina and ONT data with
those discovered from the high-quality assemblies and svmu.
Variants in this candidate set were prepared for input to Paragraph by first removing
those closer than 150 bp (the maximum read length in our dataset) to any
chromosome end and then padding the variant representation using a script written by
\shortciteA{hickey2020}. 

We ran Paragraph individually on each sample as recommended by the authors.  We
computed the average sequencing depth for each sample from the output of the
\texttt{samtools coverage} command and used 20 times that value for the
\texttt{-M} parameter of Paragraph.  We ran Paragraph using the
\texttt{multigrmpy} command and merged the resulting VCF files using
\texttt{bcftools merge} in order to get a  single genotype file for the whole
population. The resulting genotype calls were filtered as described in the main
text.

\subsection*{Comparison of WGS and SoySNP50K genotypes}
\label{annexe-sv-gwas-soysnp50k}
When we first analyzed the results from the GWAS analysis on all 389 samples,
we found discrepancies between the haplotypes at some loci and the expected
phenotype that should correspond to that haplotype for simple traits such as
flower color and seed coat color. Upon further analysis, we found that a
mismatch between the sequencing data and the identity of the accession was the
most probable cause of these discrepancies. In order to identify the samples
for which this could be the case, we compared the genotypes of SNPs obtained
from the SoySNP50K array \shortcite{song2013} to those predicted from the Illumina WGS
data. We did this for the 385 samples that had a match in the GRIN database. Briefly, we
retrieved SoySNP50K genotype calls from SoyBase \shortcite{grant2010} and extracted
all SNPs with a MAF $\geq$ 0.1 in our population. We then translated the positions
of these SNPs from assembly version 2 to assembly version 4 of Williams by
finding exact matches of 41-bp sequences surrounding the SNP positions. The
32,852 SNPs for which unambiguous positions could be found on genome assembly
version 4 were genotyped from the mapped WGS reads using \texttt{bcftools
mpileup} and \texttt{bcftools call}. We then used \texttt{bcftools gtcheck} to
compute the concordance between the WGS genotype calls and the
SoySNP50K genotype dataset. We identified 24 samples
with < 90\% concordance between the two genotype datasets and removed them from
the dataset. We also excluded an additional sample for which concordance was over 90\%
but poor raw sequencing data quality (\%GC content) suggested contamination.
The concordance between WGS data and SoySNP50K genotypes is included in
Supplemental file 3 and shown in Figure \ref{concordance-histogram}.

\subsection*{Analysis of extremely low \textit{p}-values}
\label{annexe-sv-gwas-pvalues}

For one trait (flower color), the significance of some \emph{k}-mers was so
high that their \emph{p}-values were numerically equal to 0. In order to
compute $-\log_{10} (p)$ for visualization in Manhattan plots in these cases,
we set the \emph{p}-value to the smallest numerical value that could be
represented using R on our machine (2.225074e-308). In this particular case,
the values represented on the Manhattan plots represent an upper bound on the
\emph{p}-value and not an exact value.

% Detailed results that are not meant to fit the main paper
\section*{Supplemental results}
\label{annexe-sv-gwas-detailed-results}

This section describes results at additional loci that were not discussed in
the main text. A summary of the signals found at all loci using all three GWAS
methods is presented in Table \ref{signals_table}.

\subsection*{Flower color -- \textit{L1} locus}
\label{annexe-sv-gwas-flower-color-l1}

In addition to the \textit{W1} locus, \shortciteA{bandillo2017} reported a signal for
flower color near the \textit{L1} locus.  None of the GWAS approaches that we
tested found a signal for flower color near this locus (Figure
\ref{flower_color_manhattan}). This result is unsurprising as W1 is the major
contributor to flower color in soybean, whereas the \textit{L1} locus
(typically associated with pod color) was detected with a much less significant
\textit{p}-value (see Table \ref{signals_table}) and only conjectured as
associated with flower color by \shortciteA{bandillo2017}.

\subsection*{Pubescence color -- locus on chromosome Gm16}
\label{annexe-sv-gwas-pubescence-color-gm16}

In addition to the \textit{T} locus, the first GWAS analysis conducted on pubescence color
revealed a second locus on chromosome Gm16
(Figure \ref{pubescence_color_all_manhattan}). Upon closer analysis, this locus
appeared to be in strong LD with \emph{k}-mers mapping to chromosome Gm01 and
most importantly in moderate LD with the \emph{T} locus (Figure
\ref{pubescence_color_all_ld}), which suggests that this does not represent an
independent locus controlling pubescence color. This interpretation is also
supported by the fact that the signal at that locus only spans about 10 kb on
chromosome Gm16.

\subsection*{Seed coat color -- loci \textit{T}, \textit{O} and \textit{R}}
\label{annexe-sv-gwas-seed-coat-color-t-o}

In addition to the \textit{I} and \textit{G} loci, the \textit{k}-mer-based
GWAS analysis of five different seed coat color phenotypes (first GWAS in Table
\ref{SEED.COAT.COLOR_gwas_table}) detected weak signals overlapping the
\emph{T} and \emph{O} loci (Figure \ref{seed_coat_color_all_manhattan}c).
However, we interpreted those results as chance associations given the low
\textit{p}-value associated with the \emph{T} locus and the proximity of the
\emph{O} locus to the \emph{I} locus. None of the three GWAS approaches detected
a signal overlapping the \textit{R} locus.

\subsection*{Stem termination type -- loci \textit{Dt2} and \textit{E3}}
\label{sv-gwas-stem-termination}

We performed a GWAS analysis comparing only semi-determinate and indeterminate
stem termination phenotypes in an attempt to detect signals at the \emph{Dt2}
locus (Table \ref{STEM.TERMINATION.TYPE_gwas_table}).  However, this was
unsuccessful using all three GWAS approaches (Figure
\ref{stem_termination_sn_manhattan}). Similarly, we failed to detect signals at
the \textit{E3} locus, which is more commonly associated with maturity but was
mentioned as associated with stem termination type by \shortciteA{bandillo2017}.

% Hilum color - T locus
\subsection*{Hilum color -- \textit{T} locus}
\label{annexe-sv-gwas-results-hilum-color-t}

The \textit{T} locus discussed in the main text for pubescence color is also a determining
locus for hilum color. We detected signals overlapping the corresponding gene
using all approaches (Figures \ref{hilum_color_all_manhattan},
\ref{hilum_color_all_T_signal}). The regions defined by the top 5\% (SVs)
and top 1\% (SNPs and \textit{k}-mers) variants included
the Glyma.06g202300 gene associated with this locus. Similarly to what we
observed for the analysis of the \textit{T} locus based on pubescence color, we
did identify the documented causal variant at this locus using the Platypus and
\textit{k}-mer analyses, although these variants were not the most
significantly associated overall (Figures \ref{hilum_color_all_T_gene},
\ref{hilum_color_all_T_kmers}).

% Hilum color - I locus
\subsection*{Hilum color -- \textit{I} locus}
\label{annexe-sv-gwas-results-hilum-color-i}

Similarly, the \textit{I} locus discussed in the main text for its effect on seed coat
color also controls hilum color and we detected signals overlapping the
associated inverted duplication using all approaches (Figures
\ref{hilum_color_all_manhattan}, \ref{hilum_color_all_I_signal}). However,
as was the case for the effect of \textit{I} on seed coat color, we were
not able to link any of the variants identified to the causal variation given
its highly complex nature.

\subsection*{Hilum color -- loci \textit{W1} and \textit{O}}
\label{sv-gwas-hilum-color}

None of the three GWAS analyses that we conducted on hilum color found signals
at the \emph{W1} and \emph{O} loci studied by \shortciteA{bandillo2017} (Figures
\ref{hilum_color_blackbrown_manhattan} \ref{hilum_color_all_manhattan},
\ref{hilum_color_rbr_manhattan}).

\subsection*{Pubescence density -- loci \textit{Pd1} and \textit{P1}}
\label{sv-gwas-pubescence-density}

Our GWAS analysis on pubescence density did not detect any signals at the
\emph{Pd1} and \emph{P1} (Figure \ref{pubescence_density_manhattan}). Our
failure to detect these loci is likely due to our small population size
compared to \shortciteA{bandillo2017} and to the smaller contribution of those loci
to pubescence density in our population.

\subsection*{Seed coat luster -- loci \textit{Hps}, \textit{B?} and \textit{I}}
\label{sv-gwas-seed-coat-luster}

We performed three different GWAS analyses on seed coat luster (Table
\ref{SEED.COAT.LUSTER_gwas_table}). Since all three analyses yielded similar results
(Figure \ref{seed_coat_luster_all_manhattan},
\ref{seed_coat_luster_nointermediate_manhattan} and
\ref{seed_coat_luster_dullshiny_manhattan}), we chose to focus on the
GWAS contrasting the dull and shiny phenotypes as it yielded the most
significant signals. In addition, it also provided results that were most consistent with the
analysis previously done by \shortciteA{bandillo2017}.

The analysis detected signals at the \emph{B} locus as the sole classical locus
using all three methods (see main text and Figure
\ref{seed_coat_luster_dullshiny_manhattan}).  All methods also detected signals
overlapping the \emph{Hps} locus as defined by \shortciteA{bandillo2017}, however
results shown by \shortciteA{gijzen2003} and our results suggest that \emph{Hps} and
\emph{B} are the same locus. 

Apparent signals on chromosomes Gm20 and Gm09 (which may correspond to the B?
locus reported by \shortciteA{bandillo2017}) were also detected, but analysis of the
LD patterns between \emph{k}-mers suggested that these signals are most likely
in LD with the \emph{B} locus (Figure \ref{seed_coat_luster_dullshiny_ld}).

\shortciteA{bandillo2017} mentioned the \textit{I} locus as possibly associated
with seed coat luster, however we did not find evidence for such an association
in our dataset.

\subsection*{Pubescence form -- additional loci}
\label{sv-gwas-pubesence-form}

The first GWAS analysis on pubescence form found signals on chromosomes Gm04
and Gm15 with SNPs and \emph{k}-mers (Figure
\ref{pubescence_form_all_manhattan}). However, upon further analysis of LD
patterns, they appeared to be linked to the \emph{Pa1} locus (Figure
\ref{pubescence_form_all_ld}).


\subsection*{Maturity group}
\label{annexe-sv-gwas-maturity}

We did not detect any signals associated with maturity using our dataset
(Figure \ref{maturity_group_manhattan}). The use of maturity group as a proxy
for time to maturity instead of the number of days, combined with the limited
phenotypic variation of this trait in this collection of accessions (Table
\ref{MATURITY.GROUP_gwas_table}), likely explains our failure to detect
any signals.

\subsection*{Seed oil and protein content}
\label{sv-gwas-seed-oil-protein-content}

We conducted a GWAS analysis on seed oil content using data obtained from the
GRIN database. While several seed oil content quantitative trait loci (QTL)
have been reported due to high interest in this trait for soybean breeding
\shortcite{chaudhary2015}, we focused on two major QTL on chromosomes Gm15 and Gm20.
These loci have often been reported in \emph{G. max} and were notably observed by
\shortciteA{bandillo2015} on a collection of USDA lines using SoySNP50K chip
genotypes and GRIN phenotypic data. We identified a signal located on
chromosome Gm15 using the \emph{k}-mer approach only (Figure
\ref{oil_manhattan}). Although this signal did not overlap with the coordinates
found by \shortciteA{bandillo2017}, it was located only 2 kb away from it and 30 kb
away from Glyma.15g049200, the sugar transporter gene identified by
\shortciteA{zhang2020} as associated with that locus. Therefore, the significantly
associated \emph{k}-mers were not linked to the causal variation at that locus.

Similarly to seed oil content, we conducted a GWAS on seed protein content
using data obtained from the GRIN database. As is generally the case in
soybean given the negative correlation between seed oil and protein content,
the two major QTL considered above and reported by \shortciteA{bandillo2015} for seed
oil content are also major QTL for seed protein content. However, none of the
approaches detected a signal at this locus (Figure \ref{protein_manhattan}).

Our failure to detect signals overlapping the known genes at these two loci for
oil and protein is probably due to the fact that the oil and protein content
data that we used for GWAS were not obtained from orthogonal trials, which was
a clear limitation to this GWAS given our relatively small sample size. 

\clearpage

% Supplemental tables
\section*{Supplemental tables}
\label{annexe-sv-gwas-tables}

\clearpage

\gwastableone{FLOWER.COLOR}{flower color}
\gwastabletwo{PUBESCENCE.COLOR}{pubescence color}
\gwastabletwo{SEED.COAT.COLOR}{seed coat color}
\gwastabletwo{STEM.TERMINATION.TYPE}{stem termination type}
\gwastablethree{HILUM.COLOR}{hilum color}
\gwastabletwo{POD.COLOR}{pod color}
\gwastabletwo{PUBESCENCE.FORM}{pubescence form}
\gwastableone{PUBESCENCE.DENSITY}{pubescence density}
\gwastablethree{SEED.COAT.LUSTER}{seed coat luster}
\gwastableone{MATURITY.GROUP}{maturity group}

% Table with all signals
\begin{landscape} \footnotesize
	\begin{ThreePartTable}

		\begin{TableNotes}
		\item[a] Most significant -log\textsubscript{10}(\emph{p}) previously reported at this locus by the study in the Study column
		\item[b] Gene associated with the locus. A question mark following the identifier of the gene indicates that this is simply a candidate
			that has yet to be confirmed. A dash indicates that no candidate has been suggested yet. We have only included candidate
			genes that have been suggested in previous studies.
		\item[c] Most significant -log\textsubscript{10}(\emph{p}) reported from SNPs and indels at this locus. A dash indicates that no signal
			was detected from SNPs and indels at this locus. The same applies for other methods (SVs, \emph{k}-mers) in
			their respective columns.
		\item[d] The \emph{p}-value at the W1 locus using the \emph{k}-mers method was so small that it was numerically equal
			to zero. This value was therefore set to the -log\textsubscript{10} of the smallest value that could be represented
			in R using our machine.
		\end{TableNotes}

		\begin{longtable}{llcclcccl}

			\caption[Description of loci associated with the traits studied]%
			{Description of loci associated with the traits studied and -log\textsubscript{10}(\emph{p}) of significant
			signals detected in our study using three different genotype datasets (SNPs/indels, SVs, \emph{k}-mers).}\\
			\label{signals_table}\\

			\toprule
			Trait & Locus & Chromosome & -log\textsubscript{10}(\emph{p})\tnote{a} & Gene\tnote{b} & SNPs/indels\tnote{c} & SVs & \emph{k}-mers  & Study \\
			\midrule \endfirsthead

			\caption[]{Loci associated with the traits studied \emph{(continued)}}\\

			\toprule
			Trait & Locus & Chromosome & -log10(p) & Gene & SNPs/indels & SVs & k-mers  & Study \\
			\midrule
			\endhead

			\bottomrule
			Continued on next page \\
			\endfoot

			\bottomrule \insertTableNotes
			\endlastfoot
			\csvreader[head to column names,
			late after line=\\,
			late after last line=,
			before reading={\catcode`\#=12},
			after reading={\catcode`\#=6},
			/csv/separator=semicolon]%
			{../tables/signals_table.csv}{}{\Trait & \Locus & \Chromosome & \Pvalues & \Gene & \Platypus & \Paragraph & \Kmers & \Study}

		\end{longtable}

	\end{ThreePartTable}

\end{landscape}
\clearpage

% A table with the number of k-mers foundd by each analysis
\begin{table}
	\centering
	\caption[Number of k-mers per GWAS analysis]{Number of significant \textit{k}-mers identified per GWAS analysis}
	\label{kmer_count_table}

	\begin{threeparttable}\footnotesize

			\csvreader[head to column names,
			tabular = lrrr@{\hskip 3pt}l,
			table head = \toprule GWAS analysis\tnote{a} & Raw\tnote{b} & Filtered\tnote{c} & At known loci & (\%)\tnote{d} \\ \midrule,
			table foot=\bottomrule,
			before reading={\catcode`\#=12},
			after reading={\catcode`\#=6},
			/csv/separator=semicolon]%
			{../tables/kmer_count_table.csv}{}{\gwas & \raw & \filtered & \nsignals & (\percent)}


				\begin{tablenotes}
				\item[a] GWAS numbers refer to numbers in tables \ref{FLOWER.COLOR_gwas_table}
					to \ref{MATURITY.GROUP_gwas_table}.
				\item[b] Number of significant \textit{k}-mers in raw GWAS output.
				\item[c] Number of significant \textit{k}-mers after filtering for
					mapping quality, minimum count and redundancy.
				\item[d] Number of \textit{k}-mers overlapping known loci and
					percentage relative to number of filtered \textit{k}-mers.
				\end{tablenotes}

	\end{threeparttable}
\end{table}

\clearpage

% Supplemental figures
\section*{Supplemental figures}
\label{annexe-sv-gwas-figures}

% Figures mentioned in the main text

% Histogram with concordance rates
\begin{cfigure}
	\includegraphics[width = 6in]{concordance_histogram}
	\caption{Histogram of concordance rates between genotypes derived from WGS data and SoySNP50K genotypes for 385 soybean accessions.
	The vertical line indicates the 0.9 concordance threshold that was used for filtering out mismatching samples.}
	\label{concordance-histogram}
\end{cfigure}

\clearpage

% Figures for flower color
\manhattanplot{flower_color}{flower color}{\getvar{flower_color_nsamples}}
\signalplot{flower_color_W1}{flower color}{\emph{W1}}%
{Vertical dotted lines indicate the location of the \getvar{flower_color_W1_gene} gene associated with the locus.}
\geneplot{flower_color_W1}{flower color}{\emph{W1}}{\getvar{flower_color_W1_gene} gene}%
{Vertical dotted lines in panels (c) and (d) indicate the location of the causal SV at this locus.}
\ldplot{flower_color}{flower color}{\getvar{flower_color_ldkmers}}
%\kmerplot{flower_color_W1}

% Figures for pubescence color
\manhattanplot{pubescence_color_all}{pubescence color (first GWAS)}{\getvar{pubescence_color_all_nsamples}}
\signalplot{pubescence_color_all_T}{pubescence color (first GWAS)}{\emph{T}}%
{Vertical dotted lines indicate the location of the \getvar{pubescence_color_all_T_gene} gene associated with the locus.}
\geneplot{pubescence_color_all_T}{pubescence color (first GWAS)}{\emph{T}}{\getvar{pubescence_color_all_T_gene} gene}%
{Vertical dotted lines in panels (b) and (d) indicate the location of the causal indel at this locus.}
\kmerplot{pubescence_color_all_T}{\getvar{pubescence_color_all_T_gene}}{\emph{T}}{pubescence color (first GWAS)}{a causal indel}

\manhattanplot{pubescence_color_nogray}{pubescence color (second GWAS)}{\getvar{pubescence_color_nogray_nsamples}}
\signalplot{pubescence_color_nogray_Td}{pubescence color (second GWAS)}{\emph{Td}}%
{Vertical dotted lines indicate the location of the \getvar{pubescence_color_nogray_Td_gene} gene associated with the locus.}
\geneplot{pubescence_color_nogray_Td}{pubescence color (second GWAS)}{\emph{Td}}{\getvar{pubescence_color_nogray_Td_gene} gene}%
{Vertical dotted lines in panels (b) and (d) indicate the location of the causal SNP at this locus.}
%\kmerplot{pubescence_color_nogray_Td}

% Figures for seed coat color
\manhattanplot{seed_coat_color_all}{seed coat color (first GWAS)}{\getvar{seed_coat_color_all_nsamples}}
\signalplot{seed_coat_color_all_I}{seed coat color (first GWAS)}{\emph{I}}%
{Vertical dotted lines indicate the boundaries of the tandem duplication/inversion identified as the causal variant at this locus.}

\manhattanplot{seed_coat_color_greenyellow}{seed coat color (second GWAS)}{\getvar{seed_coat_color_greenyellow_nsamples}}
\signalplot{seed_coat_color_greenyellow_G}{seed coat color (second GWAS)}{\emph{G}}%
{Vertical dotted lines indicate the location of the \getvar{seed_coat_color_greenyellow_G_gene} gene associated with the locus.}
\geneplot{seed_coat_color_greenyellow_G}{seed coat color (second GWAS)}{\emph{G}}{\getvar{seed_coat_color_greenyellow_G_gene} gene}%
{Vertical dotted lines in panels (b) and (d) indicate the location of the causal SNP at this locus.}
%\kmerplot{seed_coat_color_greenyellow_G}

% Figures for stem termination
\manhattanplot{stem_termination_all}{stem termination type (first GWAS)}{\getvar{stem_termination_all_nsamples}}
\signalplot{stem_termination_all_Dt1}{stem termination type (first GWAS)}{\emph{Dt1}}%
{Vertical dotted lines indicate the location of the \getvar{stem_termination_all_Dt1_gene} gene associated with the locus.}
\geneplot{stem_termination_all_Dt1}{stem termination type (first GWAS)}{\emph{Dt1}}{\getvar{stem_termination_all_Dt1_gene} gene}%
{Vertical dotted lines in panels (b) and (d) indicate the location of a causal SNP at this locus.}
\kmerplot{stem_termination_all_Dt1}{\getvar{stem_termination_all_Dt1_gene}}{\emph{Dt1}}{stem termination type (first GWAS)}{a causal SNP}


% Figures for hilum color
\manhattanplot{hilum_color_blackbrown}{hilum color (second GWAS)}{\getvar{hilum_color_blackbrown_nsamples}}
\signalplot{hilum_color_blackbrown_R}{hilum color (second GWAS)}{\emph{R}}%
{Vertical dotted lines indicate the location of the \getvar{hilum_color_blackbrown_R_gene} gene associated with the locus.}
\geneplot{hilum_color_blackbrown_R}{hilum color (second GWAS)}{\emph{R}}{\getvar{hilum_color_blackbrown_R_gene} gene}%
{Vertical dotted lines indicate the location an indel (panels b and d) and a SNP (panel d) that are putatively causal at this locus.}
\kmerplot{hilum_color_blackbrown_R}{\getvar{hilum_color_blackbrown_R_gene}}{\emph{R}}{hilum color (second GWAS)}{a SNP and an indel that are putatively causal}

% Figures for pubescence density
\manhattanplot{pubescence_density}{pubescence density}{\getvar{pubescence_density_nsamples}}
\signalplot{pubescence_density_Ps}{pubescence density}{\emph{Ps}}%
{Vertical dotted lines indicate the boundaries of the causal CNV overlapping the \getvar{pubescence_density_Ps_gene} gene associated with the locus.}
\geneplot{pubescence_density_Ps}{pubescence density}{\emph{Ps}}{location of the causal CNV overlapping the \getvar{pubescence_density_Ps_gene} gene.}%
{The vertical dotted line in panel (d) indicates the location of the most significant \emph{k}-mer, which is associated with the causal CNV at this locus.
The shaded gray rectangles show the boundaries of the causal CNV.}

% Figures for seed coat luster
\signalplot{seed_coat_luster_dullshiny_B}{seed coat luster (third GWAS)}{\emph{B}}%
{Vertical dotted lines indicate the boundaries of the causal CNV associated with the locus.}
\geneplot{seed_coat_luster_dullshiny_B}{seed coat luster (third GWAS)}{\emph{B}}{location of the causal CNV}%
{The shaded gray rectangles show the boundaries of the causal CNV.}
\manhattanplot{seed_coat_luster_dullshiny}{seed coat luster (third GWAS)}{\getvar{seed_coat_luster_dullshiny_nsamples}}

% Figures for pod color
\manhattanplot{pod_color_all}{pod color (first GWAS)}{\getvar{pod_color_all_nsamples}}
\signalplot{pod_color_all_L1}{pod color (first GWAS)}{\emph{L1}}{}
\signalplot{pod_color_all_L2}{pod color (first GWAS)}{\emph{L2}}{}

% Figures for pubescence form
\manhattanplot{pubescence_form_all}{pubescence form (first GWAS)}{\getvar{pubescence_form_all_nsamples}}
\signalplot{pubescence_form_all_Pa1}{pubescence form (first GWAS)}{\emph{Pa1}}%
{Vertical dotted lines indicate the location of the \getvar{pubescence_form_all_Pa1_gene} gene that we suggest as a candidate for this locus.}
\geneplot{pubescence_form_all_Pa1}{pubescence form (first GWAS)}{\emph{Pa1}}{\getvar{pubescence_form_all_Pa1_gene} candidate gene putatively}%
{Vertical dotted lines in panels (b) and (d) indicate the location of two highly significant non-synonymous SNPs at this gene.}
\kmerplot{pubescence_form_all_Pa1}{\getvar{pubescence_form_all_Pa1_gene} candidate}{\emph{Pa1}}{pubescence form (first GWAS)}{a non-synonymous SNP}
\signalplot{pubescence_form_all_Pa2}{pubescence form (first GWAS)}{\emph{Pa2}}{}
\manhattanplot{pubescence_form_noerect}{pubescence form (second GWAS)}{\getvar{pubescence_form_noerect_nsamples}}

% Figures for corrected dry weight
\manhattanplot{corrected_dry_weight}{corrected dry weight (resistance to \emph{P. sojae})}{\getvar{corrected_dry_weight_nsamples}}
\signalplot{corrected_dry_weight_cdwGm15}{corrected dry weight (resistance to \emph{P. sojae})}{\emph{cdwGm15}}%
{Vertical dotted lines indicate the location of the \getvar{corrected_dry_weight_cdwGm15_gene} gene suggested by de Ronne et al. (2022) as associated with the locus.}
%\ldplot{corrected_dry_weight}

% Figures for novel loci
\manhattanplot{pod_color_blbr}{pod color (second GWAS)}{\getvar{pod_color_blbr_nsamples}}
\ldplot{pod_color_blbr}{pod color (second GWAS)}{\getvar{pod_color_blbr_ldkmers}}
\signalplot{pod_color_blbr_pdcGm15}{pod color (second GWAS)}{newly suggested \emph{pdcGm15}}{}

% Figure for the proportion of genome covered by only one read as a function of sequencing depth
\begin{cfigure}
	\includegraphics[width = 6.5in]{seqdepth}
	\caption[Proportion of the genome covered by one read versus sequencing depth]{%
		Proportion of the genome covered by only one read as a function
	of the sequencing depth of 363 Illumina whole-genome sequencing soybean
	samples used for GWAS in this study.}
	\label{seqdepth-figure}
\end{cfigure}

\clearpage

% Figures mentioned only in supplemental data

% Flower color

% Pubescence color
\ldplot{pubescence_color_all}{pubescence color (first GWAS)}{\getvar{pubescence_color_all_ldkmers}}

% Stem termination
\manhattanplot{stem_termination_sn}{stem termination type (second GWAS)}{\getvar{stem_termination_sn_nsamples}}
%\ldplot{stem_termination_sn}

% Hilum color
\manhattanplot{hilum_color_all}{hilum color (first GWAS)}{\getvar{hilum_color_all_nsamples}}
\signalplot{hilum_color_all_T}{hilum color (first GWAS)}{\emph{T}}%
{Vertical dotted lines indicate the location of the \getvar{hilum_color_all_T_gene} gene associated with the locus.}
\geneplot{hilum_color_all_T}{hilum color (first GWAS)}{\emph{T}}{\getvar{hilum_color_all_T_gene} gene}%
{Vertical dotted lines in panels (b) and (d) indicate the location of the causal indel at this locus.}
\kmerplot{hilum_color_all_T}{\getvar{hilum_color_all_T_gene}}{\emph{T}}{hilum color (first GWAS)}{a causal indel}
\signalplot{hilum_color_all_I}{hilum color (first GWAS)}{\emph{I}}%
{Vertical dotted lines indicate the boundaries of the tandem duplication/inversion identified as the causal variant at this locus.}

\manhattanplot{hilum_color_rbr}{hilum color (third GWAS)}{\getvar{hilum_color_rbr_nsamples}}

% Pod color

% Pubescence density

% Seed coat luster
\manhattanplot{seed_coat_luster_all}{seed coat luster (first GWAS)}{\getvar{seed_coat_luster_all_nsamples}}
\manhattanplot{seed_coat_luster_nointermediate}{seed coat luster (second GWAS)}{\getvar{seed_coat_luster_nointermediate_nsamples}}

\ldplot{seed_coat_luster_dullshiny}{seed coat luster (third GWAS)}{\getvar{seed_coat_luster_dullshiny_ldkmers}}

% Pubescence form
\ldplot{pubescence_form_all}{pubescence form (first GWAS)}{\getvar{pubescence_form_all_ldkmers}}


% Figures for maturity group
\manhattanplot{maturity_group}{maturity group}{\getvar{maturity_group_nsamples}}

% Figures for oil
\manhattanplot{oil}{seed oil content}{\getvar{oil_nsamples}}

% Figures for protein
\manhattanplot{protein}{seed protein content}{\getvar{protein_nsamples}}

% Bibliography
\bibliographystyle{apacite}              % style de la bibliographie
\bibliography{references}                   % production de la bibliographie

\end{document}

