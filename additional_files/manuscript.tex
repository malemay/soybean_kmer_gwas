\documentclass[12pt]{report}

\usepackage[utf8]{inputenc} % declare encoding as utf8
\usepackage{graphicx} % to enable \includegraphics
\usepackage{flafter} % to make sure floats appear after their position in text
\usepackage{pdflscape} % allow some pages to be in landscape orientation
\usepackage{layout} % for debugging purposes, looking at page layout
\usepackage[margin=1in]{geometry} % leaves more space for text body
\usepackage[hidelinks]{hyperref} % hyperlinks from list of figures/tables
\usepackage[labelfont=bf]{caption} % format captions and make links go to figure instead of caption
\usepackage{csvsimple} % read and display csv files as tables
\usepackage{booktabs} % prettier tables
\usepackage{threeparttable} % easy addition of footnotes below tables
\usepackage{textcomp} % makes the \textdegree command available
\usepackage{tocloft} % formatting the lists of tables and figures
\usepackage{xcolor} % enable text coloring
\usepackage{calc} % makes \widthof available to set the width of the last column
\usepackage{array} % using command \arraybackslash to restore the definition of \\ after \centering
\usepackage{longtable} % allow tables to span more than one page
\usepackage{threeparttablex} % threeparttable for longtable
\usepackage{capt-of} % put caption on next page
\usepackage{ccaption}
\usepackage{listings}
\usepackage{datatool} % allows fetching key-value pairs from a text file
\usepackage[sectionbib]{natbib}
\usepackage{chapterbib}

% Remove indent for new paragraphs
\setlength{\parindent}{0pt}

% Setting the separator to use with datatool
\DTLsetseparator{ = }

% Formatting the table of contents/subsection
\renewcommand{\contentsname}{Supplemental methods}

% Formatting the list of tables
\renewcommand{\listtablename}{Supplemental tables} % Write "Supplemental tables" instead of "List of tables"
\renewcommand\cfttabpresnum{Table } % Write "Table" before table number in list of tables
\setlength{\cfttabnumwidth}{6em} % Spacing before captions
\setlength{\cftbeforefigskip}{1ex} % Vertical spacing between entries

% Formatting the list of figures
\renewcommand{\listfigurename}{Supplemental figures} % Write "Supplemental figures" instead of "List of figures"
\renewcommand\cftfigpresnum{Figure } % Write "Figure" before fig number in list of figures
\setlength{\cftfignumwidth}{6em} % Spacing before captions
\setlength{\cftbeforetabskip}{1ex} % Vertical spacing between entries

\graphicspath{ {../figures/} } % tell Latex to look for figures in figures/

% Format figure and table numbers to follow S1, S2, ...
\renewcommand\thefigure{S\arabic{figure}} 
\renewcommand\thetable{S\arabic{table}} 

% Defining new environments as I will always want the figures centered
\newenvironment{cfigure}
	{\begin{figure} \centering}
	{\end{figure}}

\newenvironment{lsfigure}
	{\begin{landscape} \begin{figure} \centering}
	{\end{figure} \end{landscape}}

\setcounter{totalnumber}{1} % Allow a maximum of one float (table/figure) per page

\setlength{\parskip}{2ex}

% Defining a command to generate a table of phenotypic data for a single gwas
\newcommand{\gwastableone}[2]{
\begin{table}
	\caption[Phenotypic data used for the GWAS analysis of #2]%
	{Phenotype frequency and numerical coding of phenotypic data used for the GWAS analysis of #2.
	``NA'' indicates a missing value in the source data.
	Dashes indicate that this phenotype was not used in the GWAS analysis.}
	\label{#1_gwas_table}

	\centering
	\begin{threeparttable}\small

		\csvreader[head to column names,
		/csv/separator=semicolon,
		tabular = llc,
		table head = \toprule Phenotype & Frequency & GWAS \\ \midrule,
		table foot = \bottomrule]%
		{../tables/#1_gwas_table.csv}%
		{}%
		{\Value & \Frequency & \GWASa}

	\end{threeparttable}

\end{table}
\clearpage
}


% Defining a command to generate a table of phenotypic data for two gwas
\newcommand{\gwastabletwo}[2]{
\begin{table}
	\caption[Phenotypic data used for the GWAS analyses of #2]%
	{Phenotype frequency and numerical coding of phenotypic data used in two GWAS analyses of #2.
	``NA'' indicates a missing value in the source data.
	Dashes indicate that this phenotype was not used in the GWAS analysis.}
	\label{#1_gwas_table}

	\centering
	\begin{threeparttable}\small

		\csvreader[head to column names,
		/csv/separator=semicolon,
		tabular = llcc,
		table head = \toprule Phenotype & Frequency & GWAS \#1 & GWAS \# 2\\ \midrule,
		table foot = \bottomrule]%
		{../tables/#1_gwas_table.csv}%
		{}%
		{\Value & \Frequency & \GWASa & \GWASb}

	\end{threeparttable}

\end{table}
\clearpage
}

% Defining a command to generate a table of phenotypic data for three gwas
\newcommand{\gwastablethree}[2]{
\begin{table}
	\caption[Phenotypic data used for the GWAS analyses of #2]%
	{Phenotype frequency and numerical coding of phenotypic data used in three GWAS analyses of #2.
	``NA'' indicates a missing value in the source data.
	Dashes indicate that this phenotype was not used in the GWAS analysis.}
	\label{#1_gwas_table}

	\centering
	\begin{threeparttable}\small

		\csvreader[head to column names,
		/csv/separator=semicolon,
		tabular = llccc,
		table head = \toprule Phenotype & Frequency & GWAS \#1 & GWAS \# 2 & GWAS \# 3\\ \midrule,
		table foot = \bottomrule]%
		{../tables/#1_gwas_table.csv}%
		{}%
		{\Value & \Frequency & \GWASa & \GWASb & \GWASc}

	\end{threeparttable}

\end{table}
\clearpage
}

% Defining a command for manhattan plot figures
\newcommand{\manhattanplot}[3]{
\begin{cfigure}
	\includegraphics[width = 5.8in]{#1_manhattan}
	\caption[Manhattan plots for #2]{Manhattan plots generated from the GWAS analysis of #2 on #3 samples using four genotype datasets :
	(a) Platypus (SNPs and indels), (b) vg (SVs), (c) Paragraph (SVs), (d) \emph{k}-mers presence/absence.
	The x-axis shows the position along the reference assembly version 4 of Williams82.
	Each point shows the $-\log_{10} (p)$ associated with a particular marker or \emph{k}-mer.
	Horizontal dotted lines indicate the 5\% family-wise error-rate significance threshold determined from a randomization approach.
	Vertical dotted lines indicate the position of signals associated with the trait.
	Documented loci are colored according to whether they were found (green) or not (red) by a particular approach.
	The ``Gm'' prefix has been left out of chromosome names for simplicity.
	}
	\label{#1_manhattan}
\end{cfigure}

\clearpage}

% Defining a new command for signal plot figures
\newcommand{\signalplot}[4]{
\begin{cfigure}
	\includegraphics[width = 5.8in]{#1_signal}
	\caption[Zoomed-in Manhattan plots of the #3 locus]{Zoomed-in Manhattan plots of signals detected by the GWAS analysis of #2 at the #3 locus using four genotype datasets:
	(b) Platypus (SNPs and indels), (c) vg (SVs), (d) Paragraph (SVs), (e) \emph{k}-mers presence/absence.
	Panel (a) shows gene models over the genomic interval.
	Horizontal dotted lines indicate the 5\% family-wise error-rate significance threshold determined from a randomization approach.
	#4
	Gray shaded rectangles indicate the region delimited by the top 5\% (Paragraph and vg) or top 1\% (Platypus and \emph{k}-mers) associations in the signal region.
	In the case of Platypus, blue points denote markers used in the original analysis, whereas red points denote points that had originally been pruned but whose
	\emph{p}-values were computed after signal discovery.
	}
	\label{#1_signal}
\end{cfigure}

\clearpage}

% Defining a new command for gene plot figures
\newcommand{\geneplot}[5]{
\begin{cfigure}
	\includegraphics[width = 5.8in]{#1_gene}
	\caption[Zoomed-in Manhattan plots of the #3 locus gene]%
	{Zoomed-in Manhattan plots generated from the GWAS analysis of #2 at the #4 associated with the #3 locus using four genotype datasets:
	(b) Platypus (SNPs and indels), (c) vg (SVs), (d) Paragraph (SVs), (e) \emph{k}-mers presence/absence.
	#5
	Horizontal dotted lines indicate the 5\% family-wise error-rate significance threshold determined from a randomization approach.
	In the case of Platypus, blue points denote markers used in the original analysis, whereas red points denote points that had originally been pruned but whose
	\emph{p}-values were computed after signal discovery.
	Panel (a) shows gene models over the plotting interval.
	Exons are represented by rectangles whereas introns are represented by horizontal lines.
	Coding sequences are shown in blue and the direction of transcription is indicated by arrows.
	}
	\label{#1_gene}
\end{cfigure}

\clearpage}

% Defining a new command for k-mer plot figures
\newcommand{\kmerplot}[5]{
	\begin{cfigure}
		\includegraphics[width = 5.8in]{#1_kmers}
		\caption[Variants associated with significant \emph{k}-mers at the #3 locus]%
		{Identification of #5 underlying significant \emph{k}-mers at the #2 gene associated with the #3 locus for #4.
		(a) Gene model of #2. Exons are represented by rectangles whereas introns are represented by horizontal lines.
		Coding sequences are shown in blue and the direction of transcription is indicated by arrows.
		The red rectangle identifies the region that is zoomed-in in panel (b).
		(b) Nucleotide sequences of haplotypes observed in at least five samples across the dataset.
		Individual nucleotides are colored according to the $-\log_{10}(p)$ of the most significant \emph{k}-mer overlapping them.
		Dashes indicate gaps in haplotype sequence alignment whereas vertical lines indicate differences in sequence between two haplotypes.
		(c) Contingency table of the phenotypes and haplotypes observed in the dataset. Haplotypes correspond to those shown in panel (b).}
		\label{#1_kmers}
	\end{cfigure}

\clearpage}

% Defining a new command for LD plot figures
\newcommand{\ldplot}[3]{
	\begin{cfigure}
		\includegraphics[width = 5.8in]{#1_ld}
		\caption[LD plot of significant \emph{k}-mers for #2]{Pairwise LD among #3 significant \emph{k}-mers identified for #2.
		\emph{k}-mers are sorted along the y-axis according to their putative position along the reference assembly version 4 of Williams82,
		as identified by ``Gm'' chromosome labels.
		Sequences that lack a ``Gm'' label (top of the y-axis) represent unanchored scaffolds.
		\emph{k}-mers are represented in the same order along the x- and y-axis.
		The colored rectangles drawn below the x-axis represent the $-\log_{10} (p)$ of each \emph{k}-mer.}
		\label{#1_ld}
	\end{cfigure}

\clearpage}

% END OF PREAMBLE

\begin{document}

\chapter*{\textit{k}-mer-based GWAS enhances the discovery of causal variants and candidate genes in soybean}
\label{kmer-gwas}

Marc-André Lemay\textsuperscript{1,2,3},
Jonas A. Sibbesen\textsuperscript{4},
Maxime de Ronne\textsuperscript{1,2,3},
Richard Bélanger\textsuperscript{1,2,3},
François Belzile\textsuperscript{1,2,3}

\textsuperscript{1} Département de phytologie, Université Laval, Québec, QC, Canada

\textsuperscript{2} Institut de biologie intégrative et des systèmes, Université Laval, Québec, QC, Canada

\textsuperscript{3} Centre de recherche et d’innovation sur les végétaux, Université Laval, Québec, QC, Canada

\textsuperscript{4} Center for Health Data Science, University of Copenhagen, Denmark

\clearpage

\section*{Abstract}

Genome-wide association studies (GWAS) are powerful statistical methods
that detect associations between genotype and phenotype at genome scale. Despite
their power, GWAS frequently fail to pinpoint the causal variant or the gene
controlling a trait at a given locus in crop species. Assessing genetic variants beyond
single-nucleotide polymorphisms (SNPs) may alleviate this problem,
for example by including structural variants (SVs) that may have a greater functional impact.
In this study, we tested the potential
of SV- and \emph{k}-mer-based GWAS in soybean by applying these methods to 13
traits on which SNP-based GWAS had been performed previously.
We assessed the performance of each GWAS approach based on results at loci for which
the causal genes or variants were known from previous genetic studies. We found
that \emph{k}-mer-based GWAS was the most versatile approach and the best at pinpointing
causal variants or candidate genes based on the most significantly
associated \emph{k}-mers. Given the traits that we analyzed, SV-based GWAS did not add any value
compared to \textit{k}-mer-based GWAS and may not generally be worth the investment in time and
computational resources required to discover and genotype such variants.
Moreover, \textit{k}-mer-based analyses
identified promising candidate genes for loci related
to pod color, pubescence form, and resistance to the oomycete
\textit{Phytophthora sojae}. Despite promising results,
significant challenges remain regarding the downstream analysis of
\emph{k}-mer-based GWAS, notably in associating significant \emph{k}-mers with
sequence variation and ensuring that all variation types are adequately
detected from \emph{k}-mer occurrence patterns. Together, our results suggest
that coupling \emph{k}-mer- and SNP/indel-based GWAS may represent a
powerful approach for discovering candidate genes in crop species.

\section*{Introduction}
\label{sv-gwas-introduction}

Genome-wide association studies (GWAS) are analytical approaches that detect
statistical associations between phenotypic observations for a trait of
interest and the genotypes of variants distributed throughout the
genome \citep{tam2019}. This approach has been used for over 15 years
\citep{visscher2012} and has enabled significant advances in our understanding
of the genetic architecture of traits involved in human health
\citep[e.g.][]{wellcome2007gwas} and crop \citep{tibbs2021} and animal
\citep{ma2019} production, among others.

Despite their power and convenience for finding significantly associated loci,
GWAS analyses frequently fail to pinpoint the gene(s) associated with a trait and
the causal variant(s) involved \citep{tam2019}. One possible reason for this
is that many GWAS analyses do not include the causal variant itself in their variant dataset
\citep{tibbs2021}. This could be either because the analysis
is restricted to a particular type of variant, such as single-nucleotide
polymorphisms (SNPs), or because the set of markers/variants genotyped in the
association panel is limited by the technology being used
\citep[e.g.][]{bandillo2015, sonah2015}. Despite this limitation, variants
located near the gene controlling the trait often do appear statistically
associated with the phenotype through linkage disequilibrium (LD) with the
causal variant \citep{korte2013}.  A typical workflow then involves the
identification of candidate genes within the haplotype blocks defined by the
statistically associated markers and follow-up functional analyses to confirm
that a given gene controls the trait \citep[e.g.][]{wang2018, liu2020ps}.

The inclusion of structural variants (SVs) as genotype data in GWAS may partly overcome
this issue by providing a set of variants with a potentially high functional impact.
SVs include any variant involving a difference of at least 50 nucleotides, such as
deletions, insertions, or inversions \citep{ho2020}. Such variants are known to
have large phenotypic impacts by disrupting coding or regulatory sequences
\citep{marroni2014}, and have been identified as causal variants in traits such
as resistance to soybean cyst nematode \citep{cook2012}, aluminum tolerance in
wheat \citep{maron2013}, and branching in maize \citep{studer2011}.  SVs have
already been used in GWAS in studies with promising results
\citep[e.g.][]{zhang2015, akakpo2020, dominguez2020, liu2020-pangenome}, but
their inclusion in GWAS is the exception rather than the norm.

Despite their phenotypic impact, SVs are still largely understudied compared to SNPs
because of the difficulty of discovering and genotyping them with
accuracy. Comprehensive detection of SVs requires a combination of several
methods that often need to be tailored to the study species and data available
\citep{alkan2011, ho2020}. Even then, assessments of the performance of SV
discovery and genotyping generally reveal subpar sensitivity and precision
\citep[e.g.][]{cameron2019, chaisson2019, kosugi2019}. Moreover, despite recent
improvements in computational approaches for genotyping SVs
\citep[e.g.][]{siren2021, ebler2022}, such approaches remain relatively inaccurate
compared to SNP genotyping workflows.

Alternative approaches that allow one to assess a large spectrum
of variants in GWAS without genotyping SVs at population scale could be useful.
\cite{voichek2020} developed a \emph{k}-mer-based GWAS approach that may
represent such an alternative method. This approach relies on the presence or
absence of \emph{k}-mers observed in sequence reads instead of genotypes for use in
GWAS. Once a presence/absence table of \emph{k}-mers has been generated for the
population under study, this table can be used in association analyses
similarly to SNP or SV genotype calls. One advantage of using \emph{k}-mers is
that they can act as a molecular signature for any type of variant as long as
these variants result in presence/absence of \emph{k}-mers of a given length.
Another advantage is that \emph{k}-mers are not tied to a specific genomic
location and can thus be used to query associations between genotype and
phenotype at genomic locations that are not found in the reference genome.  A
few studies have applied this \emph{k}-mer-based GWAS approach
\citep[e.g.][]{tripodi2021, colque2021} and other similar approaches have been
developed \citep{rahman2018, he2021}, but similarly to the use of SVs in GWAS,
such approaches have not been widely adopted yet.

In this study, we assessed the potential of SV- and \emph{k}-mer-based GWAS in
a major crop, soybean (\emph{Glycine max}).  In order to test these methods, we
analyzed ten qualitative traits \citep{bandillo2017} and two quantitative
traits \citep{bandillo2015} that have been previously studied using a SNP array
developed for soybean \citep{song2013}. In addition, we analyzed a quantitative trait
(horizontal resistance to \emph{Phytophthora sojae}) that has been previously
studied using whole-genome sequencing (WGS) data \citep{deronne2022}.
Since the underlying genes and causal variants are already known for several of
these traits, they provide a good test case for the performance of GWAS methods.
The objectives of our study were to:

\begin{enumerate}
	\item Assess the potential of SV- and \emph{k}-mer-based GWAS approaches to
		pinpoint genes and causal variants associated with loci whose
		underlying genes or variants are already known.
	\item Identify potential candidate genes and/or causal variants at loci
		whose underlying genes or candidate variants are not yet known.
	\item Develop a set of computational tools for the downstream analysis
		of significantly associated \emph{k}-mers once these have been
		identified.
\end{enumerate}

\section*{Methods}
\label{sv-gwas-methods}

\subsection*{Sample selection and processing of sequencing data}
\label{sv-gwas-sequencing-data}

We selected 389 inbred \emph{G. max} accessions based on the availability of
Illumina WGS data and phenotypic data for resistance to the oomycete \emph{P.
sojae} \citep{deronne2022}.  We identified 741 SRA runs corresponding to
Illumina paired-end sequencing data for those 389 accessions \citep[data from][]{zhou2015, valliyodan2016, fang2017, bayer2021}
and extracted paired-end reads using the fastq-dump command v. 2.9.6
(\citeauthor[][\url{https://github.com/ncbi/sra-tools}]{sratoolkit}) with the
option \texttt{-\-split-3}.
Reads that were not paired in the raw sequencing data were not used
for downstream analyses.  The runs downloaded from the SRA and associated
metadata are listed in Additional file 1.

We first filtered the sequencing data for quality and the presence of sequencing
adapters using bbduk from BBtools v. 38.25
(\citeauthor[][\url{https://sourceforge.net/projects/bbmap/}]{bbtools}). Reads
were then mapped to the reference assembly version 4 of \emph{G. max}
cultivar Williams82 \citep{valliyodan2019} concatenated with chloroplast
and mitochondrion sequences obtained from SoyBase \citep{grant2010}. We used
bwa-mem v. 0.7.17-r1188 \citep{li2009-bwa} with default parameters to map the
reads in paired-end mode. Reads that were left unpaired following adapter- and
quality-trimming were aligned separately in single-end mode with bwa-mem and
merged with mapped paired-end reads using samtools \texttt{merge}
v. 1.12 \citep{li2009-samtools}.  We then added read groups identifying
individual SRA runs to the BAM files and merged the reads belonging to the same
accession using bamaddrg
(\citeauthor[][\url{https://github.com/ekg/bamaddrg}]{bamaddrg}). We used the
resulting BAM files for all downstream analyses requiring aligned reads.
Metadata on mapping depth of aligned data can be found in Additional file 2.

\subsection*{Discovery and genotyping of SVs}
\label{sv-gwas-sv-discovery-genotyping}

Our SV analysis pipeline consisted of separate SV discovery and genotyping steps.
The \emph{discovery} step identified a set of candidate SVs using various methods,
whereas the \emph{genotyping} step determined the genotype
of the candidate SVs for all accessions in the population from the mapped Illumina reads. We used three different
approaches in the SV discovery step to generate a set of candidate variants:

\begin{itemize}
	\item We used Illumina WGS data from the 389 accessions mentioned above
		to call SVs in the population following a pipeline outlined
		by \cite{lemay2022}. In brief, we combined information from
		the SV calling programs AsmVar \citep{liu2015}, Manta \citep{chen2016},
		smoove \citep{pedersen2019} and SvABA \citep{wala2018} to generate
		a set of candidate SVs from the mapped Illumina data.
	\item We used SVs discovered by \cite{lemay2022} from Oxford Nanopore
		sequencing data of 17 Canadian soybean cultivars.
	\item We called SVs from 26 high-quality genome assemblies published by
		\cite{liu2020-pangenome}, as well as that of the cultivar Zhonghuang 13 (ZH13),
		\textit{G. soja} accession W05, and \textit{G. max} cultivar Lee.
		We used methods similar to those of \cite{liu2020-pangenome} to call
		SVs from the comparison of these assemblies to that of Williams82.
\end{itemize}

The SVs called using each of these approaches were merged using SVmerge
\citep{wong2010} and the resulting set of candidate SVs was used for
genotyping. We used Paragraph \citep{chen2019} to genotype the candidate SVs
from the mapped Illumina reads for all accessions. Detailed methods regarding
the SV discovery and genotyping steps can be found in Supplemental data.

\subsection*{Discovery and genotyping of SNPs and indels}
\label{sv-gwas-snp-discovery}

We used Platypus v. 0.8.1.1 \citep{rimmer2014} to call SNPs and indels from the
mapped reads of the 389 accessions. The 21.1 M SNPs and
indels from the raw output of Platypus were filtered to keep only those with:

\begin{enumerate}
	\item \texttt{FILTER} field set to \texttt{PASS}
	\item minor allele frequency (MAF) $\geq$ 0.02
	\item fraction of missing genotypes $\leq$ 0.5
	\item heterozygosity rate $\leq$ 0.2.
\end{enumerate}

Moreover, we pruned the dataset using the \texttt{-\-indep-pairwise} option of
PLINK v. 1.90b5.3 \citep{purcell2007} with a window of 1,000 markers, a
step of 100 markers and an $r^2$ LD threshold of 0.9. This pruning step was
implemented to reduce the computational requirements of the GWAS analysis by
removing co-segregating markers. Markers located on unanchored scaffolds were
also removed. The resulting dataset was used for GWAS and comprised 773,060
SNPs and 151,570 indels.

\subsection*{Computing the presence/absence table of \emph{k}-mers}
\label{sv-gwas-kmer-pav}

We used the approach outlined by \cite{voichek2020} for \emph{k}-mer-based GWAS
and their recommendations on the
associated GitHub page (v. 0.2-beta, \url{https://github.com/voichek/kmersGWAS}) to generate
a presence/absence \emph{k}-mer table for use in GWAS.  The first step of
the approach involved counting all \emph{k}-mers of length 31 (31-mers) present
in the trimmed (FASTQ) reads for each accession using KMC3 v. 3.2.1
\citep{kokot2017}. \emph{k}-mers were counted twice, the first time using their
canonized form (the first of either the observed \textit{k}-mer or its reverse
complement in lexicographical order) and the second time using the observed
\emph{k}-mers themselves (non-canonized form). The two sets of \textit{k}-mers
were then combined and filtered by
keeping only those seen in at least 5 accessions, and observed in both canonized
and non-canonized form in at least 20\% of the accessions in which they are found.
We then generated a table indicating the presence or absence of each
\emph{k}-mer in all accessions. This table was used to compute a kinship
table for use in GWAS and was also used directly as input genotypes for the
GWAS analyses. We used \emph{k}-mers with a MAF $\geq$ 0.02 for both the
computation of kinship and GWAS.

\subsection*{Phenotypic data}
\label{sv-gwas-phenotypic-data}

We analyzed ten qualitative traits previously studied by \cite{bandillo2017}: flower
color, pubescence color, seed coat color, stem termination type, hilum color,
pod color, pubescence form, pubescence density, seed coat luster, and maturity
group.  We obtained phenotypic data for these traits from the GRIN database
(\url{https://npgsweb.ars-grin.gov/gringlobal/search}) by querying the database
using the PI identifiers of the 389 accessions used.  Four out of the 389
accessions did not match the GRIN database and were therefore not used for the
analysis of qualitative traits. The phenotypes of qualitative traits were recoded
to numerical values for GWAS analysis following methods similar to those of
\cite{bandillo2017}.  For some traits, we conducted more than one GWAS analysis
using targeted subsets of the observed phenotypes in order to focus on specific
loci. The phenotypes and numerical values used for all analyses as well as the
number of observations in the dataset that we used for GWAS are listed in
Tables \ref{FLOWER.COLOR_gwas_table} to \ref{MATURITY.GROUP_gwas_table}.

We analyzed three quantitative traits in addition to the ten qualitative traits
mentioned above. Seed oil and protein content were retrieved from the GRIN
database. In cases where more than one value was listed for a given accession,
we computed the average of those values and used it for GWAS analyses. As a third quantitative trait, we
used the horizontal resistance to the oomycete \textit{P. sojae} (corrected dry
weight, CDW) as recently assessed in a hydroponic assay \citep{deronne2022}.
Our analysis included 340 accessions whose phenotypic data
was already published by \cite{deronne2022} as well as 49 additional accessions for
which sequencing data was not yet available when their study was conducted.
Because of ambiguity regarding the identity of one of the accessions (HN019), this accession
was dropped from all GWAS analyses. Similarly to seed oil and
protein content, the CDW values for some accessions for which more than one
observation was available were averaged.

Because of clear discrepancies between the observed haplotypes and the reported
phenotypes for some simple traits such as flower color and seed coat color, we
suspected possible errors in the identity of some of the accessions. To
investigate this, we compared the SNP genotypes derived from the SoySNP50K
array and our WGS data at over 32,000 SNPs, and identified 24 accessions with <
90\% concordance between the two genotype datasets (see Supplemental data for
detailed methods). These samples were removed from the dataset in addition to another
accession that exhibited an atypical GC content suggesting contamination.
Following these filtering steps and the removal of HN019, 363 accessions
remained for GWAS analyses. The concordance between WGS data and SoySNP50K
genotypes is included in Additional file 2 and averaged 98.3\% among retained
lines. The phenotypic data used in this study can be found in Additional file 3.

\subsection*{GWAS analyses}
\label{sv-gwas-gwas-methods}

We used the GAPIT3 R package v. 3.1.0 \citep{wang2021} for conducting GWAS on
the SNP/indel and SV datasets described above. SVs and indels required some additional
preprocessing steps to make variant
representation suitable for downstream analyses \citep{lemay2022gwas}. Briefly, we recoded all
variants as SNPs prior to converting the VCF files to Hapmap diploid format
using the TASSEL command-line tools \citep{bradbury2007}; variant IDs enabled
retrieval of the proper metadata for each variant after GWAS.

We ran GAPIT using a mixed linear model (MLM) with 9 principal components and the VanRaden
algorithm for computing the kinship matrix. We used an MLM model for
consistency with the statistical model used by the \textit{k}-mer-based
approach.  We used a randomization approach to determine the 5\% family-wise
error-rate threshold for each GWAS analysis as described in \cite{voichek2020}.
To compute this threshold, we permuted the phenotypic observations 100 times
and computed a GWAS on these permuted phenotypes to obtain a distribution of
top \emph{p}-values under the null hypothesis. The fifth most significant of
the top 100 \emph{p}-values was used as the significance threshold for
inference, i.e. markers with \emph{p}-values lower than this threshold were
considered significant.

We conducted GWAS based on \emph{k}-mers using the method developed by
\cite{voichek2020}. In brief, the
analysis was conducted in two steps. The first step used an approximate model
to identify 1 million potentially significant \emph{k}-mers. These were then used in
the second step as input to an exact model implemented in GEMMA \citep{zhou2012}, resulting
in a \emph{p}-value for each \emph{k}-mer. We
used the kinship computed from the \emph{k}-mers to correct for relatedness
between accessions. The program automatically outputs a list of \emph{k}-mers
that passed a 5\% family-wise error-rate threshold determined by a
randomization approach, which we considered significant for the purposes of
downstream analyses.

\subsection*{Analysis of significant \emph{k}-mers}
\label{sv-gwas-kmers-analysis-methods}

\emph{k}-mers are not intrinsically associated with a particular genomic region
and a tailored analysis was therefore required to associate them with genomic
coordinates. To do so, we queried the reads (in BAM format) of all accessions for matches
to significant \emph{k}-mers for a given trait. Because this is a
computationally demanding operation (there were typically tens of thousands of
significant \emph{k}-mers per trait), we developed a C program called katcher
(\url{https://github.com/malemay/katcher}) that
uses the htslib library \citep{bonfield2021} for efficient reading and
writing of BAM files. katcher and its associated utilities allow for efficient
retrieval and annotation of mapped reads containing any \emph{k}-mer from a set
of interest.

Once we retrieved all the reads containing significant \emph{k}-mers, we next
needed to gather the mapping information contained in those reads for the
purposes of identifying contiguous signals and comparing them to annotations. To
simplify the amount of information to process in downstream analyses, we
considered only the most significant \emph{k}-mer contained in a given read and
the position where this \emph{k}-mer was most often observed in
the dataset. Moreover, we limited the analysis to reads with a minimum mapping
quality of 20 and to \emph{k}-mers that were observed at least 10 times in at
least one accession. The latter was done to remove spurious \emph{k}-mers that may originate
from sequencing errors. All downstream analyses (Manhattan plots,
identification of signals, analysis of genes located near significant
\emph{k}-mers, etc.) used this processed set of significant \emph{k}-mers.

\subsection*{Downstream analyses of significant association signals}
\label{sv-gwas-methods-downstream}

We used contiguous regions of significant associations, which we called
``signals'', as a basis to compare our results with those obtained by previous
studies and to query the genome for candidate genes. We defined the boundaries
of those signals by grouping any significant markers or \emph{k}-mers that
occurred within 250 kb of one another.  We did consider the length of deletions
or \emph{k}-mers in defining the distance between significant associations.

For the GWAS analyses computed from Platypus genotype calls, we retrieved all
markers that had initially been pruned in a region ranging from 50 kb
downstream to 50 kb upstream of significant signals and computed their
\emph{p}-values by running GAPIT using the same parameters as noted above. The
\emph{p}-values thus computed on markers that were initially left out were not
used for initial signal discovery, but provided finer resolution when analyzing
the results of the Platypus GWAS analyses. Indeed, some markers that had not
been included in the original GWAS may be the causal variant or simply yield
more significant \emph{p}-values than those initially used.

To help with the interpretation of results and the identification of candidate
genes, we defined regions of highly significant associations within each signal
by delimiting a range of coordinates containing the top 5\% most significant
associations (Paragraph) or 1\% most significant associations (Platypus
and \emph{k}-mers) found within the signal. In cases where such a fraction
represented fewer than two markers or \emph{k}-mers, the highly significant
region was simply delimited by the two most significant associations.

Because previous studies \citep{bandillo2015, bandillo2017} used assembly version 1 of
Williams82, we converted the genomic coordinates of their signals to assembly version 4
using the same method that we used for converting
SoySNP50K positions (see Supplemental data). For the analysis of
qualitative traits, we only compared our results to signals at named classical
loci and ignored previously reported minor signals with \emph{p}-values that
just barely made it above the significance threshold.

\subsection*{Linking significant \emph{k}-mers to variants}
\label{sv-gwas-methods-linking-kmers}

We used the reads retrieved by katcher and an assembly-based approach for
identifying the variants associated with a given set of \emph{k}-mers. In
addition to the reads themselves, we also retrieved their paired reads even
though they may not have matched a significant \emph{k}-mer in the first place.
These read pairs were used for \emph{de novo} assembly with SPAdes v. 3.15.4
\citep{spades} using the \texttt{--careful} parameter. The assemblies were done
individually for each accession and then aligned to the Williams82 reference
assembly using bwa mem with default parameters.  Aligned assembled sequences
that overlapped loci of interest were realigned using multiple sequence
alignment with the MAFFT program v. 7.475 \citep{katoh2002} for identifying and
visualizing the haplotypes at these loci. The identification of variants
supported by significant \textit{k}-mers was based entirely on visual analysis
of haplotype alignments. We only considered haplotypes occurring at least five
times in the population in such analyses in order to leave out potential
assembly errors. In a single case (\emph{Dt1} locus), assembly at the locus
succeeded for a very low number of accessions (26). In this case, we instead used
consensus sequences from the reads aligned with bwa for comparing the
haplotypes across accessions. 

\subsection*{Computation of LD between \emph{k}-mers}
\label{sv-gwas-ld-kmers}

In some cases, the analysis of \emph{k}-mers revealed signals at previously
undocumented loci or found a very large number of associations throughout the
genome. To shed light on these situations, we computed the pairwise LD between
\emph{k}-mers in order to identify \emph{k}-mers that co-segregated and
likely corresponded to a single locus. Because of the sheer number of pairwise
comparisons that needed to be made for some traits (> 20,000 significant
\emph{k}-mers kept even after the filtering steps mentioned above), we limited
the number of \emph{k}-mers used for LD calculation to 1,500. For traits where
more than 1,500 \emph{k}-mers were found, we computed LD on a subsample of
these by selecting the 500 \emph{k}-mers with the most significant
\emph{p}-values and sampling the remaining 1000 \emph{k}-mers randomly. This
random sampling was done with a probability inversely proportional to the
number of \emph{k}-mers matching a given chromosome or scaffold such that
\emph{k}-mers were sampled from the whole genome and not simply from the most
significant locations.

\subsection*{Software used}
\label{sv-gwas-software-used}

Unless otherwise noted, all analyses were conducted using R v. 4.2.0
\citep{r2022} and Bioconductor v. 3.15 \citep{huber2015}. We used Bioconductor
packages Biostrings v. 2.64.0 \citep{pages2022}, GenomicRanges v. 1.48.0
\citep{lawrence2013}, Rsamtools v. 2.12.0 \citep{rsamtools2022}, rtracklayer v.
1.56.0 \citep{lawrence2009} and VariantAnnotation v. 1.42.1
\citep{obenchain2014}.  We gathered several functions used for the downstream
analysis of GWAS results into a package called gwask which is available on
GitHub (\url{https://github.com/malemay/gwask}).

\section*{Results and discussion}
\label{sv-gwas-results-discussion}

We analyzed a total of 13 traits using three different GWAS approaches, one
based on SNP and indel genotypes (Platypus), one based on SV genotypes (Paragraph),
and one based on the presence/absence of \textit{k}-mers.
Because two or three different GWAS analyses were conducted for some of the traits, a
total of 22 GWAS analyses were performed using each of the all approaches.
Given the large volume of data that this represents, we will focus here on the
most noteworthy results. Readers interested in detailed results for loci not
discussed here are referred to Supplemental data. In particular, table \ref{signals_table}
provides a summary of the \textit{p}-values obtained by each approach at each
of the loci considered in this study.

\subsection*{Loci with known genes or causal mutations}
\label{sv-gwas-results-cloned-genes}

We first wanted to assess the performance of the various GWAS
approaches by analyzing the results at loci for which the genes have
already been cloned. These loci provide interesting test cases because
the expected results (causal genes or variants) are known. Table \ref{loci-table} shows
a summary of the results discussed below.

% Table 1 with results on all loci with cloned genes
\begin{landscape}
	\begin{table}
		\caption[Performance of three GWAS approaches at detecting causal genes/variants]%
		{Performance of three GWAS approaches at detecting causal genes or variants at cloned loci.}
		\label{loci-table}

		\footnotesize
		\centering
		\begin{threeparttable}

			\csvreader[head to column names,
			tabular = llllccc,
			table head = \toprule Trait & Locus & Gene & Causal variant & Platypus\tnote{a} & Paragraph\tnote{b} & \textit{k}-mers \\ \midrule,
			table foot = \bottomrule]%
			{../tables/loci_table.csv}%
			{}%
			{\Trait & \Locus & \Gene & \Causal & \platypus & \paragraph & \kmers}

			\begin{tablenotes}
			\item[a] Platypus genotypes SNPs and indels
			\item[b] Paragraph genotypes SVs
			\item[c] Locus not discussed in main text. See Supplemental data for detailed results.
			\item[]- The approach detected no signal overlapping the gene.
			\item[]* The signal found at this locus overlaps the gene.
			\item[]** The region defined by the top 5\% (Paragraph) or 1\% (Platypus and \textit{k}-mers) variants or \textit{k}-mers overlaps the gene.
			\item[]*** The most significantly associated marker or \textit{k}-mer is the causal variant.
			\end{tablenotes}

		\end{threeparttable}
	\end{table}
\end{landscape}

\subsubsection*{Flower color -- \textit{W1} locus}
\label{sv-gwas-main-results-flower-color-l1}

\cite{zabala2007} identified the flavonoid 3’5’-hydroxylase (F3’5’H) gene
Glyma.13g\-072100 as associated with the \textit{W1} locus for flower color. In
Williams82 and other accessions with white flowers, a 65-bp insertion into the
third exon of the gene results in a premature stop codon that renders the
F3’5’H enzyme non-functional and prevents pigmentation of the flower. All
GWAS approaches detected a signal overlapping the gene at this locus (Figures
\ref{flower_color_manhattan}, \ref{flower_color_W1_signal}). The most
significant SV and \textit{k}-mers corresponded to the known causal variant
(Figures \ref{flower-color-main-figure}a-b and \ref{flower_color_W1_gene}).
In particular, the assembly of the reads containing the most significant \textit{k}-mers
and the alignment of the resulting haplotypes showed that these \textit{k}-mers
directly tagged the causal variant (Figure \ref{flower-color-main-figure}c-e).
As the catalogue of variants analyzed by Platypus contains only SNPs and small
indels, variants located in the gene proved to be among the most closely
associated, but the causal variant could not be captured given the large size
of this variant.

The \textit{k}-mer GWAS analysis also identified presumably spurious significant
\textit{k}-mers scattered throughout the genome (Figure
\ref{flower_color_manhattan}). An analysis of the pairwise LD between a subset
of those \textit{k}-mers showed that most \textit{k}-mers formed a single LD block
linked to the \textit{W1} locus (Figure \ref{flower_color_ld}). A small LD block
occurred on chromosome Gm09, however it was in moderate LD with the one on chromosome
Gm13 and therefore probably does not represent a separate locus.

We should note that when we first analyzed the results of the GWAS on flower
color, we observed that two accessions bearing the haplotype typically
associated with purple flowers were noted as having white flowers. Upon further
inspection, we found that the most likely cause of this discrepancy was a
mismatch between the genotype data and the identity of the accession for which
the phenotypic data was obtained. Indeed, when we compared SNP calls made
directly from the WGS data to those made using the SoySNP50K array, we observed
that these two accessions had low concordance rates of only 62\% and 70\%. To
ensure that such mismatches would not affect our GWAS analyses, we computed the
concordance between genotype calls made from WGS data and the SoySNP50K array
for all samples (see the Methods section) and
excluded 24 accessions for which the concordance rate was below 90\%.

% Figure 1 with the results on flower color at locus W1
\begin{figure}
	\centering
	\includegraphics[width = 5in]{flower_color_W1_main_figure.png}
	\caption[Results of SV- and \textit{k}-mer-based GWAS at the
	\textit{W1} locus for flower color]{(Caption on the following page.)}
	\label{flower-color-main-figure}
\end{figure}

\begin{figure}
	\contcaption{Results of SV- and \textit{k}-mer-based GWAS at the
	\textit{W1} locus for flower color. (a) Genome-wide Manhattan plots.
	Horizontal dotted lines indicate the significance
	threshold whereas vertical dotted lines indicate the location of the
	\textit{W1} locus. (b) Zoomed-in Manhattan plots at the \textit{W1} locus.
	The vertical dotted lines indicate the location of the
	Glyma.13g072100 gene associated with this locus. The shaded gray
	rectangles show the regions delimited by the top 5\% (SVs) and
	1\% (\textit{k}-mers) associations.  (c) Gene model of Glyma.13g072100.
	Rectangles represent exons, colored rectangles represent coding
	sequences, and arrows indicate the direction of transcription. The red
	rectangle highlights the region shown in panel (d). (d) Pairwise
	alignment of the two haplotype sequences in the population at the
	\textit{W1} locus. Nucleotides are
	colored according to the \textit{p}-value of the most significant
	\textit{k}-mer overlapping them. Vertical dotted lines between haplotypes
	indicate sequence differences, whereas dashes indicate gaps in the
	alignment. (e) Contingency table of the phenotypes and haplotypes found
	in the population at the \textit{W1} locus.}
\end{figure}

\subsubsection*{Pubescence color -- \textit{T} locus}
\label{sv-gwas-main-results-pubescence-color-t}

% Pubescence color - T locus
\cite{zabala2003} identified Glyma.06g202300 as the gene associated with the
\textit{T} locus for pubescence color and found a 1-bp deletion as
a putative causal variant at this locus. For all three GWAS approaches, we
found a signal overlapping this gene (Figure
\ref{pubescence_color_all_manhattan}). In
particular, the regions defined by the top 1\% most associated variants identified by the
Platypus and \textit{k}-mer approaches included Glyma.06g202300 (Figure
\ref{pubescence_color_all_T_signal}). While the most significant
\textit{k}-mers did occur within the sequence of this gene, they did not
correspond to the causal variant (Figure \ref{pubescence_color_all_T_gene}).
Still, the fourth and fifth most strongly associated \textit{k}-mers
corresponded to the causal variant (Figures \ref{pubescence_color_all_T_gene},
\ref{pubescence_color_all_T_kmers}). We
observed that not all accessions with gray pubescence harbored the deleted
nucleotide (Figure \ref{pubescence_color_all_T_kmers}c), which could indicate either that other causal variants are resulting in
a non-functional F3’H in this population, or that the causal
variant lies elsewhere. As for Platypus, the most significant
variant within the sequence of Glyma.06g202300 was the causal
variant, however it was not the most significantly associated variant overall
(Figures \ref{pubescence_color_all_T_signal},
\ref{pubescence_color_all_T_gene}). Paragraph did not detect the causal
variant at this locus given that this approach does not target small indels.

\subsubsection*{Pubescence color -- \textit{Td} locus}
\label{sv-gwas-main-results-pubescence-color-td}

% Pubescence color - Td locus
We conducted a separate GWAS analysis contrasting only accessions with tawny and light
tawny pubescence because the \textit{Td} locus is known to control
these differences in color. Accordingly, we
detected signals in this region using all three approaches, but only Platypus
and the \textit{k}-mers identified a signal that overlapped Glyma.03g258700
(Figures \ref{pubescence-color-main-figure}a-b,
\ref{pubescence_color_nogray_manhattan}, \ref{pubescence_color_nogray_Td_signal}).
While the most significant \textit{k}-mer identified the causal variant documented
at this locus \citep{yan2020}, it was the fourth most significantly associated variant
for the Platypus genotype calls (Figures
\ref{pubescence-color-main-figure}b-d, \ref{pubescence_color_nogray_Td_gene}).
Interestingly, some of the accessions bearing the haplotype associated with tawny pubescence
displayed light tawny pubescence, suggesting that there may be more than one causal
variant at this locus (Figure \ref{pubescence-color-main-figure}e).

% Figure 2 with the results on pubescence color at locus Td
\begin{figure}
	\centering
	\includegraphics[width = 5in]{pubescence_color_nogray_Td_main_figure.png}
	\caption[Results of SNP/indel- and \textit{k}-mer-based GWAS at the
	\textit{Td} locus for pubescence color]{(Caption on the following page.)}
	\label{pubescence-color-main-figure}
\end{figure}

\begin{figure}
	\contcaption{Results of SNP/indel- and \textit{k}-mer-based GWAS at the
	\textit{Td} locus for pubescence color. (a) Genome-wide Manhattan plots.
	Horizontal dotted lines indicate the significance
	threshold whereas vertical dotted lines indicate the location of the
	\textit{Td} locus. (b) Zoomed-in Manhattan plots at the \textit{Td} locus.
	The vertical dotted lines indicate the location of the
	Glyma.03g258700 gene associated with this locus. The shaded gray
	rectangles show the regions delimited by the top 1\% associations.
	In the case of SNP/indels, blue points denote markers used in the
	original analysis, whereas red points denote points that had originally
	been pruned but whose \emph{p}-values were computed after signal
	discovery.
	(c) Gene model of Glyma.03g258700. Rectangles represent exons, colored
	rectangles represent coding sequences, and the arrow indicates the
	direction of transcription. The red rectangle highlights the region
	shown in panel (d). (d) Pairwise alignment of the two haplotype
	sequences in the population at the \textit{Td} locus. Nucleotides are
	colored according to the
	\textit{p}-value of the most significant \textit{k}-mer overlapping
	them. The vertical dotted line between haplotypes indicates the location of
	the causal SNP at this locus. (e) Contingency table of
	the phenotypes and haplotypes found in the population at the
	\textit{Td} locus.}
\end{figure}

\subsubsection*{Seed coat color - \textit{I} locus}
\label{sv-gwas-main-results-seed-coat-color-i}

% Seed coat color - I locus
The causal variant at the \textit{I} locus for seed coat color is a complex SV
that takes the form of a 10.91-kb inverted duplication of three chalcone
synthase (CHS) genes \citep[CHS1, CHS3 and CHS4;][]{tuteja2008}. The presence of
this inverted duplication results in the silencing of all CHS genes in the seed
coat specifically and thus in the absence of seed coat pigmentation
\citep{tuteja2009}. While we did identify signals overlapping this causal variant
using all three GWAS approaches (Figures \ref{seed_coat_color_all_manhattan},
\ref{seed_coat_color_all_I_signal}), none of the approaches identified
the causal variant. This failure to detect the causal variant is likely due
to its complexity, which makes it difficult to detect and genotype it. While it
is likely that at least some of the significantly associated \textit{k}-mers
actually are derived from this causal variant, the lack of systematic methods for
linking \textit{k}-mers to sequence variation hindered our ability to do so.

\subsubsection*{Seed coat color - \textit{G} locus}
\label{sv-gwas-main-results-seed-coat-color-g}

% Seed coat color - G locus
We conducted a second GWAS analysis contrasting accessions with yellow and
green seed coats in order to target the \textit{G} locus. Accordingly, this
analysis found strong signals overlapping the \emph{G} locus using all three
approaches (Figure \ref{seed_coat_color_greenyellow_manhattan}).
\cite{wang2018} identified Glyma.01g198500 as the gene associated with
this locus and found an A>G SNP affecting transcript splicing as the causal
variant. Consistently with their results, we identified the two
most significant \emph{k}-mers and the second most significant marker from the
Platypus analysis as corresponding to that causal variant (Figures
\ref{seed-coat-color-main-figure},
\ref{seed_coat_color_greenyellow_G_signal},
\ref{seed_coat_color_greenyellow_G_gene}). Unsurprisingly, Paragraph
did not identify the causal variant given its type.

% Figure 3 with the results on pubescence color at locus G
\begin{figure}
	\centering
	\includegraphics[width = 5in]{seed_coat_color_greenyellow_G_main_figure.png}
	\caption[Results of SNP/indel- and \textit{k}-mer-based GWAS at the
	\textit{G} locus for seed coat color]{(Caption on the following page.)}
	\label{seed-coat-color-main-figure}
\end{figure}

\begin{figure}
	\contcaption{Results of SNP/indel- and \textit{k}-mer-based GWAS at the
	\textit{G} locus for seed coat color. (a) Genome-wide Manhattan plots.
	Horizontal dotted lines indicate the significance
	threshold whereas vertical dotted lines indicate the location of the
	\textit{G} locus. (b) Zoomed-in Manhattan plots at the \textit{G} locus.
	The vertical dotted lines indicate the location of the
	Glyma.01g198500 gene associated with this locus. The shaded gray
	rectangles show the regions delimited by the top 1\% associations.
	In the case of SNP/indels, blue points denote markers used in the
	original analysis, whereas red points denote points that had originally
	been pruned but whose \emph{p}-values were computed after signal
	discovery.
	(c) Gene model of Glyma.01g198500. Rectangles represent exons, colored
	rectangles represent coding sequences, and the arrows indicate the
	direction of transcription. The red rectangle highlights the region
	shown in panel (d). (d) Pairwise alignment of the two haplotype
	sequences in the population at the \textit{G} locus. Nucleotides are
	colored according to the
	\textit{p}-value of the most significant \textit{k}-mer overlapping
	them. The vertical dotted line between haplotypes indicates the location of
	the causal SNP at this locus. (e) Contingency table of
	the phenotypes and haplotypes found in the population at the
	\textit{G} locus.}
\end{figure}

% Stem termination - Dt1 locus
\subsubsection*{Stem termination type - \textit{Dt1} locus}
\label{sv-gwas-main-results-stem-termination-dt1}

While all GWAS approaches found signals near the Glyma.19g194300 gene
associated with the \textit{Dt1} locus for stem termination type, the
\textit{k}-mer and Platypus analyses performed best since the region defined by the most highly associated markers
(top 1\%) included the causal gene (Figures
\ref{stem_termination_all_manhattan}, \ref{stem_termination_all_Dt1_signal}).
We did identify one of the SNPs suggested as putative causal variants for this
locus by \cite{tian2010} and \cite{liu2010}, however this SNP was only the
135\textsuperscript{th} most associated marker found by Platypus and the
12\textsuperscript{th} most associated \emph{k}-mer (Figures
\ref{stem_termination_all_Dt1_gene}, \ref{stem_termination_all_Dt1_kmers}).
Our results are consistent with previous findings in
suggesting that there is more than one causal variant at the \textit{Dt1} locus \citep{liu2010, tian2010}. Indeed, a
few cultivars bearing the allele associated with \emph{dt1} showed
indeterminate or semi-determinate phenotypes and some cultivars showed a
determinate phenotype without bearing the allele associated with \emph{dt1}
(Figure \ref{stem_termination_all_Dt1_kmers}). The existence of several causal
variants might explain why our analyses did not identify a causal variant more
clearly.

% Hilum color - T locus
\subsubsection*{Hilum color -- \textit{T} locus}
\label{sv-gwas-main-results-hilum-color-t}

The \textit{T} locus mentioned above for pubescence color is also a determining
locus for hilum color. We detected signals overlapping the corresponding gene
using all approaches (Figures \ref{hilum_color_all_manhattan},
\ref{hilum_color_all_T_signal}). The regions defined by the top 5\% (Paragraph)
and top 1\% (Platypus and \textit{k}-mers) variants/\textit{k}-mers included
the Glyma.06g202300 gene associated with this locus. Similarly to what we
observed for the analysis of the \textit{T} locus based on pubescence color, we
did identify the documented causal variant at this locus using the Platypus and
\textit{k}-mer analyses, although these variants were not the most
significantly associated overall (Figures \ref{hilum_color_all_T_gene},
\ref{hilum_color_all_T_kmers}).

% Hilum color - I locus
\subsubsection*{Hilum color -- \textit{I} locus}
\label{sv-gwas-main-results-hilum-color-i}

Similarly, the \textit{I} locus discussed above for its effect on seed coat
color also controls hilum color and we detected signals overlapping the
associated inverted duplication using all approaches (Figures
\ref{hilum_color_all_manhattan}, \ref{hilum_color_all_I_signal}). However,
as was the case for the effect of \textit{I} on seed coat color, we were
not able to link any of the variants identified to the causal variation given
its highly complex nature.

% Hilum color - R locus
\subsubsection*{Hilum color -- \textit{R} locus}
\label{sv-gwas-main-results-hilum-color-r}

We conducted a GWAS analysis that only compared accessions with black and brown
hila in an attempt to detect stronger signals at the \textit{R} locus.  We
accordingly detected a signal overlapping this locus using all approaches, but
only Platypus and \textit{k}-mers detected signals overlapping the
Glyma.09g235100 gene associated with \textit{R} (Figures
\ref{hilum_color_blackbrown_manhattan}, \ref{hilum_color_blackbrown_R_signal}).
\cite{gillman2011} identified this R2R3 MYB transcription factor as the
molecular basis for the \textit{R} locus and documented four different
loss-of-function alleles linked to brown hilum color. Although our analyses did
detect strong signals in the vicinity of this gene, the most associated markers
were not located in the body of the gene (Figure
\ref{hilum_color_blackbrown_R_signal}).  Interestingly, the most significantly
associated \emph{k}-mers detected by our analysis mapped 2 kb upstream of
another gene (Glyma.09g234900) annotated as a MYB transcription factor and
putatively involved in the regulation of anthocyanin biosynthesis. This gene as
well as Glyma.09g235100 and two others are part of a tandem array of four R2R3
MYB genes considered as candidates by \cite{gillman2011}, and of which
Glyma.09g235100 was the only one expressed in the seed coat. Given the
convincing evidence provided by \cite{gillman2011}, it is most likely that
Glyma.09g235100 is indeed the gene associated with the \emph{R} locus and that
our failure to identify the most significantly associated \emph{k}-mers as
causal variants is due to the coexistence of several variants that can cause
loss of function at the \emph{R} locus.  Still, we observed two causal variants
previously documented by \cite{gillman2011} in our dataset (Figures
\ref{hilum_color_blackbrown_R_gene}, \ref{hilum_color_blackbrown_R_kmers}). 

% Pubescence density - Ps locus
\subsubsection*{Pubescence density -- Ps locus}
\label{sv-gwas-main-results-pubescence-density-ps}

We conducted a GWAS analysis comparing accessions with normal and semi-sparse
pubes\-cence in order to detect loci associated with pubescence density. This analysis
detected signals at the \textit{Ps} locus using all three approaches (Figures
\ref{pubescence-density-main-figure}a-b, \ref{pubescence_density_manhattan}, \ref{pubescence_density_Ps_signal}
and \ref{pubescence_density_Ps_gene}). The causal
variant at this recently cloned locus is a copy number variant overlapping the
Glyma.12g187200 gene, with higher copy number resulting in reduced pubescence \citep{liu2020ps}.
We identified a region of $\sim$42 kb that varied in copy number across
the population and was associated with the trait (Figure
\ref{pubescence-density-main-figure}c), and therefore possibly corresponded to the
25.6-kb region identified as the causal CNV by \cite{liu2020ps}. We found that
reads containing the most significantly associated \textit{k}-mer mapped to either of the two ends
of the CNV (Figure \ref{pubescence-density-main-figure}b,d). This suggests that
the most associated \textit{k}-mer is indeed linked to the CNV
that is causal at this locus, while \textit{k}-mers within the
CNV region may represent differences in sequence between the copies.

The difference in size between the CNV reported by \cite{liu2020ps} and our
results may be due to the presence of some unresolved sequence (a stretch of
4,635 'N' nucleotides) over that interval in the reference sequence that we used.
Because our SV filtering process removed any SV overlapping 'N' nucleotides in
the reference, we reasoned that the CNV may have been found by our pipeline but
filtered out later on. Indeed, found that a 42.8-kb duplication corresponding
to the one described above among the genotype calls made by smoove. An association
analysis of that single variant using GAPIT found a \textit{p}-value of 1.9e-23,
which would have made it the most significant SV. In this particular case, the
failure of the SV GWAS analysis to find identify the causal variant was due
to our filtering parameters and not to limitations of the SV genotype calls per se.

% Figure 4 with the results on pubescence density at locus Ps
\begin{figure}
	\centering
	\includegraphics[width = 6in]{pubescence_density_Ps_main_figure.png}
	\caption[Results of \textit{k}-mer-based GWAS at the
	\textit{Ps} locus for pubescence density.]{Results of \textit{k}-mer-based GWAS at the
	\textit{Ps} locus for pubescence density.
	(a) Genome-wide Manhattan plot showing the location of significant \textit{k}-mers. The
	horizontal dotted line indicates the significance threshold whereas the
	vertical dotted line indicates the location of the \textit{Ps} locus.
	(b) Zoomed-in Manhattan plot at the \textit{Ps} locus. The
	vertical dotted line indicates the location of the most significant
	\textit{k}-mer. The shaded gray area shows the location of the causal CNV.
	(c) Estimated copy number (sequencing depth at position divided by
	accession average sequencing depth) of accessions with normal and
	semi-sparse pubescence at the \textit{Ps} locus. Solid lines represent
	the median copy number across 73 (semi-sparse pubescence) and 180
	accessions (normal pubescence) computed using a 100-bp sliding window.
	Shaded areas outline the first and third quartiles. The start and end
	of the CNV region can be clearly inferred from the divergence between
	the copy number of accessions with contrasting pubescence.
	(d) Histogram showing the location of 1,169 mapped reads containing the
	most significant \textit{k}-mer in accessions with semi-sparse pubescence.
	Reads containing this \textit{k}-mer map to either end of the CNV
	region, suggesting that they are associated with this variant. Panels
	(b), (c) and (d) share the same x-axis.}
	\label{pubescence-density-main-figure}
\end{figure}

% Seed coat luster - B locus
\subsubsection*{Seed coat luster -- B locus}
\label{sv-gwas-main-results-seed-coat-luster-b}

Next, we conducted a GWAS analysis comparing dull and shiny seed coat
phenotypes for the seed coat luster trait. Seed coat luster has been shown by
\cite{gijzen1999} to be largely caused by the deposition of a hydrophobic
protein (HPS) at the seed surface. Copy number variation of this gene at the
\textit{B} locus was later shown to explain variation in luster, although
sequence variation in the HPS sequence may also play a role \citep{gijzen2006}.
Using a sequencing depth-based analysis, we identified a 31-kb region spanning
positions 9,386,109-9,417,431 on chromosome Gm15 that exhibited variation in
copy number across accessions. This CNV overlapped four genes annotated as
containing HPS domains and therefore likely represented the causal variant
(Figures \ref{seed_coat_luster_dullshiny_B_signal},
\ref{seed_coat_luster_dullshiny_B_gene}).  We found signals overlapping this
CNV using all approaches, and the most significantly associated \textit{k}-mer
was notably located within the CNV region (Figures
\ref{seed_coat_luster_dullshiny_manhattan},
\ref{seed_coat_luster_dullshiny_B_signal}). However, we were not able to
directly link any of the variants or \textit{k}-mers identified to the causal
CNV.

\subsection*{Analysis of loci with unknown causal genes}
\label{sv-gwas-results}
In addition to analyzing loci for which the genes are already known, we were
able to suggest candidate genes at loci for which the underlying genes are not
known yet. 

% Pod color - L1 and L2 loci
\subsubsection*{Pod color -- \textit{L1} and \textit{L2} loci}
\label{sv-gwas-main-results-pod-color-l1-l2}

Both the \textit{L1} and the \textit{L2} loci for pod
color have yet to be cloned. While \cite{he2015} proposed the gene
Glyma.19g101700 as likely corresponding to \textit{L1} based on fine mapping
and expression analysis, our results as well as those reported by
\cite{bandillo2017} do not support this gene as a candidate due to its location
$\sim$3 Mb away from GWAS signals. Our analyses based on
\textit{k}-mers and Platypus found signals overlapping 14 and 6
genes, respectively (Figures \ref{pod_color_all_manhattan},
\ref{pod_color_all_L1_signal}). Of these, a MATE transporter (Glyma.19g120300)
appears as a prime candidate as the most significant \emph{k}-mer was observed
within its sequence and it is highly expressed in pods according to data hosted
on SoyBase \citep{grant2010, severin2010}. Moreover, the role of some MATE
proteins in transporting flavonoids has been demonstrated \citep{chen2015}.

The signals found by Platypus and \emph{k}-mer analyses at the \textit{L2}
locus overlap 24 and 50 genes, respectively (Figures
\ref{pod_color_all_manhattan}, \ref{pod_color_all_L2_signal}). Despite this
somewhat high number, we noticed that the closest gene to the
most significant \emph{k}-mer (Glyma.03g005800) is also a MATE transporter,
similarly to our top candidate for \emph{L1}. This gene may therefore represent
an interesting candidate for this locus for future studies.

% Pubescence form - Pa1 and Pa2 loci
\subsubsection*{Pubescence form -- \textit{Pa1} and \textit{Pa2} loci}
\label{sv-gwas-main-results-pubescence-form-pa1-pa2}

The two known loci for pubescence form in soybean, \textit{Pa1} and
\textit{Pa2}, have also not been cloned yet, although \cite{gilbert2017} has suggested
Glyma.12g213900 as a candidate for \textit{Pa1} based on data published
by \cite{bandillo2017}. This gene encodes a Myb transcription factor annotated
as playing a role in trichome branching. Indeed, the most
significant signals found by all three approaches were located near this gene
(Figures \ref{pubescence-form-main-figure}a-b,
\ref{pubescence_form_all_manhattan}, \ref{pubescence_form_all_Pa1_signal}).
The analyses based on Platypus and \emph{k}-mers identified two highly
significant non-synonymous SNPs in Glyma.12g213900 that were associated with appressed pubescence
(Figures \ref{pubescence-form-main-figure}c-d, \ref{pubescence_form_all_Pa1_gene},
\ref{pubescence_form_all_Pa1_kmers}). The alternative alleles for these two
SNPs co-occurred together in our dataset and may represent causal variants at
the \emph{Pa1} locus. The signals observed at the \emph{Pa2} locus were much
weaker and the signal observed from the \emph{k}-mer GWAS overlapped a total of
20 genes (Figures \ref{pubescence_form_all_manhattan},
\ref{pubescence_form_all_Pa2_signal}), none of which appeared obviously linked
to pubescence form.

A second GWAS analysis focused only on appressed and semi-appressed phenotypes
with the objective of enhancing the signal at the \emph{Pa2} locus. However,
this analysis only detected weak signals at the \emph{Pa1} locus (Figure
\ref{pubescence_form_noerect_manhattan}) and its results were therefore not
analyzed further.

% Figure 5 with the results on pubescence form at locus Pa1
\begin{figure}
	\centering
	\includegraphics[width = 5in]{pubescence_form_all_Pa1_main_figure.png}
	\caption[Results of SNP/indel- and \textit{k}-mer-based GWAS at the
	\textit{Pa1} locus for pubescence form]{(Caption on the following page.)}
	\label{pubescence-form-main-figure}
\end{figure}

\begin{figure}
	\contcaption{Results of SNP/indel- and \textit{k}-mer-based GWAS at the
	\textit{Pa1} locus for pubescence form. (a) Genome-wide Manhattan plots.
	Horizontal dotted lines indicate the significance
	threshold whereas vertical dotted lines indicate the location of the
	\textit{Pa1} and \textit{Pa2} loci. (b) Zoomed-in Manhattan plots at the \textit{Pa1} locus.
	The vertical dotted lines indicate the location of the
	Glyma.12g213900 gene suggested as a candidate for this locus. The shaded gray
	rectangles show the regions delimited by the top 1\% associations.
	In the case of SNP/indels, blue points denote markers used in the
	original analysis, whereas red points denote points that had originally
	been pruned but whose \emph{p}-values were computed after signal
	discovery.
	(c) Gene model of Glyma.12g213900. Rectangles represent exons, colored
	rectangles represent coding sequences, and the arrows indicate the
	direction of transcription.
	(d) Manhattan plots at the location of the Glyma.12g213900 gene.
	Vertical lines indicate the location of two non-synonymous
	SNPs associated with appressed pubescence, which may represent causal variants
	at this locus. Panels (c) and (d) share the same axis.}
\end{figure}

% Resistance to Phytophthora sojae
\subsubsection*{Resistance to \textit{P. sojae}}
\label{sv-gwas-main-results-psojae}

We used corrected dry weight (CDW) as described by \cite{deronne2022} as a
measure of horizontal resistance to \emph{P. sojae}. We found a strong signal
overlapping the genomic region reported by these authors on Gm15 using all all GWAS
approaches (Figure \ref{corrected_dry_weight_manhattan}). \cite{deronne2022}
proposed Glyma.15g217100 as a candidate gene for explaining phenotypic
variation at that locus based on a combination of functional annotation and
contrasting gene expression in resistant and susceptible lines four days after
inoculation. Although the signals found by our analyses did overlap
Glyma.15g217100, we found that this gene was located at the distal end of the signal
in a region with markers or \emph{k}-mers showing a weaker
association (Figure \ref{corrected_dry_weight_cdwGm15_signal}). The
large size of the observed signal means that the gene associated with the phenotype could
be located almost anywhere in this region, however, we identified a particularly
interesting region defined by the top 1\% associations in the Platypus
analysis. This region overlaps only three annotated genes and also contains the
most significant \emph{k}-mer (Figure
\ref{corrected_dry_weight_cdwGm15_signal}). One of the genes in this region is
a calcium ion binding protein with an EF-hand domain (Glyma.15g217700).
Given the important role played by calcium signaling in defense
responses \citep{zhang2014}, we suggest Glyma.15g217700 as an interesting
candidate gene for explaining resistance to \emph{P. sojae} in this population.

% Stem terminination and pod color - novel loci
\subsubsection*{Novel loci for stem termination and pod color}
\label{sv-gwas-main-results-novel-loci}

In addition to verifying known loci, we also observed potentially novel loci in
the \textit{k}-mer GWAS analysis of stem termination and pod color.
When comparing accessions
with indeterminate and semi-determinate stem termination types, we observed
previously undocumented signals on chromosomes Gm11, Gm16 and Gm18 in
addition to the known \textit{Dt1} locus (Figure \ref{stem-termination-main-figure}a).
Analysis of the pairwise LD between significant \textit{k}-mers found no
LD between these signals, suggesting that they may represent \textit{bona fide}
loci associated with stem termination type
(Figure \ref{stem-termination-main-figure}b). Similarly, a \textit{k}-mer GWAS
analysis comparing accessions with black and brown pods found a previously
undocumented signal on chromosome Gm15. \textit{k}-mers at this new locus showed no LD with \textit{k}-mers
at the known \textit{L1} locus and may therefore represent a distinct locus (Figures
\ref{pod_color_blbr_manhattan}, \ref{pod_color_blbr_ld}). The observed signal overlaps a total of
27 genes (Figure \ref{pod_color_blbr_pdcGm15_signal}), but none of those are
annotated as having an obvious role in controlling pod color.


% Figure 6 with the results novel loci for stem termination type
\begin{figure}
	\centering
	\includegraphics[width = 6in]{stem_termination_sn_main_figure.png}
	\caption[Detection of novel signals for stem termination type
	using a \textit{k}-mer GWAS analysis contrasting accessions with
	indeterminate and semi-determinate stems]{Detection of novel signals for stem termination type
	using a \textit{k}-mer-based GWAS analysis contrasting accessions with
	indeterminate and semi-determinate stems.
	(a) Genome-wide Manhattan plot showing the location of significant \textit{k}-mers.
	The horizontal dotted line indicates the significance threshold
	whereas vertical dotted lines indicate the location of the \textit{Dt1}
	classical locus as well as three new loci of interest (\textit{stGm11},
	\textit{stGm16}, and \textit{stGm18}).
	(b) Pairwise LD between the 162 significant \textit{k}-mers identified.
	\textit{k}-mers are sorted along the axes according to their
	putative position on the reference assembly version 4 of Williams82.
	\textit{k}-mers are
	represented in the same order along the x- and y-axis. The blue
	rectangles below the x-axis represent the
	-log\textsubscript{10}(\textit{p}) of each \textit{k}-mer.}
	\label{stem-termination-main-figure}
\end{figure}

\subsection*{Potential of \emph{k}-mer-based GWAS}
\label{sv-gwas-potential-kmers}

One of the most striking results from the analyses presented above is the
impressive performance of \emph{k}-mer-based GWAS compared to SNP/indel- and
SV-based GWAS. The GWAS analyses using \emph{k}-mers identified
the causal variant as the most significantly associated variant at four loci
with various types of causal variants: \emph{W1} for flower color, \emph{Td}
for pubescence color, \emph{G} for seed coat color, and \emph{Ps} for
pubescence density (Table \ref{loci-table}). In comparison, this result was achieved by SV-based GWAS
at the \textit{W1} locus only, and at none of the loci for SNP/indel-based GWAS.
The \textit{k}-mer approach also systematically performed at least as well as
the other three approaches in identifying signals overlapping known variants or genes
(Table \ref{loci-table}). Moreover, whenever \textit{k}-mer-based GWAS did
detect a signal, the region defined by the 1\% most significant \textit{k}-mers
overlapped the known causal variant or gene. 

Even in cases where the most significantly associated \emph{k}-mers did not
pinpoint known causal variants, they were often located near
those variants. For example, the most significantly associated \emph{k}-mers
mapping to the \emph{T} locus for pubescence color were located in intronic sequences
of the gene known to be associated with this locus.
Similarly, the most significantly associated
\emph{k}-mers at the \emph{B} locus for seed coat luster mapped within a
genomic interval known to vary in copy number at that locus. In addition, the most
significant \emph{k}-mers at the \emph{R} locus mapped just upstream of a
homolog of the gene associated with the locus. Even though the \emph{k}-mer
analysis did not identify the causal variants directly in those cases, it still
provided a very strong starting point for candidate gene identification.

As would be expected of any GWAS analysis, the success of the \emph{k}-mer approach
depends on the quality of the phenotypic data and the complexity of the genetic
architecture of the trait. It is clear that the loci for which the
\emph{k}-mer approach was most successful (\emph{W1}, \emph{Td}, \emph{G} and
\emph{Ps}) have a simple genetic architecture and phenotypes that can be easily
determined. For other traits, challenges in determining accurate phenotypes may
result in a decreased ability to identify causal variants or genes. For
example, \cite{bandillo2017} noted that observed phenotypes for stem
termination type in the GRIN database may not correspond to the expected
genotypes at classical loci, and that the intermediate phenotype for seed coat
luster is not always consistent across different environments. For qualitative
traits, the numerical scale used to recode phenotypes for use in GWAS is somewhat
arbitrary and may influence the final results. In the case of quantitative
phenotypes, the usual limitations to the precision and stochastic variation of
phenotypic values will necessarily make the interpretation more difficult than
for simple qualitative traits. Notwithstanding those limitations,
\textit{k}-mer-based GWAS clearly performed better than any of the other
approaches tested here given the same phenotypic data. This is partly due
to the fact that the \textit{k}-mer approach encompasses all variant types.
Moreover, it is likely that the simpler workflow that results from using
\textit{k}-mer presence/absence avoids some of the errors introduced during
variant calling and genotyping, and therefore yields more accurate results.

\subsection*{Limitations and challenges of \emph{k}-mer-based GWAS}
\label{limitations-of-kmers-GWAS}

While we have have shown that using \emph{k}-mers for GWAS in soybean performed
well compared to other GWAS methods, this
approach does present some unique challenges and limitations. One obvious limitation
is that using the presence/absence of \emph{k}-mers may not be appropriate in
species where heterozygosity is common (i.e., most species other than inbred
crops) as the \emph{k}-mers associated with the presence of both alleles will
be observed at heterozygous loci. Moreover, although the presence/absence of
\emph{k}-mers can detect variation independently of its type, some types of
variation may be more difficult to detect than others.
For example, copy number variation at a locus may not be
detected from presence/absence alone if all copies are identical and occur in
tandem, such that no \textit{k}-mer can differentiate between two copies and
more. For similar reasons, \emph{k}-mers may also fail to capture variation
occurring in repeated sequences. Alternative models based on \emph{k}-mer
counts instead of presence/absence have been developed \citep{rahman2018,
he2021} which may solve issues like these and broaden the scope of variants
that can be identified from \emph{k}-mers.

The complexity of analyzing the results of \emph{k}-mer-based GWAS is
another limitation of this approach. Although the method developed by
\cite{voichek2020} is in itself relatively simple and computationally
inexpensive, downstream analysis and interpretation of the output is
challenging. Indeed, there are no state-of-the-art methods for identifying the
putative genomic coordinates of a \emph{k}-mer, grouping significant
\emph{k}-mers together into coherent signals, and linking them to biologically
meaningful sequence variation. Going forward, mapping
the \emph{k}-mers or reads to a graph-based genome \citep[using e.g. the vg
toolkit;][]{siren2021} may represent an
improvement to using a single linear genome.
Grouping significant \emph{k}-mers by LD instead of by genomic
position, as was done for some of the analyses in this paper, could prove
useful in determining how many loci control a trait. In this study, we used an
assembly-based approach similar to what others have previously done
\citep[e.g.][]{voichek2020, rahman2018}. This approach allowed us to link
\emph{k}-mers to their underlying variation for simple cases involving SNPs and
indels, but we were unable to link significantly associated \emph{k}-mers to
more challenging causal variants such as those explaining the \emph{I} or
\emph{B} loci. In this context, developing approaches to systematically link
significant \emph{k}-mers to the sequence variation that underlies their
presence is much needed. Furthermore, finding ways to limit the number of spurious
associations found for some traits, as was the case for flower color and
pubescence color in our study, is also needed in order to obtain robust results.

In this work, we provide a set of tools enabling the downstream
analysis of \emph{k}-mer-based GWAS. Most importantly, we developed the
katcher program for retrieving all reads containing
any of a set of significant \emph{k}-mers in a computationally efficient manner. Previous studies usually
mapped \emph{k}-mers to a reference genome to identify their reference-based
coordinates \citep[e.g.][]{voichek2020, he2021, rahman2018, tripodi2021}.
However, while this approach may work for \emph{k}-mers that show little
divergence from the reference, it might fail to identify the genomic location
of \emph{k}-mers that diverge significantly or are not found at all in the
reference. Moreover, it does not take into account the sequence context of the read where
the \emph{k}-mer was found, which may provide valuable information for mapping
purposes. Our approach avoids these pitfalls by instead systematically retrieving reads
containing significant \emph{k}-mers, including those that did not map at all
to the reference assembly. While we were able to position the most significant
\emph{k}-mers on the reference sequence in our analyses, it may not be the case
for highly variable species or if the causal variation at a locus results from
a large novel insertion.

\subsection*{Use of SV genotypes for GWAS}
\label{use-of-svs-for-gwas}

In addition to \emph{k}-mers, we also assessed the use of SVs in GWAS analysis of soybean.
Although we did find the causal variant as the most significantly associated SV
at the \emph{W1} locus, results were otherwise rather inconclusive.
Indeed, most of the other known or putative causal variants were SNPs or
indels, and therefore were not represented in the SV datasets. At
most loci where SVs are known to be the underlying causal variant (e.g.
\emph{I}, \emph{B}, \emph{Ps}), the analyses based on SVs did not find the causal variants, most likely
due to the difficulty of calling and genotyping such variants
\citep{kosugi2019, lemay2022}. In the case of the \textit{Ps} locus, this limitation
was due to a filtering step in our pipeline rather than to an inherent inability to
properly call and genotype the causal variant.

Given the current performance of SV
genotyping programs, it is not clear whether using SVs for GWAS provides any
advantage over using \emph{k}-mer- or SNP-based GWAS, or a combination of both.
Indeed, in the single case where the SV-based analyses identified the causal
variant, the \emph{k}-mer-based GWAS identified it as well. Moreover,
SNP/indel-based GWAS provided better guidance in delimiting the signals found
at a given locus because of the much higher density of markers compared to SVs.
The usefulness of SV-based GWAS compared to SNP-based
GWAS will largely depend on whether SVs can represent variation that is not
otherwise in LD with SNPs. In tomato, \cite{dominguez2020} found that most
transposable element insertions associated with phenotypic variation in
agronomic traits were not tagged by SNPs. On the other hand, in humans,
\cite{maretty2017} found that the majority of SVs were in high LD with SNPs. It
is therefore likely that how well SVs can be tagged by SNPs will depend on
the particular type of SV and on the species. A more
efficient approach than SV-based GWAS may be to conduct \emph{k}-mer-based GWAS
and only then link known SVs to significant \emph{k}-mers.
Alternatively, \textit{k}-mer-based GWAS can be conducted specifically for
\textit{k}-mers that are associated with SVs \citep{jayakodi2020}.
Previous studies have
obtained noteworthy results by using SVs for GWAS analyses \citep[e.g.][]{zhang2015, akakpo2020,
dominguez2020}, however it would be worthwhile to see whether these results
could be replicated using \emph{k}-mer-based GWAS.

\section*{Conclusion}

In conclusion, we used SV-based and \emph{k}-mer-based GWAS to study ten
qualitative traits and three quantitative traits in soybean cultivars and
compared the results to traditional SNP/indel-based GWAS. While
\emph{k}-mer-based GWAS proved to be a powerful approach in pinpointing the
causal variation or genes associated with known loci, it is unclear whether
conducting SV-based GWAS is worth the investment. In addition, we used the
results from the \emph{k}-mer- and SNP-based GWAS to suggest candidate genes
for a few classical loci that have yet to be cloned. Based on our results, we
believe that an optimal workflow may involve conducting SNP and \emph{k}-mer
GWAS in parallel to identify significant signals and candidate genes,
potentially in combination with SV datasets. Given the large
number of samples for which WGS data are now available in several species,
applying \emph{k}-mer-based GWAS to leverage already existing sequencing and
phenotypic data appears feasible and promising. As part of our work, we
developed several computational tools that should help other researchers with the downstream
analysis of a \emph{k}-mer-based GWAS results. However, much work
remains to be done in developing state-of-the-art methods for the downstream
analysis of significant \emph{k}-mers. In particular, better approaches are
needed for linking \emph{k}-mers to sequence variation and moving from
presence/absence-based methods to count-based methods.

\section*{Additional files}
\label{sv-gwas-additional-files}

All additional files can be accessed on figshare \citep{kmer-gwas-figshare}.

\begin{description}
	\item[Additional file 1.csv] Metadata on the sequencing runs retrieved
		from the Sequence Read Archive (SRA) (CSV 71 KB)
	\item[Additional file 2.csv] Metadata on individual accessions following mapping
		and comparison to SoySNP50K data (CSV 36KB)
	\item[Additional file 3.csv] Phenotypic dataset used for GWAS analyses (CSV 135
		KB)
\end{description}

\section*{Availability of data and materials}
\label{sv-gwas-availability}

Some of the datasets generated as part of this work are available on figshare
\citep{kmer-gwas-figshare}.

The Illumina sequencing data used for the analyses are listed in Additional file 1
and available through NCBI BioProjects \citeauthor{PRJNA257011}, \citeauthor{PRJNA289660}
and \citeauthor{PRJNA639876}.

The high-quality assemblies generated by \cite{liu2020-pangenome} are available
on the Genome Warehouse through Accession Number \citeauthor{PRJCA002030}.

The assemblies of ZH13, W05 and Lee are available on SoyBase
(\url{https://soybase.org/GlycineBlastPages/blast_descriptions.php}).

The SVs identified from Oxford Nanopore data by \cite{lemay2022} are available
on figshare \citep{soybean-sv-figshare}.

The following software used in this work is available on GitHub:

\begin{itemize}
	\item Code used for the analyses:
		\url{https://github.com/malemay/soybean_kmer_gwas}
	\item katcher software for retrieving reads containing \textit{k}-mers:
		\url{https://github.com/malemay/katcher}
	\item gwask R package used for processing the output of GAPIT and
		\textit{k}-mer GWAS analyses, and for plotting:
		\url{https://github.com/malemay/gwask}
	\item Forked svmu version that was used for the analyses shown in this
		paper: \url{https://github.com/malemay/svmutools}
	\item svmutools R package for converting the output of svmu to VCF
		format: \url{https://github.com/malemay/svmutools}
\end{itemize}

\section*{Funding}
\label{sv-gwas-funding}

This work was supported by the SoyaGen grant (\url{https://www.soyagen.ca})
awarded to F. Belzile and R. Bélanger, and funded by Génome Québec, Genome
Canada, the government of Canada, the Ministère de l'Économie, Science et
Innovation du Québec, Semences Prograin Inc., Syngenta Canada Inc., Sevita
Genetics, Coop Fédérée, Grain Farmers of Ontario, Saskatchewan Pulse Growers,
Manitoba Pulse \& Soybean Growers, the Canadian Field Crop Research Alliance
and Producteurs de grains du Québec. M-A. Lemay has been supported by a NSERC
Canada Vanier Graduate Scholarship, a FRQNT doctoral B2X scholarship, a NSERC
Michael Smith Foreign Study Supplement, and a scholarship from the
AgroPhytoSciences NSERC CREATE Training Program. None of the funding bodies
were involved in study design, data acquisition, data analysis, interpretation
of the results, or manuscript writing.

\section*{Authors' contributions}
\label{sv-gwas-contributions}

Conception and design of the study: MAL, JAS, MDR, RB, FB.
Phenotypic data acquisition (resistance to \textit{Phytophthora sojae}): MDR, RB.
Data analysis: MAL.
Data interpretation: MAL, JAS, FB.
Software writing: MAL.
Manuscript drafting: MAL, JAS, FB.
All authors have revised the manuscript and approved its submission.

\section*{Acknowledgements}
\label{sv-gwas-acknowledgwments}

We would like to thank Yoav Voichek for help regarding \textit{k}-mer GWAS
analysis.  We also thank Brian Boyle and Martine Jean for valuable comments
regarding the methodology. We are thankful to the Digital Research Alliance of
Canada for using their high-performance computing servers.

\bibliographystyle{genome_research.bst}
\bibliography{references.bib}


\chapter*{Supplemental data to ``\textit{k}-mer-based GWAS enhances the discovery of causal variants and candidate genes in soybean''}
\label{supdata}

% Format figure and table numbers to follow S1, S2, ...
\renewcommand\thefigure{S\arabic{figure}} 
\setcounter{figure}{0}
\setcounter{table}{0}
\renewcommand\thetable{S\arabic{table}} 

% Reading the contents of the variables.txt file first
\DTLloaddb[noheader, keys={key,value}]{variables}{variables.txt}
% Defining a new command to fetch the value from the key
\newcommand{\getvar}[1]{\DTLfetch{variables}{key}{#1}{value}}

% Turning page numbering off for the supplements
\setcounter{page}{1}

% List of authors
\textbf{Authors}: Marc-André Lemay, Maxime de Ronne, Richard Bélanger, François Belzile

% Detailed methods for some aspects of the paper
\section*{Detailed methods}
\label{annexe-sv-gwas-detailed-methods}

\subsection*{Discovery of structural variation from Illumina data}
\label{sv-gwas-sv-discovery-illumina}

We used the Illumina whole-genome sequencing (WGS) data from the 389 selected \emph{G. max} accessions
for SV discovery using methods similar to those described by
\cite{lemay2022}.  Briefly, we used four different SV discovery programs or
pipelines: 

\begin{enumerate}

\item We used AsmVar \citep[version of 2015-04-16, ][]{liu2015} to call SVs
	from \emph{de novo} assemblies generated with SOAPdenovo2 v. 2.04
		\citep{luo2012} and aligned to the reference genome using LAST
		v. 1047 \citep{kielbasa2011}.
\item We ran Manta v. 1.6.0 \citep{chen2016} in randomly selected batches: 77 batches of
	5 samples and 1 batch of 4 samples. We used the candidate SVs
		identified from each of the batches for downstream analyses.
\item We used smoove v. 0.2.4 \citep{pedersen2019} to obtain a dataset of candidate SVs
	using the following sequence of commands:
	\texttt{smoove call}, \texttt{smoove merge}, \texttt{smoove
		genotype} and \texttt{smoove paste}.
		For three (3) of the 389 samples, we were unable
		to obtain candidate variants due to a reproducible
		error (segmentation fault) when running \texttt{smoove
		call}.
	\item We used SvABA v. 1.1.3 \citep{wala2018} to call SVs from each of the
	samples separately. We classified the SVs called as breakends by SvABA
		as either deletions, duplications or inversions, and converted
		them to explicit sequence using a custom R script. Variants
		called as indels by SvABA were used as is for downstream
		analyses.
\end{enumerate}

We filtered the SVs called by each tool to remove those meeting any of the following conditions:

\begin{itemize}
	\item smaller than 50 bp or larger than 500 kb in size
	\item located on unanchored scaffolds or organellar genomes
	\item classified as unresolved breakends (\texttt{SVTYPE=BND})
	\item containing at least one N in the alternate allele sequence
\end{itemize}

Next, we converted all filtered variants to a sequence-explicit format using
\texttt{bayesTyperTools convertAllele} \citep{sibbesen2018} and normalized them
using bcftools \texttt{norm} v. 1.10.2-105 \citep{li2009-samtools}.

A single VCF file was generated from the variants called using Illumina data by
running SVmerge \citep{wong2010} in two steps. In the first step, we merged
very similar variants among the ones called by the same tool using a
sliding window of 5 bp. Next, we used the output of the first merging steps
for all four tools and merged similar variants across the four datasets using
a sliding window of 15 bp. This approach was used because a
sliding window of 15 bp on the outputs of all tools without pre-merging
resulted in a computationally intractable number of pairwise comparisons.

\subsection*{Discovery of structural variation from Oxford Nanopore sequencing data}
\label{sv-gwas-sv-discovery-ont}

We used SVs discovered among 17 Canadian soybean cultivars sequenced using
Oxford Nano\-pore Technologies (ONT) by \cite{lemay2022} as an additional
source of candidate SVs. Although this set of cultivars does not overlap the
ones analyzed in this study, it is likely that many of the SVs occurring
in Canadian germplasm also occur in the germplasm used for this study given that modern soybean
cultivars are derived from a rather small set of progenitors. We merged the SVs detected in the
long-read (ONT) data with the ones discovered in the short-read (Illumina) data
using SVmerge and systematically favored SVs discovered by Illumina whenever
SVs from the two sets were to be merged.  Our rationale for favoring variants
discovered from the Illumina data was that whenever both technologies discover
the same variants, the breakpoints of SVs discovered from Illumina data are
likely to be more precise given the higher basecalling accuracy.

\subsection*{Discovery of structural variation from high-quality assemblies}
\label{sv-gwas-sv-discovery-assembly}

In addition to Illumina data from 389 accessions, we also used
high-quality genome assemblies of 29 \emph{G. max} or \emph{Glycine soja}
accessions to call SVs. We
retrieved 26 high-quality genomes assembled by \cite{liu2020-pangenome} as well
as the genome of the cultivar Zhonghuang 13 (ZH13) from the Genome Warehouse
repository (\url{https://ngdc.cncb.ac.cn/gwh/}). Furthermore, we retrieved the genomes
of \emph{G. soja} accession W05 and \emph{G. max} cultivar Lee from SoyBase
\citep{grant2010}.

We used methods based on \cite{liu2020-pangenome} to call SVs from the
genome assemblies.  First, we used the nucmer program of the MUMmer suite of
tools v. 4.0.0rc1 \citep{marcais2018} with option \texttt{-c 1000} to align
each of the assemblies to the Williams82 assembly. Next, we used the
delta-filter program of the MUMmer suite to keep only one-to-one alignments
between any genome and the Williams82 reference using the \texttt{-1} option.
The filtered alignments were used to call SVs with a version of the svmu
program \citep{chakraborty2019} that we forked from the original repository
(\url{https://github.com/malemay/svmu}) in order to improve execution time and
memory usage.

We used a set of tools that we assembled in an R package called svmutools
(\url{https://github.com/malemay/svmutools}) to convert the svmu output to VCF
format. We limited ourselves to processing deletions and insertions because of
the complexity of resolving other types of variants from whole-genome
assemblies.  Copy number variants and inversions called by svmu were therefore
excluded from downstream analyses. The
resulting set of variants was filtered to remove those smaller than 50 bp or larger than 500
kb and those with any N in either the reference or alternate allele sequence.
Finally, we normalized the VCF files using \texttt{bcftools norm} and merged
them using SVmerge to obtain a single file with variants discovered through
comparison of high-quality genome assemblies.

\subsection*{Genotyping structural variation using Paragraph}
\label{sv-gwas-sv-genotyping-paragraph}

We used Paragraph v. 2.4a \citep{chen2019}
to generate a set of SV genotypes from the Illumina data of the 389 accessions.
The set of SVs used as input for Paragraph was generated using SVmerge by
merging variants discovered from the Illumina and ONT data with
those discovered from the high-quality assemblies and svmu.
Variants in this candidate set were prepared for input to Paragraph by first removing
those closer than 150 bp (the maximum read length in our dataset) to any
chromosome end and then padding the variant representation using a script written by
\cite{hickey2020}. 

We ran Paragraph individually on each sample as recommended by the authors.  We
computed the average sequencing depth for each sample from the output of the
\texttt{samtools coverage} command and used 20 times that value for the
\texttt{-M} parameter of Paragraph.  We ran Paragraph using the
\texttt{multigrmpy} command and merged the resulting VCF files using
\texttt{bcftools merge} in order to get a  single genotype file for the whole
population. The resulting genotype calls were filtered as described in the main
text.

\subsection*{Comparison of WGS and SoySNP50K genotypes}
\label{annexe-sv-gwas-soysnp50k}
When we first analyzed the results from the GWAS analysis on all 389 samples,
we found discrepancies between the haplotypes at some loci and the expected
phenotype that should correspond to that haplotype for simple traits such as
flower color and seed coat color. Upon further analysis, we found that a
mismatch between the sequencing data and the identity of the accession was the
most probable cause of these discrepancies. In order to identify the samples
for which this could be the case, we compared the genotypes of SNPs obtained
from the SoySNP50K array \citep{song2013} to those predicted from the Illumina WGS
data. We did this for the 385 samples that had a match in the GRIN database. Briefly, we
retrieved SoySNP50K genotype calls from SoyBase \citep{grant2010} and extracted
all SNPs with a MAF $\geq$ 0.1 in our population. We then translated the positions
of these SNPs from assembly version 2 to assembly version 4 of Williams by
finding exact matches of 41-bp sequences surrounding the SNP positions. The
32,852 SNPs for which unambiguous positions could be found on genome assembly
version 4 were genotyped from the mapped WGS reads using \texttt{bcftools
mpileup} and \texttt{bcftools call}. We then used \texttt{bcftools gtcheck} to
compute the concordance between the WGS genotype calls and the
SoySNP50K genotype dataset. We identified 24 samples
with < 90\% concordance between the two genotype datasets and removed them from
the dataset. We also excluded an additional sample for which concordance was over 90\%
but poor raw sequencing data quality (\%GC content) suggested contamination.
The concordance between WGS data and SoySNP50K genotypes is included in
Additional file 3 and shown in Figure \ref{concordance-histogram}.

\subsection*{Analysis of extremely low \textit{p}-values}
\label{annexe-sv-gwas-pvalues}

For one trait (flower color), the significance of some \emph{k}-mers was so
high that their \emph{p}-values were numerically equal to 0. In order to
compute $-\log_{10} (p)$ for visualization in Manhattan plots in these cases,
we set the \emph{p}-value to the smallest numerical value that could be
represented using R on our machine (2.225074e-308). In this particular case,
the values represented on the Manhattan plots represent an upper bound on the
\emph{p}-value and not an exact value.

% Detailed results that are not meant to fit the main paper
\section*{Supplemental results}
\label{annexe-sv-gwas-detailed-results}

This section describes results at additional loci that were not discussed in
the main text. A summary of the signals found at all loci using all three GWAS
methods is presented in Table \ref{signals_table}.

\subsection*{Flower color -- \textit{L1} locus}
\label{annexe-sv-gwas-flower-color-l1}

In addition to the \textit{W1} locus, \cite{bandillo2017} reported a signal for
flower color near the \textit{L1} locus.  None of the GWAS approaches that we
tested found a signal for flower color near this locus (Figure
\ref{flower_color_manhattan}). This result is unsurprising as W1 is the major
contributor to flower color in soybean, whereas the \textit{L1} locus
(typically associated with pod color) was detected with a much less significant
\textit{p}-value (see Table \ref{signals_table}) and only conjectured as
associated with flower color by \cite{bandillo2017}.

\subsection*{Pubescence color -- locus on chromosome Gm16}
\label{annexe-sv-gwas-pubescence-color-gm16}

In addition to the \textit{T} locus, the first GWAS analysis conducted on pubescence color
revealed a second locus on chromosome Gm16
(Figure \ref{pubescence_color_all_manhattan}). Upon closer analysis, this locus
appeared to be in strong LD with \emph{k}-mers mapping to chromosome Gm01 and
most importantly in moderate LD with the \emph{T} locus (Figure
\ref{pubescence_color_all_ld}), which suggests that this does not represent an
independent locus controlling pubescence color. This interpretation is also
supported by the fact that the signal at that locus only spans about 10 kb on
chromosome Gm16.

\subsection*{Seed coat color -- loci \textit{T}, \textit{O} and \textit{R}}
\label{annexe-sv-gwas-seed-coat-color-t-o}

In addition to the \textit{I} and \textit{G} loci, the \textit{k}-mer-based
GWAS analysis of five different seed coat color phenotypes (first GWAS in Table
\ref{SEED.COAT.COLOR_gwas_table}) detected weak signals overlapping the
\emph{T} and \emph{O} loci (Figure \ref{seed_coat_color_all_manhattan}c).
However, we interpreted those results as chance associations given the low
\textit{p}-value associated with the \emph{T} locus and the proximity of the
\emph{O} locus to the \emph{I} locus. None of the three GWAS approaches detected
a signal overlapping the \textit{R} locus.

\subsection*{Stem termination type -- loci \textit{Dt2} and \textit{E3}}
\label{sv-gwas-stem-termination}

We performed a GWAS analysis comparing only semi-determinate and indeterminate
stem termination phenotypes in an attempt to detect signals at the \emph{Dt2}
locus (Table \ref{STEM.TERMINATION.TYPE_gwas_table}).  However, this was
unsuccessful using all three GWAS approaches (Figure
\ref{stem_termination_sn_manhattan}). Similarly, we failed to detect signals at
the \textit{E3} locus, which is more commonly associated with maturity but was
mentioned as associated with stem termination type by \cite{bandillo2017}.

% Hilum color - T locus
\subsection*{Hilum color -- \textit{T} locus}
\label{annexe-sv-gwas-results-hilum-color-t}

The \textit{T} locus discussed in the main text for pubescence color is also a determining
locus for hilum color. We detected signals overlapping the corresponding gene
using all approaches (Figures \ref{hilum_color_all_manhattan},
\ref{hilum_color_all_T_signal}). The regions defined by the top 5\% (SVs)
and top 1\% (SNPs and \textit{k}-mers) variants included
the Glyma.06g202300 gene associated with this locus. Similarly to what we
observed for the analysis of the \textit{T} locus based on pubescence color, we
did identify the documented causal variant at this locus using the Platypus and
\textit{k}-mer analyses, although these variants were not the most
significantly associated overall (Figures \ref{hilum_color_all_T_gene},
\ref{hilum_color_all_T_kmers}).

% Hilum color - I locus
\subsection*{Hilum color -- \textit{I} locus}
\label{annexe-sv-gwas-results-hilum-color-i}

Similarly, the \textit{I} locus discussed in the main text for its effect on seed coat
color also controls hilum color and we detected signals overlapping the
associated inverted duplication using all approaches (Figures
\ref{hilum_color_all_manhattan}, \ref{hilum_color_all_I_signal}). However,
as was the case for the effect of \textit{I} on seed coat color, we were
not able to link any of the variants identified to the causal variation given
its highly complex nature.

\subsection*{Hilum color -- loci \textit{W1} and \textit{O}}
\label{sv-gwas-hilum-color}

None of the three GWAS analyses that we conducted on hilum color found signals
at the \emph{W1} and \emph{O} loci studied by \cite{bandillo2017} (Figures
\ref{hilum_color_blackbrown_manhattan} \ref{hilum_color_all_manhattan},
\ref{hilum_color_rbr_manhattan}).

\subsection*{Pubescence density -- loci \textit{Pd1} and \textit{P1}}
\label{sv-gwas-pubescence-density}

Our GWAS analysis on pubescence density did not detect any signals at the
\emph{Pd1} and \emph{P1} (Figure \ref{pubescence_density_manhattan}). Our
failure to detect these loci is likely due to our small population size
compared to \cite{bandillo2017} and to the smaller contribution of those loci
to pubescence density in our population.

\subsection*{Seed coat luster -- loci \textit{Hps}, \textit{B?} and \textit{I}}
\label{sv-gwas-seed-coat-luster}

We performed three different GWAS analyses on seed coat luster (Table
\ref{SEED.COAT.LUSTER_gwas_table}). Since all three analyses yielded similar results
(Figure \ref{seed_coat_luster_all_manhattan},
\ref{seed_coat_luster_nointermediate_manhattan} and
\ref{seed_coat_luster_dullshiny_manhattan}), we chose to focus on the
GWAS contrasting the dull and shiny phenotypes as it yielded the most
significant signals. In addition, it also provided results that were most consistent with the
analysis previously done by \cite{bandillo2017}.

The analysis detected signals at the \emph{B} locus as the sole classical locus
using all three methods (see main text and Figure
\ref{seed_coat_luster_dullshiny_manhattan}).  All methods also detected signals
overlapping the \emph{Hps} locus as defined by \cite{bandillo2017}, however
results shown by \cite{gijzen2003} and our results suggest that \emph{Hps} and
\emph{B} are the same locus. 

Apparent signals on chromosomes Gm20 and Gm09 (which may correspond to the B?
locus reported by \cite{bandillo2017}) were also detected, but analysis of the
LD patterns between \emph{k}-mers suggested that these signals are most likely
in LD with the \emph{B} locus (Figure \ref{seed_coat_luster_dullshiny_ld}).

\cite{bandillo2017} mentioned the \textit{I} locus as possibly associated
with seed coat luster, however we did not find evidence for such an association
in our dataset.

\subsection*{Pubescence form -- additional loci}
\label{sv-gwas-pubesence-form}

The first GWAS analysis on pubescence form found signals on chromosomes Gm04
and Gm15 with SNPs and \emph{k}-mers (Figure
\ref{pubescence_form_all_manhattan}). However, upon further analysis of LD
patterns, they appeared to be linked to the \emph{Pa1} locus (Figure
\ref{pubescence_form_all_ld}).


\subsection*{Maturity group}
\label{annexe-sv-gwas-maturity}

We did not detect any signals associated with maturity using our dataset
(Figure \ref{maturity_group_manhattan}). The use of maturity group as a proxy
for time to maturity instead of the number of days, combined with the limited
phenotypic variation of this trait in this collection of accessions (Table
\ref{MATURITY.GROUP_gwas_table}), likely explains our failure to detect
any signals.

\subsection*{Seed oil and protein content}
\label{sv-gwas-seed-oil-protein-content}

We conducted a GWAS analysis on seed oil content using data obtained from the
GRIN database. While several seed oil content quantitative trait loci (QTL)
have been reported due to high interest in this trait for soybean breeding
\citep{chaudhary2015}, we focused on two major QTL on chromosomes Gm15 and Gm20.
These loci have often been reported in \emph{G. max} and were notably observed by
\cite{bandillo2015} on a collection of USDA lines using SoySNP50K chip
genotypes and GRIN phenotypic data. We identified a signal located on
chromosome Gm15 using the \emph{k}-mer approach only (Figure
\ref{oil_manhattan}). Although this signal did not overlap with the coordinates
found by \cite{bandillo2017}, it was located only 2 kb away from it and 30 kb
away from Glyma.15g049200, the sugar transporter gene identified by
\cite{zhang2020} as associated with that locus. Therefore, the significantly
associated \emph{k}-mers were not linked to the causal variation at that locus.

Similarly to seed oil content, we conducted a GWAS on seed protein content
using data obtained from the GRIN database. As is generally the case in
soybean given the negative correlation between seed oil and protein content,
the two major QTL considered above and reported by \cite{bandillo2015} for seed
oil content are also major QTL for seed protein content. However, none of the
approaches detected a signal at this locus (Figure \ref{protein_manhattan}).

Our failure to detect signals overlapping the known genes at these two loci for
oil and protein is probably due to the fact that the oil and protein content
data that we used for GWAS were not obtained from orthogonal trials, which was
a clear limitation to this GWAS given our relatively small sample size. 

\clearpage

% Supplemental tables
\section*{Supplemental tables}
\label{annexe-sv-gwas-tables}

\clearpage


% Table with all signals
\begin{landscape} \footnotesize
	\begin{ThreePartTable}

		\begin{TableNotes}
		\item[a] Most significant -log\textsubscript{10}(\emph{p}) previously reported at this locus by the study in the Study column
		\item[b] Gene associated with the locus. A question mark following the identifier of the gene indicates that this is simply a candidate
			that has yet to be confirmed. A dash indicates that no candidate has been suggested yet. We have only included candidate
			genes that have been suggested in previous studies.
		\item[c] Most significant -log\textsubscript{10}(\emph{p}) reported from SNPs and indels at this locus. A dash indicates that no signal
			was detected from SNPs and indels at this locus. The same applies for other methods (SVs, \emph{k}-mers) in
			their respective columns.
		\item[d] The \emph{p}-value at the W1 locus using the \emph{k}-mers method was so small that it was numerically equal
			to zero. This value was therefore set to the -log\textsubscript{10} of the smallest value that could be represented
			in R using our machine.
		\end{TableNotes}

		\begin{longtable}{llcclcccl}

			\caption[Description of loci associated with the traits studied]%
			{Description of loci associated with the traits studied and -log\textsubscript{10}(\emph{p}) of significant
			signals detected in our study using three different genotype datasets (SNPs/indels, SVs, \emph{k}-mers).}\\
			\label{signals_table}\\

			\toprule
			Trait & Locus & Chromosome & -log\textsubscript{10}(\emph{p})\tnote{a} & Gene\tnote{b} & SNPs/indels\tnote{c} & SVs & \emph{k}-mers  & Study \\
			\midrule \endfirsthead

			\caption[]{Loci associated with the traits studied \emph{(continued)}}\\

			\toprule
			Trait & Locus & Chromosome & -log10(p) & Gene & SNPs/indels & SVs & k-mers  & Study \\
			\midrule
			\endhead

			\bottomrule
			Continued on next page \\
			\endfoot

			\bottomrule \insertTableNotes
			\endlastfoot
			\csvreader[head to column names,
			late after line=\\,
			late after last line=,
			before reading={\catcode`\#=12},
			after reading={\catcode`\#=6},
			/csv/separator=semicolon]%
			{../tables/signals_table.csv}{}{\Trait & \Locus & \Chromosome & \Pvalues & \Gene & \Platypus & \Paragraph & \Kmers & \Study}

		\end{longtable}

	\end{ThreePartTable}

\end{landscape}
\clearpage

\gwastableone{FLOWER.COLOR}{flower color}
\gwastabletwo{PUBESCENCE.COLOR}{pubescence color}
\gwastabletwo{SEED.COAT.COLOR}{seed coat color}
\gwastabletwo{STEM.TERMINATION.TYPE}{stem termination type}
\gwastablethree{HILUM.COLOR}{hilum color}
\gwastabletwo{POD.COLOR}{pod color}
\gwastabletwo{PUBESCENCE.FORM}{pubescence form}
\gwastableone{PUBESCENCE.DENSITY}{pubescence density}
\gwastablethree{SEED.COAT.LUSTER}{seed coat luster}
\gwastableone{MATURITY.GROUP}{maturity group}


% Supplemental figures
\section*{Supplemental figures}
\label{annexe-sv-gwas-figures}

% Figures mentioned in the main text
% Figures for flower color
\manhattanplot{flower_color}{flower color}{\getvar{flower_color_nsamples}}
\signalplot{flower_color_W1}{flower color}{\emph{W1}}%
{Vertical dotted lines indicate the location of the \getvar{flower_color_W1_gene} gene associated with the locus.}
\geneplot{flower_color_W1}{flower color}{\emph{W1}}{\getvar{flower_color_W1_gene} gene}%
{Vertical dotted lines in panels (c) and (d) indicate the location of the causal SV at this locus.}
\ldplot{flower_color}{flower color}{\getvar{flower_color_ldkmers}}
%\kmerplot{flower_color_W1}

% Figures for pubescence color
\manhattanplot{pubescence_color_all}{pubescence color (first GWAS)}{\getvar{pubescence_color_all_nsamples}}
\signalplot{pubescence_color_all_T}{pubescence color (first GWAS)}{\emph{T}}%
{Vertical dotted lines indicate the location of the \getvar{pubescence_color_all_T_gene} gene associated with the locus.}
\geneplot{pubescence_color_all_T}{pubescence color (first GWAS)}{\emph{T}}{\getvar{pubescence_color_all_T_gene} gene}%
{Vertical dotted lines in panels (b) and (d) indicate the location of the causal indel at this locus.}
\kmerplot{pubescence_color_all_T}{\getvar{pubescence_color_all_T_gene}}{\emph{T}}{pubescence color (first GWAS)}{a causal indel}

\manhattanplot{pubescence_color_nogray}{pubescence color (second GWAS)}{\getvar{pubescence_color_nogray_nsamples}}
\signalplot{pubescence_color_nogray_Td}{pubescence color (second GWAS)}{\emph{Td}}%
{Vertical dotted lines indicate the location of the \getvar{pubescence_color_nogray_Td_gene} gene associated with the locus.}
\geneplot{pubescence_color_nogray_Td}{pubescence color (second GWAS)}{\emph{Td}}{\getvar{pubescence_color_nogray_Td_gene} gene}%
{Vertical dotted lines in panels (b) and (d) indicate the location of the causal SNP at this locus.}
%\kmerplot{pubescence_color_nogray_Td}

% Figures for seed coat color
\manhattanplot{seed_coat_color_all}{seed coat color (first GWAS)}{\getvar{seed_coat_color_all_nsamples}}
\signalplot{seed_coat_color_all_I}{seed coat color (first GWAS)}{\emph{I}}%
{Vertical dotted lines indicate the boundaries of the tandem duplication/inversion identified as the causal variant at this locus.}

\manhattanplot{seed_coat_color_greenyellow}{seed coat color (second GWAS)}{\getvar{seed_coat_color_greenyellow_nsamples}}
\signalplot{seed_coat_color_greenyellow_G}{seed coat color (second GWAS)}{\emph{G}}%
{Vertical dotted lines indicate the location of the \getvar{seed_coat_color_greenyellow_G_gene} gene associated with the locus.}
\geneplot{seed_coat_color_greenyellow_G}{seed coat color (second GWAS)}{\emph{G}}{\getvar{seed_coat_color_greenyellow_G_gene} gene}%
{Vertical dotted lines in panels (b) and (d) indicate the location of the causal SNP at this locus.}
%\kmerplot{seed_coat_color_greenyellow_G}

% Figures for stem termination
\manhattanplot{stem_termination_all}{stem termination type (first GWAS)}{\getvar{stem_termination_all_nsamples}}
\signalplot{stem_termination_all_Dt1}{stem termination type (first GWAS)}{\emph{Dt1}}%
{Vertical dotted lines indicate the location of the \getvar{stem_termination_all_Dt1_gene} gene associated with the locus.}
\geneplot{stem_termination_all_Dt1}{stem termination type (first GWAS)}{\emph{Dt1}}{\getvar{stem_termination_all_Dt1_gene} gene}%
{Vertical dotted lines in panels (b) and (d) indicate the location of a causal SNP at this locus.}
\kmerplot{stem_termination_all_Dt1}{\getvar{stem_termination_all_Dt1_gene}}{\emph{Dt1}}{stem termination type (first GWAS)}{a causal SNP}


% Figures for hilum color
\manhattanplot{hilum_color_blackbrown}{hilum color (second GWAS)}{\getvar{hilum_color_blackbrown_nsamples}}
\signalplot{hilum_color_blackbrown_R}{hilum color (second GWAS)}{\emph{R}}%
{Vertical dotted lines indicate the location of the \getvar{hilum_color_blackbrown_R_gene} gene associated with the locus.}
\geneplot{hilum_color_blackbrown_R}{hilum color (second GWAS)}{\emph{R}}{\getvar{hilum_color_blackbrown_R_gene} gene}%
{Vertical dotted lines indicate the location an indel (panels b and d) and a SNP (panel d) that are putatively causal at this locus.}
\kmerplot{hilum_color_blackbrown_R}{\getvar{hilum_color_blackbrown_R_gene}}{\emph{R}}{hilum color (second GWAS)}{a SNP and an indel that are putatively causal}

% Figures for pubescence density
\manhattanplot{pubescence_density}{pubescence density}{\getvar{pubescence_density_nsamples}}
\signalplot{pubescence_density_Ps}{pubescence density}{\emph{Ps}}%
{Vertical dotted lines indicate the boundaries of the causal CNV overlapping the \getvar{pubescence_density_Ps_gene} gene associated with the locus.}
\geneplot{pubescence_density_Ps}{pubescence density}{\emph{Ps}}{location of the causal CNV overlapping the \getvar{pubescence_density_Ps_gene} gene.}%
{The vertical dotted line in panel (d) indicates the location of the most significant \emph{k}-mer, which is associated with the causal CNV at this locus.
The shaded gray rectangles show the boundaries of the causal CNV.}

% Figures for seed coat luster
\signalplot{seed_coat_luster_dullshiny_B}{seed coat luster (third GWAS)}{\emph{B}}%
{Vertical dotted lines indicate the boundaries of the causal CNV associated with the locus.}
\geneplot{seed_coat_luster_dullshiny_B}{seed coat luster (third GWAS)}{\emph{B}}{location of the causal CNV}%
{The shaded gray rectangles show the boundaries of the causal CNV.}
\manhattanplot{seed_coat_luster_dullshiny}{seed coat luster (third GWAS)}{\getvar{seed_coat_luster_dullshiny_nsamples}}

% Figures for pod color
\manhattanplot{pod_color_all}{pod color (first GWAS)}{\getvar{pod_color_all_nsamples}}
\signalplot{pod_color_all_L1}{pod color (first GWAS)}{\emph{L1}}{}
\signalplot{pod_color_all_L2}{pod color (first GWAS)}{\emph{L2}}{}

% Figures for pubescence form
\manhattanplot{pubescence_form_all}{pubescence form (first GWAS)}{\getvar{pubescence_form_all_nsamples}}
\signalplot{pubescence_form_all_Pa1}{pubescence form (first GWAS)}{\emph{Pa1}}%
{Vertical dotted lines indicate the location of the \getvar{pubescence_form_all_Pa1_gene} gene that we suggest as a candidate for this locus.}
\geneplot{pubescence_form_all_Pa1}{pubescence form (first GWAS)}{\emph{Pa1}}{\getvar{pubescence_form_all_Pa1_gene} candidate gene putatively}%
{Vertical dotted lines in panels (b) and (d) indicate the location of two highly significant non-synonymous SNPs at this gene.}
\kmerplot{pubescence_form_all_Pa1}{\getvar{pubescence_form_all_Pa1_gene} candidate}{\emph{Pa1}}{pubescence form (first GWAS)}{a non-synonymous SNP}
\signalplot{pubescence_form_all_Pa2}{pubescence form (first GWAS)}{\emph{Pa2}}{}
\manhattanplot{pubescence_form_noerect}{pubescence form (second GWAS)}{\getvar{pubescence_form_noerect_nsamples}}

% Figures for corrected dry weight
\manhattanplot{corrected_dry_weight}{corrected dry weight (resistance to \emph{P. sojae})}{\getvar{corrected_dry_weight_nsamples}}
\signalplot{corrected_dry_weight_cdwGm15}{corrected dry weight (resistance to \emph{P. sojae})}{\emph{cdwGm15}}%
{Vertical dotted lines indicate the location of the \getvar{corrected_dry_weight_cdwGm15_gene} gene suggested by de Ronne et al. (2022) as associated with the locus.}
%\ldplot{corrected_dry_weight}

% Figures for novel loci
\manhattanplot{pod_color_blbr}{pod color (second GWAS)}{\getvar{pod_color_blbr_nsamples}}
\ldplot{pod_color_blbr}{pod color (second GWAS)}{\getvar{pod_color_blbr_ldkmers}}
\signalplot{pod_color_blbr_pdcGm15}{pod color (second GWAS)}{newly suggested \emph{pdcGm15}}{}

% Figures mentioned only in supplemental data

% Histogram with concordance rates
\begin{cfigure}
	\includegraphics[width = 6in]{concordance_histogram}
	\caption{Histogram of concordance rates between genotypes derived from WGS data and SoySNP50K genotypes for 385 soybean accessions.
	The vertical line indicates the 0.9 concordance threshold that was used for filtering out mismatching samples.}
	\label{concordance-histogram}

\end{cfigure}

\clearpage

% Flower color

% Pubescence color
\ldplot{pubescence_color_all}{pubescence color (first GWAS)}{\getvar{pubescence_color_all_ldkmers}}

% Stem termination
\manhattanplot{stem_termination_sn}{stem termination type (second GWAS)}{\getvar{stem_termination_sn_nsamples}}
%\ldplot{stem_termination_sn}

% Hilum color
\manhattanplot{hilum_color_all}{hilum color (first GWAS)}{\getvar{hilum_color_all_nsamples}}
\signalplot{hilum_color_all_T}{hilum color (first GWAS)}{\emph{T}}%
{Vertical dotted lines indicate the location of the \getvar{hilum_color_all_T_gene} gene associated with the locus.}
\geneplot{hilum_color_all_T}{hilum color (first GWAS)}{\emph{T}}{\getvar{hilum_color_all_T_gene} gene}%
{Vertical dotted lines in panels (b) and (d) indicate the location of the causal indel at this locus.}
\kmerplot{hilum_color_all_T}{\getvar{hilum_color_all_T_gene}}{\emph{T}}{hilum color (first GWAS)}{a causal indel}
\signalplot{hilum_color_all_I}{hilum color (first GWAS)}{\emph{I}}%
{Vertical dotted lines indicate the boundaries of the tandem duplication/inversion identified as the causal variant at this locus.}

\manhattanplot{hilum_color_rbr}{hilum color (third GWAS)}{\getvar{hilum_color_rbr_nsamples}}

% Pod color

% Pubescence density

% Seed coat luster
\manhattanplot{seed_coat_luster_all}{seed coat luster (first GWAS)}{\getvar{seed_coat_luster_all_nsamples}}
\manhattanplot{seed_coat_luster_nointermediate}{seed coat luster (second GWAS)}{\getvar{seed_coat_luster_nointermediate_nsamples}}

\ldplot{seed_coat_luster_dullshiny}{seed coat luster (third GWAS)}{\getvar{seed_coat_luster_dullshiny_ldkmers}}

% Pubescence form
\ldplot{pubescence_form_all}{pubescence form (first GWAS)}{\getvar{pubescence_form_all_ldkmers}}


% Figures for maturity group
\manhattanplot{maturity_group}{maturity group}{\getvar{maturity_group_nsamples}}

% Figures for oil
\manhattanplot{oil}{seed oil content}{\getvar{oil_nsamples}}

% Figures for protein
\manhattanplot{protein}{seed protein content}{\getvar{protein_nsamples}}

% Bibliography
\bibliographystyle{genome_research}              % style de la bibliographie
\bibliography{references}                   % production de la bibliographie



\end{document}

