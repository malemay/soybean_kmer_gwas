%% BioMed_Central_Tex_Template_v1.06
%%                                      %
%  bmc_article.tex            ver: 1.06 %
%                                       %

%%% additional documentclass options:
%  [doublespacing]
%  [linenumbers]   - put the line numbers on margins

%%% loading packages, author definitions

%\documentclass[twocolumn]{bmcart}% uncomment this for twocolumn layout and comment line below
\documentclass[doublespacing]{bmcart}

%%% Load packages
\usepackage{amsthm,amsmath}
\RequirePackage[numbers]{natbib}
%\RequirePackage[authoryear]{natbib}% uncomment this for author-year bibliography
\RequirePackage{hyperref}
\usepackage[utf8]{inputenc} %unicode support
%\usepackage[applemac]{inputenc} %applemac support if unicode package fails
%\usepackage[latin1]{inputenc} %UNIX support if unicode package fails

% Packages added by M-A Lemay
\usepackage{csvsimple} % read and display csv files as tables
\usepackage{booktabs} % prettier tables
\usepackage{threeparttable} % easy addition of footnotes below tables

\def\includegraphic{}
\def\includegraphics{}

%%% Put your definitions there:
\startlocaldefs
\endlocaldefs

%%% Begin ...
\begin{document}

%%% Start of article front matter
\begin{frontmatter}

\begin{fmbox}
\dochead{Research}

%%%%%%%%%%%%%%%%%%%%%%%%%%%%%%%%%%%%%%%%%%%%%%
%%                                          %%
%% Enter the title of your article here     %%
%%                                          %%
%%%%%%%%%%%%%%%%%%%%%%%%%%%%%%%%%%%%%%%%%%%%%%

\title{k-mer-based GWAS enhances the discovery of causal variants and candidate genes in soybean}

\author[
  addressref={aff1,aff2,aff3},                   % id's of addresses, e.g. {aff1,aff2}
  email={marc-andre.lemay.2@ulaval.ca}   % email address
]{\inits{M.A.L.}\fnm{Marc-André} \snm{Lemay}}

\author[
  addressref={aff1,aff2,aff3},
  email={maxime.de-ronne.1@ulaval.ca}
]{\inits{M.D.R.}\fnm{Maxime} \snm{de Ronne}}

\author[
  addressref={aff1,aff2,aff3},
  email={richard.belanger@fsaa.ulaval.ca}
]{\inits{R.B.}\fnm{Richard} \snm{Bélanger}}

\author[
  addressref={aff1,aff2,aff3},
  email={francois.belzile@fsaa.ulaval.ca}
]{\inits{F.B.}\fnm{François} \snm{Belzile}}

\address[id=aff1]{%
  \orgdiv{Département de phytologie},
  \orgname{Université Laval},
  \city{Québec},
  \cny{QC, Canada}
}

\address[id=aff2]{%
  \orgdiv{Institut de biologie intégrative et des systèmes},
  \orgname{Université Laval},
  \city{Québec},
  \cny{QC, Canada}
}

\address[id=aff3]{%
  \orgdiv{Centre de recherche et d’innovation sur les végétaux},
  \orgname{Université Laval},
  \city{Québec},
  \cny{QC, Canada}
}

%%%%%%%%%%%%%%%%%%%%%%%%%%%%%%%%%%%%%%%%%%%%%%
%%                                          %%
%% Enter short notes here                   %%
%%                                          %%
%% Short notes will be after addresses      %%
%% on first page.                           %%
%%                                          %%
%%%%%%%%%%%%%%%%%%%%%%%%%%%%%%%%%%%%%%%%%%%%%%

%\begin{artnotes}
%%\note{Sample of title note}     % note to the article
%\note[id=n1]{Equal contributor} % note, connected to author
%\end{artnotes}

\end{fmbox}% comment this for two column layout

%%%%%%%%%%%%%%%%%%%%%%%%%%%%%%%%%%%%%%%%%%%%%%%
%%                                           %%
%% The Abstract begins here                  %%
%%                                           %%
%% Please refer to the Instructions for      %%
%% authors on https://www.biomedcentral.com/ %%
%% and include the section headings          %%
%% accordingly for your article type.        %%
%%                                           %%
%%%%%%%%%%%%%%%%%%%%%%%%%%%%%%%%%%%%%%%%%%%%%%%

\begin{abstractbox}

\begin{abstract} % abstract
\parttitle{Background}
Genome-wide association studies (GWAS) are powerful statistical methods that
detect associations between genotype and phenotype at genome scale. Despite
their power, GWAS frequently fail to pinpoint the causal variant or the gene
controlling a trait at a given locus in crop species. Assessing genetic
variants beyond single-nucleotide polymorphisms (SNPs) could alleviate this
problem, for example by including structural variants (SVs).

\parttitle{Results}
In this study, we tested the potential of SV- and \emph{k}-mer-based GWAS in
soybean by applying these methods to 13 traits.  We also performed
conventional GWAS analysis based on SNPs and small indels for
comparison.  We assessed the performance of each GWAS approach based on
results at loci for which the causal genes or variants were known from
previous genetic studies. We found that \emph{k}-mer-based GWAS was the
most versatile approach and the best at pinpointing causal variants or
candidate genes based on the most significantly associated
\emph{k}-mers.  Moreover, \textit{k}-mer-based analyses identified
promising candidate genes for loci related to pod color, pubescence
form, and resistance to the oomycete \textit{Phytophthora sojae}. In
our dataset, SV-based GWAS did not add value compared to
\textit{k}-mer-based GWAS and may not be worth the time and
computational resources required to genotype SVs at population scale.

\parttitle{Conclusions}
Despite promising results, significant challenges remain regarding the
downstream analysis of \emph{k}-mer-based GWAS.  Notably, better
methods are needed to associate significant \emph{k}-mers with sequence
variation. Together, our results suggest that coupling \emph{k}-mer-
and SNP/indel-based GWAS is a powerful approach for discovering
candidate genes in crop species.

\end{abstract}

%%%%%%%%%%%%%%%%%%%%%%%%%%%%%%%%%%%%%%%%%%%%%%
%%                                          %%
%% The keywords begin here                  %%
%%                                          %%
%% Put each keyword in separate \kwd{}.     %%
%%                                          %%
%%%%%%%%%%%%%%%%%%%%%%%%%%%%%%%%%%%%%%%%%%%%%%

\begin{keyword}
\kwd{Genome-wide association studies}
\kwd{\textit{k}-mer GWAS}
\kwd{Soybean genomics}
\kwd{Crop genomics}
\kwd{\textit{Phytophthora sojae}}
\kwd{Structural variation}
\end{keyword}

% MSC classifications codes, if any
%\begin{keyword}[class=AMS]
%\kwd[Primary ]{}
%\kwd{}
%\kwd[; secondary ]{}
%\end{keyword}

\end{abstractbox}
%
%\end{fmbox}% uncomment this for two column layout

\end{frontmatter}

%%%%%%%%%%%%%%%%%%%%%%%%% start of article main body
% <put your article body there>

%%%%%%%%%%%%%%%%
%% Background %%
%%
\section*{Background}
Genome-wide association studies (GWAS) are analytical approaches that detect
statistical associations between phenotypic observations for a trait of
interest and the genotypes of variants distributed throughout the
genome \cite{tam2019}. This approach has been used for over 15 years
\cite{visscher2012} and has enabled significant advances in our understanding
of the genetic architecture of traits involved in human health
\cite[e.g.][]{wellcome2007gwas} and crop \cite{tibbs2021} and animal
\cite{ma2019} production, among others.

Despite their power and convenience for finding significantly associated loci,
GWAS analyses frequently fail to pinpoint the gene(s) associated with a trait and
the causal variant(s) involved \cite{tam2019}. One possible reason for this
is that many GWAS analyses do not include the causal variant itself in their variant dataset
\cite{tibbs2021}. This could be either because the analysis
is restricted to a particular type of variant, such as single-nucleotide
polymorphisms (SNPs), or because the set of markers/variants genotyped in the
association panel is limited by the technology being used
\cite[e.g.][]{bandillo2015, sonah2015}. Despite this limitation, variants
located near the gene controlling the trait often do appear statistically
associated with the phenotype through linkage disequilibrium (LD) with the
causal variant \cite{korte2013}.  A typical workflow then involves the
identification of candidate genes within the haplotype blocks defined by the
statistically associated markers and follow-up functional analyses to confirm
that a given gene controls the trait \cite[e.g.][]{wang2018, liu2020ps}.

The inclusion of structural variants (SVs) as genotype data in GWAS may partly overcome
this issue by providing a set of variants with a potentially high functional impact.
SVs include any variant involving a difference of at least 50 nucleotides, such as
deletions, insertions, or inversions \cite{ho2020}. Such variants are known to
have large phenotypic impacts by disrupting coding or regulatory sequences
\cite{marroni2014}, and have been identified as causal variants in traits such
as resistance to soybean cyst nematode \cite{cook2012}, aluminum tolerance in
wheat \cite{maron2013}, and branching in maize \cite{studer2011}.  SVs have
already been used in GWAS with promising results
\cite[e.g.][]{zhang2015, akakpo2020, dominguez2020, liu2020-pangenome}, but
their use in GWAS is the exception rather than the norm.

Despite their phenotypic impact, SVs are still largely understudied compared to
SNPs because discovering and genotyping them with accuracy is difficult.
Comprehensive detection of SVs requires a combination of several methods that
often need to be tailored to the study species and data available
\cite{alkan2011, ho2020}. Even then, assessments of the performance of SV
discovery and genotyping generally reveal subpar sensitivity and precision
\cite[e.g.][]{cameron2019, chaisson2019, kosugi2019}. Moreover, despite recent
improvements in computational approaches for genotyping SVs
\cite[e.g.][]{siren2021, ebler2022}, such approaches remain relatively
inaccurate compared to SNP genotyping workflows.

Alternative approaches that allow one to assess a large spectrum
of variants in GWAS without genotyping SVs at population scale could be useful.
\cite{voichek2020} developed a \emph{k}-mer-based GWAS approach that may
represent such an alternative method. This approach relies on the presence or
absence of \emph{k}-mers observed in sequence reads in place of variant genotypes for use in
GWAS. Once a presence/absence table of \emph{k}-mers has been generated for the
population under study, this table can be used in association analyses
similarly to SNP or SV genotype calls. One advantage of using \emph{k}-mers is
that they can act as a molecular signature for any type of variant as long as
these variants result in presence/absence of \emph{k}-mers of a given length.
Another advantage is that \emph{k}-mers are not tied to a specific genomic
location and can thus be used to query associations between genotype and
phenotype at genomic locations that are not included in a reference genome.  A
few studies have applied this \emph{k}-mer-based GWAS approach
\cite[e.g.][]{tripodi2021, colque2021} and other similar approaches have been
developed \cite{rahman2018, he2021}, but similarly to the use of SVs in GWAS,
such approaches have not been widely adopted yet.

In this study, we assessed the potential of SV- and \emph{k}-mer-based GWAS in
a major crop, soybean (\emph{Glycine max}). We also conducted conventional GWAS
based on SNPs and indels. In order to test these methods, we analyzed ten
qualitative traits \cite{bandillo2017} and two quantitative traits
\cite{bandillo2015} that have been previously studied using a SNP array
developed for soybean \cite{song2013}. In addition, we analyzed a quantitative
trait (horizontal resistance to \emph{Phytophthora sojae}) that has been
previously studied using whole-genome sequencing (WGS) data
\cite{deronne2022}.  Since the underlying genes and causal variants are
already known for several of these traits, they provide a good test case for
the performance of GWAS methods.  The objectives of our study were to:

\begin{enumerate}
	\item Assess the potential of SV- and \emph{k}-mer-based GWAS approaches to
		pinpoint genes and causal variants associated with loci whose
		underlying genes or variants are already known.
	\item Identify potential candidate genes and/or causal variants at loci
		whose underlying genes or candidate variants are not yet known.
	\item Develop a set of computational tools for the downstream analysis
		of significantly associated \emph{k}-mers once these have been
		identified.
\end{enumerate}

\section*{Results and discussion}

We analyzed a total of 13 traits in a population of 363 resequenced \textit{G.
max} accessions using three GWAS approaches that differed in the genotypes
used. The first approach used SNP and indel genotypes called by Platypus.
Despite both SNPs and indels being used, we will refer to this approach as
SNP-based for simplicity. Our SV-based workflow used various sources of data
for SV discovery and a single program (Paragraph) for genotyping SVs at
population scale from Illumina WGS data. Finally, the third GWAS approach
relied on the presence/absence of \textit{k}-mers in the WGS data as genotypes.
Because of the large number of \textit{k}-mers analyzed, we only report results
for those that were considered significantly associated with traits of
interest.

Since two or three different GWAS analyses were conducted for some of the
traits, a total of 22 GWAS analyses were performed using each of the three
approaches.  Given the large volume of data that this represents, we focus here
on the most noteworthy results. Readers interested in results at loci that are
not discussed here are referred to Additional file 1. In particular, Table
S1 (Additional file 1) provides a summary of the
\textit{p}-values obtained by each approach at each of the loci considered in
this study.

\subsection*{Loci with known genes or causal variants}

We first wanted to assess the performance of the various GWAS
approaches by analyzing the results at loci for which the genes have
already been cloned. These loci provide interesting test cases because
the expected results (causal genes or variants) are known. Table \ref{loci-table} shows
a summary of the results discussed below.

\subsubsection*{Flower color -- \textit{W1} locus}

\cite{zabala2007} identified the flavonoid 3’5’-hydroxylase (F3’5’H) gene
Glyma.13g\-072100 as associated with the \textit{W1} locus for flower color. In
Williams82 and other accessions with white flowers, a 65-bp insertion into the
third exon of the gene results in a premature stop codon that renders the
F3’5’H enzyme non-functional and prevents pigmentation of the flower. All
GWAS approaches detected a signal overlapping the gene at this locus (Additional file 1: Figures
S1 and S2). The most
significant SV and \textit{k}-mers corresponded to the known causal variant
(Figure \ref{flower-color-main-figure}a-b and Additional file 1: S3c-d).
In particular, the assembly of the reads containing the most significant \textit{k}-mers
and the alignment of the resulting haplotypes showed that these \textit{k}-mers
directly tagged the causal variant (Figure \ref{flower-color-main-figure}c-e).
As the catalogue of variants analyzed by Platypus contains only SNPs and small
indels, the causal variant could not be captured by this approach. Still, SNPs
and indels located within the gene sequence were among the most significantly
associated ones (Additional file 1: Figure S3b).

The \textit{k}-mer GWAS analysis also identified presumably spurious significant
\textit{k}-mers scattered throughout the genome (Additional file 1: Figure
S1c). An analysis of the pairwise LD between a subset
of those \textit{k}-mers showed that most \textit{k}-mers formed a single LD block
linked to the \textit{W1} locus (Additional file 1: Figure S4). A small LD block
occurred on chromosome Gm09, however it was in moderate LD with the one on chromosome
Gm13 and therefore probably does not represent a separate locus.

We should note that when we first analyzed the results of the GWAS on flower
color, we observed that two accessions bearing the haplotype typically
associated with purple flowers were noted as having white flowers. Upon further
inspection, we found that the most likely cause of this discrepancy was a
mismatch between the genotype data and the identity of the accession for which
the phenotypic data was obtained. Indeed, when we compared SNP calls made
directly from the WGS data to those made using the SoySNP50K array, we observed
that these two accessions had low concordance rates of only 62\% and 70\%. To
ensure that such mismatches would not affect our GWAS analyses, we computed the
concordance between genotype calls made from WGS data and the SoySNP50K array
for all samples (see the Methods) and excluded 24 accessions for which the
concordance rate was below 90\%.

\subsubsection*{Pubescence color -- \textit{T} locus}

\cite{zabala2003} identified Glyma.06g202300 as the gene associated with the
\textit{T} locus for pubescence color and found a 1-bp deletion as a putative
causal variant at this locus. For all three GWAS approaches, we found a signal
overlapping this gene (Additional file 1: Figure S5). In
particular, the regions defined by the 1\% most associated variants
identified by the SNP- and \textit{k}-mer-based approaches included
Glyma.06g202300 (Additional file 1: Figure S6). While the most
significant \textit{k}-mers did occur within the sequence of this gene, they
did not correspond to the documented causal variant (Additional file 1: Figure
S7d).  Still, the fourth and fifth most strongly
associated \textit{k}-mers corresponded to this causal variant (Additional file 1: Figures
S7d and S8). We
observed that not all accessions with gray pubescence harbored the deleted
nucleotide (Additional file 1: Figure S8c), which could indicate
either that other causal variants are resulting in a non-functional F3’H in
this population, or that the causal variant lies elsewhere. As for SNP-based GWAS,
the most significant variant within the sequence of Glyma.06g202300 was the
causal variant, however it was not the most significantly associated variant
overall (Additional file 1: Figures S6b and
S7b). SV-based GWAS did not detect the causal variant
at this locus, as expected from its size.

\subsubsection*{Pubescence color -- \textit{Td} locus}

% Pubescence color - Td locus
We conducted a separate GWAS analysis contrasting only accessions with tawny
and light tawny pubescence because the \textit{Td} locus is known to control
these differences in color. Accordingly, we detected signals in this region
using all three approaches, but only SNP- and \textit{k}-mer-based GWAS
identified a signal that overlapped Glyma.03g258700 (Figure
\ref{pubescence-color-main-figure}a-b, and Additional file 1: Figures S9 and
S10).  While the most significant
\textit{k}-mer identified the causal variant documented at this locus
\citep{yan2020}, it was the fourth most significantly associated variant for
the SNP-based approach (Figure \ref{pubescence-color-main-figure}b-d, Additional file 1: Figure 
S11).  Interestingly, some of the accessions
bearing the haplotype associated with tawny pubescence displayed light tawny
pubescence, suggesting that there may be more than one causal variant at this
locus (Figure \ref{pubescence-color-main-figure}e).

\subsubsection*{Seed coat color - \textit{I} locus}

% Seed coat color - I locus
The causal variant at the \textit{I} locus for seed coat color is a complex SV
that takes the form of a 10.91-kb inverted duplication of three chalcone
synthase (CHS) genes \citep[CHS1, CHS3 and CHS4;][]{tuteja2008}. The presence of
this inverted duplication results in the silencing of all CHS genes in the seed
coat specifically and thus in the absence of seed coat pigmentation
\citep{tuteja2009}. While we did find signals overlapping this causal variant
using all three GWAS approaches (Additional file 1: Figures S12 and
S13), none of the approaches identified
it. This failure to detect the causal variant is likely due
to its complexity, which makes it difficult to detect and genotype it. While it
is likely that at least some of the significantly associated \textit{k}-mers
actually are derived from this causal variant, the lack of systematic methods for
linking \textit{k}-mers to sequence variation hindered our ability to do so.

\subsubsection*{Seed coat color - \textit{G} locus}

We conducted a second GWAS analysis contrasting accessions with yellow and
green seed coats in order to target the \textit{G} locus. Accordingly, this
analysis found strong signals overlapping the \emph{G} locus using all three
approaches (Additional file 1: Figure S14).
\cite{wang2018} identified Glyma.01g198500 as the gene associated with
this locus and found an A>G SNP affecting transcript splicing as the causal
variant. Consistently with their results, we identified the two
most significant \emph{k}-mers and the second most significant marker from the
SNP-based analysis as corresponding to that causal variant (Figure
\ref{seed-coat-color-main-figure} and Additional file 1: Figures
S15 and
S16). Unsurprisingly, SV-based GWAS
did not identify the causal variant given its type.

\subsubsection*{Stem termination type - \textit{Dt1} locus}

While all GWAS approaches found signals near the Glyma.19g194300 gene
associated with the \textit{Dt1} locus for stem termination type, the SNP- and
\textit{k}-mer-based analyses performed best since the region defined by their
most highly associated markers (top 1\%) included the causal gene (Additional file 1: Figures
S17 and S18).
We did identify one of the SNPs suggested as putative causal variants for this
locus by \cite{tian2010} and \cite{liu2010}, however this SNP was only the
135\textsuperscript{th} most associated marker found by Platypus and the
12\textsuperscript{th} most associated \emph{k}-mer (Additional file 1: Figures
S19 and S20).
Our results are consistent with previous findings in suggesting that there is
more than one causal variant at the \textit{Dt1} locus \citep{liu2010,
tian2010}. Indeed, a few cultivars bearing the allele associated with
\emph{dt1} showed indeterminate or semi-determinate phenotypes and some
cultivars showed a determinate phenotype without bearing the allele associated
with \emph{dt1} (Additional file 1: Figure S20c). The existence of
several causal variants might explain why our analyses did not identify a
causal variant more clearly.

\subsubsection*{Hilum color -- \textit{R} locus}

We conducted a GWAS analysis that only compared accessions with black and brown
hila in an attempt to detect stronger signals at the \textit{R} locus.  We
accordingly detected a signal overlapping this locus using all approaches, but
only SNP- and \textit{k}-mer-based GWAS detected signals overlapping the
Glyma.09g235100 gene associated with \textit{R} (Additional file 1: Figures
S21 and S22).
\cite{gillman2011} identified this R2R3 MYB transcription factor as the
molecular basis for the \textit{R} locus and documented four different
loss-of-function alleles linked to brown hilum color. Although our analyses did
detect strong signals in the vicinity of this gene, the most associated markers
were not located in the body of the gene (Additional file 1: Figure
S22).  Interestingly, the most significantly
associated \emph{k}-mers detected by our analysis mapped 2 kb upstream of
another gene (Glyma.09g234900) annotated as a MYB transcription factor and
putatively involved in the regulation of anthocyanin biosynthesis. This gene as
well as Glyma.09g235100 and two others are part of a tandem array of four R2R3
MYB genes considered as candidates by \cite{gillman2011}, and of which
Glyma.09g235100 was the only one expressed in the seed coat. Given the
convincing evidence provided by \cite{gillman2011}, it is most likely that
Glyma.09g235100 is indeed the gene associated with the \emph{R} locus and that
our failure to identify the most significantly associated \emph{k}-mers as
causal variants is due to the coexistence of several variants that can cause
loss of function at the \emph{R} locus.  Still, we observed two causal variants
previously documented by \cite{gillman2011} in our dataset (Additional file 1: Figures
S23 and S24). 

\subsubsection*{Pubescence density -- \textit{Ps} locus}

We conducted a GWAS analysis comparing accessions with normal and semi-sparse
pubes\-cence in order to detect loci associated with pubescence density. This analysis
detected signals at the \textit{Ps} locus using all three approaches (Figure
\ref{pubescence-density-main-figure}a-b and Additional file 1: Figures S25, S26
and S27). The causal
variant at this recently cloned locus is a copy number variant (CNV) overlapping the
Glyma.12g187200 gene, with higher copy number resulting in reduced pubescence \citep{liu2020ps}.
We identified a region of $\sim$42 kb that varied in copy number across
the population and was associated with the trait (Figure
\ref{pubescence-density-main-figure}c), and therefore possibly corresponded to the
25.6-kb region identified as the causal CNV by \cite{liu2020ps}. We found that
reads containing the most significantly associated \textit{k}-mer mapped to either of the two ends
of the CNV (Figure \ref{pubescence-density-main-figure}d). This suggests that
the most associated \textit{k}-mer is indeed linked to the CNV
that is causal at this locus, while \textit{k}-mers within the
CNV region may represent differences in sequence between the copies.

The difference in size between the CNV reported by \cite{liu2020ps} and our
results may be due to the presence of some unresolved sequence (a stretch of
4,635 N nucleotides) over that interval in the reference sequence that we used.
Because our SV filtering process removed any alternate allele containing N nucleotides,
we reasoned that the CNV may have been found by our pipeline but
filtered out later on. Indeed, we found a 42.8-kb duplication corresponding
to the one described above among the genotype calls made by smoove. An association
analysis of that single variant using GAPIT found a \textit{p}-value of 1.9e-23,
which would have made it the most significant SV. In this particular case, the
failure of the SV-based GWAS analysis to find identify the causal variant was due
to our filtering parameters and not to limitations of the SV genotype calls per se.

\subsubsection*{Seed coat luster -- \textit{B} locus}

We conducted a GWAS analysis comparing dull and shiny seed coat phenotypes for
the seed coat luster trait. Seed coat luster has been shown by
\cite{gijzen1999} to be largely caused by the deposition of a hydrophobic
protein (HPS) at the seed surface. Copy number variation of this gene at the
\textit{B} locus was later shown to explain variation in luster, although
sequence variation in the HPS sequence may also play a role \citep{gijzen2006}.
Using a sequencing depth-based analysis, we identified a 31-kb region spanning
positions 9,386,109-9,417,431 on chromosome Gm15 that exhibited variation in
copy number across accessions. This CNV overlapped four genes annotated as
containing HPS domains and therefore likely represented the causal variant
(Additional file 1: Figures S28 and
S29).  We found signals overlapping this
CNV using all approaches, and the most significantly associated \textit{k}-mer
was notably located within the CNV region (Additional file 1: Figures
S28 and
S30). However, we were not able to
directly link any of the variants or \textit{k}-mers identified to the causal
CNV.

\subsection*{Analysis of loci with unknown causal genes}

In addition to analyzing loci for which the genes are already known, we were
able to suggest candidate genes at loci for which the underlying genes are not
known yet. 

\subsubsection*{Pod color -- \textit{L1} and \textit{L2} loci}

Both the \textit{L1} and the \textit{L2} loci for pod color have yet to be
cloned. While \cite{he2015} proposed the gene Glyma.19g101700 as likely
corresponding to \textit{L1} based on fine mapping and expression analysis, our
results as well as those reported by \cite{bandillo2017} do not support this
gene as a candidate due to its location $\sim$3 Mb away from GWAS signals. Our
analyses based on \textit{k}-mers and SNPs found signals overlapping 14 and
6 genes, respectively (Additional file 1: S31 and
S32). Of these, a MATE transporter (Glyma.19g120300)
appears as a prime candidate as the most significant \emph{k}-mer was observed
within its sequence and it is highly expressed in pods according to data hosted
on SoyBase \citep{grant2010, severin2010}. Moreover, the role of some MATE
proteins in transporting flavonoids has been demonstrated \citep{chen2015}.

The signals found by SNP- and \emph{k}-mer-based analyses at the \textit{L2}
locus overlap 24 and 50 genes, respectively (Additional file 1: Figures
S31 and S33). Despite this
somewhat high number, we noticed that the closest gene to the most significant
\emph{k}-mer (Glyma.03g005800) is also a MATE transporter, similarly to our top
candidate for \emph{L1}. This gene may therefore represent an interesting
candidate at this locus for future studies.

\subsubsection*{Pubescence form -- \textit{Pa1} and \textit{Pa2} loci}
\label{sv-gwas-main-results-pubescence-form-pa1-pa2}

The two known loci for pubescence form in soybean, \textit{Pa1} and
\textit{Pa2}, have also not been cloned yet. Nevertheless, \cite{gilbert2017}
has suggested Glyma.12g213900 as a candidate for \textit{Pa1} based on data
published by \cite{bandillo2017}. This gene encodes a MYB transcription factor
annotated as playing a role in trichome branching. Indeed, the most significant
signals found by all three approaches were located near this gene (Figure
\ref{pubescence-form-main-figure}a-b, and Additional file 1: Figures S34 and
S35).  The analyses based on SNPs and
\emph{k}-mers identified two highly significant non-synonymous SNPs in
Glyma.12g213900 that were associated with appressed pubescence (Figure
\ref{pubescence-form-main-figure}c-d and Additional file 1: Figures S36 and
S37). The alternative alleles for these two
SNPs co-occurred together in our dataset and may represent causal variants at
the \emph{Pa1} locus. The signals observed at the \emph{Pa2} locus were much
weaker and the signal observed from the \emph{k}-mer GWAS overlapped a total of
20 genes (Additional file 1: Figures S34 and
S38), none of which appeared obviously linked
to pubescence form.

A second GWAS analysis focused only on appressed and semi-appressed phenotypes
with the objective of enhancing the signal at the \emph{Pa2} locus. However,
this analysis only detected weak signals at the \emph{Pa1} locus (Additional file 1: Figure
S39) and its results were therefore not
analyzed further.

\subsubsection*{Resistance to \textit{P. sojae}}

We used corrected dry weight (CDW) as described by \cite{deronne2022} as a
measure of horizontal resistance to \emph{P. sojae}. We found a strong signal
overlapping the genomic region reported by these authors on Gm15 using all three GWAS
approaches (Additional file 1: Figure S40). \cite{deronne2022}
proposed Glyma.15g217100 as a candidate gene for explaining phenotypic
variation at that locus based on a combination of functional annotation and
contrasting gene expression in resistant and susceptible lines four days after
inoculation. Although the signals found by our analyses did overlap
Glyma.15g217100, we found that this gene was located at the distal end of the signal
in a region with markers or \emph{k}-mers showing a weaker
association (Additional file 1: Figure S41). The
large size of the observed signal means that the gene associated with the phenotype could
be located almost anywhere in this region. However, we identified a particularly
interesting region defined by the top 1\% associations in the SNP-based
analysis. This region overlaps only three annotated genes and also contains the
most significant \emph{k}-mer (Additional file 1: Figure
S41). One of the genes in this region is
a calcium ion binding protein with an EF-hand domain (Glyma.15g217700).
Given the important role played by calcium signaling in defense
responses \citep{zhang2014}, we suggest Glyma.15g217700 as an interesting
candidate gene for explaining resistance to \emph{P. sojae} in this population.

\subsubsection*{Novel loci for stem termination and pod color}

In addition to verifying known loci, we also observed potentially novel loci in
the \textit{k}-mer GWAS analysis of stem termination and pod color.  When
comparing accessions with indeterminate and semi-determinate stem termination
types, we observed previously undocumented signals on chromosomes Gm11, Gm16
and Gm18 in addition to the known \textit{Dt1} locus (Figure
\ref{stem-termination-main-figure}a).  Analysis of the pairwise LD between
significant \textit{k}-mers found no LD between these signals, suggesting that
they may represent \textit{bona fide} loci associated with stem termination
type (Figure \ref{stem-termination-main-figure}b). Similarly, a \textit{k}-mer
GWAS analysis comparing accessions with black and brown pods found a previously
undocumented signal on chromosome Gm15 (Additional file 1: Figure S42).
\textit{k}-mers at this new locus
showed no LD with \textit{k}-mers at the known \textit{L1} locus and may
therefore represent a distinct locus (Additional file 1: Figure 
S43). The observed signal overlaps a total of 27 genes
(Additional file 1: Figure S44), but none of those are annotated
as having an obvious role in controlling pod color.

\subsection*{Potential of \emph{k}-mer-based GWAS}

One of the most striking results from the analyses presented above is the
impressive performance of \emph{k}-mer-based GWAS compared to SNP- and SV-based
GWAS. The GWAS analyses using \emph{k}-mers identified the causal variant as
the most significantly associated variant at four loci with various types of
causal variants: \emph{W1} for flower color, \emph{Td} for pubescence color,
\emph{G} for seed coat color, and \emph{Ps} for pubescence density (Table
\ref{loci-table}). In comparison, this result was achieved by SV-based GWAS at
the \textit{W1} locus only, and at none of the loci for SNP-based GWAS.
The \textit{k}-mer approach also systematically performed at least as well as
the other two approaches in identifying signals overlapping known variants or
genes (Table \ref{loci-table}). Moreover, whenever \textit{k}-mer-based GWAS
did detect a signal, the region defined by the 1\% most significant
\textit{k}-mers overlapped the known causal variant or gene. 

Even in cases where the most significantly associated \emph{k}-mers did not
pinpoint known causal variants, they were often located near those variants.
For example, the most significantly associated \emph{k}-mers mapping to the
\emph{T} locus for pubescence color were located in intronic sequences of the
gene known to be associated with this locus.  Similarly, the most significantly
associated \emph{k}-mers at the \emph{B} locus for seed coat luster mapped
within a genomic interval known to vary in copy number at that locus. In
addition, the most significant \emph{k}-mers at the \emph{R} locus mapped just
upstream of a homolog of the gene associated with the locus. Even though the
\emph{k}-mer analysis did not identify the causal variants directly in those
cases, it still provided a very strong starting point for candidate gene
identification.

As would be expected of any GWAS analysis, the success of the \emph{k}-mer approach
depends on the quality of the phenotypic data and the complexity of the genetic
architecture of the trait. It is clear that the loci for which the
\emph{k}-mer approach was most successful (\emph{W1}, \emph{Td}, \emph{G} and
\emph{Ps}) have a simple genetic architecture and phenotypes that can be easily
determined. For other traits, challenges in determining accurate phenotypes may
result in a decreased ability to identify causal variants or genes. For
example, \cite{bandillo2017} noted that observed phenotypes for stem
termination type in the GRIN database may not correspond to the expected
genotypes at classical loci, and that the intermediate phenotype for seed coat
luster is not always consistent across different environments. For qualitative
traits, the numerical scale used to recode phenotypes for use in GWAS is somewhat
arbitrary and may influence the final results. In the case of quantitative
traits, the usual limitations to the precision and stochastic variation of
phenotypic values will necessarily make the interpretation more difficult than
for simple qualitative traits. Notwithstanding those limitations,
\textit{k}-mer-based GWAS clearly performed better than any of the other
approaches tested here given the same phenotypic data. This is partly due
to the fact that the \textit{k}-mer approach encompasses all variant types.
Moreover, it is likely that the simpler workflow that results from using
\textit{k}-mer presence/absence avoids some of the errors introduced during
variant calling and genotyping, and therefore yields more accurate results.

\subsection*{Limitations and challenges of \emph{k}-mer-based GWAS}

While we have have shown that using \emph{k}-mers for GWAS in soybean performed
well compared to other GWAS methods, this approach does present some unique
challenges and limitations. One obvious limitation is that using the
presence/absence of \emph{k}-mers may not be appropriate in species where
heterozygosity is common (i.e., most species other than inbred crops) as the
\emph{k}-mers associated with the presence of both alleles will be observed at
heterozygous loci. Moreover, although the presence/absence of \emph{k}-mers can
detect variation independently of its type, some types of variation may be more
difficult to detect than others.  For example, copy number variation at a locus
may not be detected from presence/absence alone if all copies are identical and
occur in tandem, such that no \textit{k}-mer can differentiate between two
copies and more. For similar reasons, \emph{k}-mers may also fail to capture
variation occurring in repeated sequences. Alternative models based on
\emph{k}-mer counts instead of presence/absence have been developed
\citep{rahman2018, he2021} which may solve issues like these and broaden the
scope of variants that can be identified from \emph{k}-mers.

The complexity of analyzing the results of \emph{k}-mer-based GWAS is another
limitation of this approach. Although the method developed by
\cite{voichek2020} is in itself relatively simple and computationally
inexpensive, downstream analysis and interpretation of the output is
challenging. Indeed, there are no state-of-the-art methods for identifying the
putative genomic coordinates of a \emph{k}-mer, grouping significant
\emph{k}-mers together into coherent signals, and linking them to biologically
meaningful sequence variation. Going forward, mapping the \emph{k}-mers or
reads to a graph-based genome \citep[e.g. using the vg toolkit;][]{siren2021}
may represent an improvement to using a single linear genome.  Grouping
significant \emph{k}-mers by LD instead of by genomic position, as was done for
some of the analyses in this paper, could prove useful in determining how many
loci control a trait. In this study, we used an assembly-based approach similar
to what others have previously done \citep[e.g.][]{voichek2020, rahman2018}.
This approach allowed us to link \emph{k}-mers to their underlying variation
for simple cases involving SNPs and indels, but we were unable to link
significantly associated \emph{k}-mers to more challenging causal variants such
as those explaining the \emph{I} or \emph{B} loci. In this context, developing
approaches to systematically link significant \emph{k}-mers to the sequence
variation that underlies their presence is much needed. Furthermore, finding
ways to limit the number of spurious associations found for some traits, as was
the case for flower color and pubescence color in our study, is also needed in
order to obtain robust results.

In this work, we provide a set of tools enabling the downstream analysis of
\emph{k}-mer-based GWAS. Most importantly, we developed the katcher program for
retrieving all reads containing any of a set of significant \emph{k}-mers in a
computationally efficient manner. Previous studies usually mapped \emph{k}-mers
to a reference genome to identify their reference-based coordinates
\citep[e.g.][]{voichek2020, he2021, rahman2018, tripodi2021}.  However, while
this approach may work for \emph{k}-mers that show little divergence from the
reference, it might fail to identify the genomic location of \emph{k}-mers that
diverge significantly or are not found at all in the reference. Moreover, it
does not take into account the sequence context of the read where the
\emph{k}-mer was found, which may provide valuable information for mapping
purposes. Our approach avoids these pitfalls by instead systematically
retrieving reads containing significant \emph{k}-mers, including those that did
not map at all to the reference assembly. While we were able to position the
most significant \emph{k}-mers on the reference sequence in our analyses, it
may not be the case for highly variable species or if the causal variation at a
locus results from a large novel insertion.

\subsection*{Use of SV genotypes for GWAS}

In addition to \emph{k}-mers, we also assessed the use of SVs in GWAS analysis
of soybean.  Although we did find the causal variant as the most significantly
associated SV at the \emph{W1} locus, results were otherwise rather
inconclusive.  Indeed, most of the other known or putative causal variants were
SNPs or indels, and therefore were not represented in the SV datasets. At most
loci where SVs are known to be the underlying causal variant (e.g.  \emph{I},
\emph{B}, \emph{Ps}), the analyses based on SVs did not find the causal
variants, most likely due to the difficulty of calling and genotyping such
variants \citep{kosugi2019, lemay2022}. In the case of the \textit{Ps} locus,
this limitation was due to a filtering step in our pipeline rather than to an
inherent inability to properly call and genotype the causal variant.

Given the current performance of SV genotyping programs, it is not clear
whether using SVs for GWAS provides any advantage over using \emph{k}-mer- or
SNP-based GWAS, or a combination of both.  Indeed, in the single case where the
SV-based analyses identified the causal variant, the \emph{k}-mer-based GWAS
identified it as well. Moreover, SNP-based GWAS provided better guidance
in delimiting the signals found at a given locus because of the much higher
density of markers compared to SVs.  The usefulness of SV-based GWAS compared
to SNP-based GWAS will largely depend on whether SVs can represent variation
that is not otherwise in LD with SNPs. In tomato, \cite{dominguez2020} found
that most transposable element insertions associated with phenotypic variation
in agronomic traits were not tagged by SNPs. On the other hand, in humans,
\cite{maretty2017} found that the majority of SVs were in high LD with SNPs. It
is therefore likely that how well SVs can be tagged by SNPs will depend on the
particular type of SV and on the species. A more efficient approach than
SV-based GWAS may be to conduct \emph{k}-mer-based GWAS and only then link
known SVs to significant \emph{k}-mers.  Alternatively, \textit{k}-mer-based
GWAS can be conducted specifically for \textit{k}-mers that are associated with
SVs \citep{jayakodi2020}.  Previous studies have obtained noteworthy results by
using SVs for GWAS analyses \citep[e.g.][]{zhang2015, akakpo2020,
dominguez2020}, however it would be worthwhile to see whether these results
could be replicated using \emph{k}-mer-based GWAS.

\section*{Conclusions}

In conclusion, we used SV-based and \emph{k}-mer-based GWAS to study ten
qualitative traits and three quantitative traits in soybean and
compared the results to conventional SNP-based GWAS. While \emph{k}-mer-based
GWAS proved to be a powerful approach in pinpointing the causal variation or
genes associated with known loci, it is unclear whether conducting SV-based
GWAS is worth the investment. In addition, we used the results from the
\emph{k}-mer- and SNP-based GWAS to suggest candidate genes for a few classical
loci that have yet to be cloned. Based on our results, we believe that an
optimal workflow may involve conducting SNP-based and \emph{k}-mer-based GWAS
in parallel to identify significant signals and candidate genes, potentially in
combination with SV datasets. Given the large number of samples for which WGS
data is now available in several species, applying \emph{k}-mer-based GWAS to
leverage already existing sequencing and phenotypic data appears feasible and
promising.  As part of our work, we developed several computational tools that
should help other researchers with the downstream analysis of
\emph{k}-mer-based GWAS. However, much work remains to be done in
developing state-of-the-art methods for the downstream analysis of significant
\emph{k}-mers. In particular, better approaches are needed for linking
\emph{k}-mers to sequence variation and moving from presence/absence-based
methods to count-based methods.

\section*{Methods}

\subsection*{Sample selection and processing of sequencing data}

We selected 389 inbred \emph{G. max} accessions based on the availability of
Illumina WGS data and phenotypic data for resistance to the oomycete \emph{P.
sojae} \citep{deronne2022}.  We identified 741 SRA runs corresponding to
Illumina paired-end sequencing data for those 389 accessions \citep[data from][]{zhou2015, valliyodan2016, fang2017, bayer2021}
and extracted paired-end reads using the fastq-dump command v. 2.9.6
(\citeauthor[][\url{https://github.com/ncbi/sra-tools}]{sratoolkit}) with the
option \texttt{-\-split-3}.
Reads that were not paired in the raw sequencing data were not used
for downstream analyses.  The runs downloaded from the SRA and associated
metadata are listed in Additional file 2.

We first filtered the sequencing data for quality and the presence of sequencing
adapters using bbduk from BBtools v. 38.25
(\citeauthor[][\url{https://sourceforge.net/projects/bbmap/}]{bbtools}). Reads
were then mapped to the reference assembly version 4 of \emph{G. max}
cultivar Williams82 \citep{valliyodan2019} concatenated with chloroplast
and mitochondrion sequences obtained from SoyBase \citep{grant2010}. We used
bwa-mem v. 0.7.17-r1188 \citep{li2009-bwa} with default parameters to map the
reads in paired-end mode. Reads that were left unpaired following adapter- and
quality-trimming were aligned separately in single-end mode with bwa-mem and
merged with mapped paired-end reads using samtools \texttt{merge}
v. 1.12 \citep{li2009-samtools}.  We then added read groups identifying
individual SRA runs to the BAM files and merged the reads belonging to the same
accession using bamaddrg
(\citeauthor[][\url{https://github.com/ekg/bamaddrg}]{bamaddrg}). We used the
resulting BAM files for all downstream analyses requiring aligned reads.
Metadata on mapping depth of aligned data can be found in Additional file 3.

\subsection*{Discovery and genotyping of SVs}

Our SV analysis pipeline consisted of separate SV discovery and genotyping steps.
The \emph{discovery} step identified a set of candidate SVs using various methods,
whereas the \emph{genotyping} step determined the genotype
of the candidate SVs for all accessions in the population from the mapped Illumina reads. We used three different
approaches in the SV discovery step to generate a set of candidate variants:

\begin{itemize}
	\item We used Illumina WGS data from the 389 accessions mentioned above
		to call SVs in the population following a pipeline outlined
		by \cite{lemay2022}. In brief, we combined information from
		the SV calling programs AsmVar \citep{liu2015}, Manta \citep{chen2016},
		smoove \citep{pedersen2019} and SvABA \citep{wala2018} to generate
		a set of candidate SVs from the mapped Illumina data.
	\item We used SVs discovered by \cite{lemay2022} from Oxford Nanopore
		sequencing data of 17 Canadian soybean cultivars.
	\item We called SVs from 26 high-quality genome assemblies published by
		\cite{liu2020-pangenome}, as well as that of the cultivar Zhonghuang 13 (ZH13),
		\textit{G. soja} accession W05, and \textit{G. max} cultivar Lee.
		We used methods similar to those of \cite{liu2020-pangenome} to call
		SVs from the comparison of these assemblies to that of Williams82.
\end{itemize}

The SVs called using each of these approaches were merged using SVmerge
\citep{wong2010} and the resulting set of candidate SVs was used for
genotyping. We used Paragraph \citep{chen2019} to genotype the candidate SVs
from the mapped Illumina reads for all accessions. We filtered the genotyped
SVs by setting the genotype calls made from less than two reads to missing and
by removing variants with a minor allele frequency < 0.02 or a proportion of
missing data > 0.5. The resulting dataset of 186,306 genotyped SVs was used for
GWAS. Detailed methods regarding the SV discovery and genotyping steps can be
found in Additional file 1.

\subsection*{Discovery and genotyping of SNPs and indels}

We used Platypus v. 0.8.1.1 \citep{rimmer2014} to call SNPs and indels from the
mapped reads of the 389 accessions. The 21.1 M SNPs and
indels from the raw output of Platypus were filtered to keep only those with:

\begin{enumerate}
	\item \texttt{FILTER} field set to \texttt{PASS}
	\item minor allele frequency (MAF) $\geq$ 0.02
	\item fraction of missing genotypes $\leq$ 0.5
	\item heterozygosity rate $\leq$ 0.2
\end{enumerate}

Moreover, we pruned the dataset using the \texttt{-\-indep-pairwise} option of
PLINK v. 1.90b5.3 \citep{purcell2007} with a window of 1,000 markers, a
step of 100 markers and an $r^2$ LD threshold of 0.9. This pruning step was
implemented to reduce the computational requirements of the GWAS analysis by
removing co-segregating markers. Markers located on unanchored scaffolds were
also removed. The resulting dataset was used for GWAS and comprised 773,060
SNPs and 151,570 indels.

\subsection*{Computing the presence/absence table of \emph{k}-mers}

We used the approach outlined by \cite{voichek2020} for \emph{k}-mer-based GWAS
and their recommendations on the
associated GitHub page (v. 0.2-beta, \url{https://github.com/voichek/kmersGWAS}) to generate
a presence/absence \emph{k}-mer table for use in GWAS.  The first step of
the approach involved counting all \emph{k}-mers of length 31 (31-mers) present
in the trimmed (FASTQ) reads for each accession using KMC3 v. 3.2.1
\citep{kokot2017}. \emph{k}-mers were counted twice, the first time using their
canonized form (the first of either the observed \textit{k}-mer or its reverse
complement in lexicographical order) and the second time using the observed
\emph{k}-mers themselves (non-canonized form). The two sets of \textit{k}-mers
were then combined and filtered by
keeping only those seen in at least 5 accessions, and observed in both canonized
and non-canonized form in at least 20\% of the accessions in which they are found.
We then generated a table indicating the presence or absence of each
\emph{k}-mer in all accessions. This table was used to compute a kinship
table for use in GWAS and was also used directly as input genotypes for the
GWAS analyses. We used \emph{k}-mers with a MAF $\geq$ 0.02 for both the
computation of kinship and GWAS.

\subsection*{Phenotypic data}

We analyzed ten qualitative traits previously studied by \cite{bandillo2017}: flower
color, pubescence color, seed coat color, stem termination type, hilum color,
pod color, pubescence form, pubescence density, seed coat luster, and maturity
group.  We obtained phenotypic data for these traits from the GRIN database
(\url{https://npgsweb.ars-grin.gov/gringlobal/search}) by querying the database
using the PI identifiers of the 389 accessions used.  Four out of the 389
accessions did not match the GRIN database and were therefore not used for the
analysis of qualitative traits. The phenotypes of qualitative traits were recoded
to numerical values for GWAS analysis following methods similar to those of
\cite{bandillo2017}.  For some traits, we conducted more than one GWAS analysis
using targeted subsets of the observed phenotypes in order to focus on specific
loci. The phenotypes and numerical values used for all analyses as well as the
number of observations in the dataset that we used for GWAS are listed in
Tables S2 to S11 (Additional file 1).

We analyzed three quantitative traits in addition to the ten qualitative traits
mentioned above. Seed oil and protein content were retrieved from the GRIN
database. In cases where more than one value was listed for a given accession,
we computed the average of those values and used it for GWAS analyses. As a third quantitative trait, we
used the horizontal resistance to the oomycete \textit{P. sojae} (corrected dry
weight, CDW) as recently assessed in a hydroponic assay \citep{deronne2022}.
Our analysis included 340 accessions whose phenotypic data
was already published by \cite{deronne2022} as well as 49 additional accessions for
which sequencing data was not yet available when their study was conducted.
Because of ambiguity regarding the identity of one of the accessions (HN019), this accession
was dropped from all GWAS analyses. Similarly to seed oil and
protein content, the CDW values for some accessions for which more than one
observation was available were averaged.

Because of clear discrepancies between the observed haplotypes and the reported
phenotypes for some simple traits such as flower color and seed coat color, we
suspected possible errors in the identity of some of the accessions. To
investigate this, we compared the SNP genotypes derived from the SoySNP50K
array and our WGS data at over 32,000 SNPs, and identified 24 accessions with <
90\% concordance between the two genotype datasets (see Additional file 1 for
detailed methods). These samples were removed from the dataset in addition to another
accession that exhibited an atypical GC content suggesting contamination.
Following these filtering steps and the removal of HN019, 363 accessions
remained for GWAS analyses. The concordance between WGS data and SoySNP50K
genotypes is included in Additional file 3 and averaged 98.3\% among retained
lines (Additional file 1: Figure S45). The phenotypic data used in this study can be found in Additional file 4.

\subsection*{GWAS analyses}

We used the GAPIT3 R package v. 3.1.0 \citep{wang2021} for conducting GWAS on
the SNP/indel and SV datasets described above. SVs and indels required some additional
preprocessing steps to make variant
representation suitable for downstream analyses \citep{lemay2022gwas}. Briefly, we recoded all
variants as SNPs prior to converting the VCF files to Hapmap diploid format
using the TASSEL command-line tools \citep{bradbury2007}; variant IDs enabled
retrieval of the proper metadata for each variant after GWAS.

We ran GAPIT using a mixed linear model (MLM) with 9 principal components and the VanRaden
algorithm for computing the kinship matrix. We used an MLM model for
consistency with the statistical model used by the \textit{k}-mer-based
approach.  We used a randomization approach to determine the 5\% family-wise
error-rate threshold for each GWAS analysis as described in \cite{voichek2020}.
To compute this threshold, we permuted the phenotypic observations 100 times
and computed a GWAS on these permuted phenotypes to obtain a distribution of
top \emph{p}-values under the null hypothesis. The fifth most significant of
the top 100 \emph{p}-values was used as the significance threshold for
inference, i.e. markers with \emph{p}-values lower than this threshold were
considered significant.

We conducted GWAS based on \emph{k}-mers using the method developed by
\cite{voichek2020}. In brief, the
analysis was conducted in two steps. The first step used an approximate model
to identify 1 million potentially significant \emph{k}-mers. These were then used in
the second step as input to an exact model implemented in GEMMA \citep{zhou2012}, resulting
in a \emph{p}-value for each \emph{k}-mer. We
used the kinship computed from the \emph{k}-mers to correct for relatedness
between accessions. The program automatically outputs a list of \emph{k}-mers
that passed a 5\% family-wise error-rate threshold determined by a
randomization approach, which we considered significant for the purposes of
downstream analyses.

\subsection*{Analysis of significant \emph{k}-mers}

\emph{k}-mers are not intrinsically associated with a particular genomic region
and a tailored analysis was therefore required to associate them with genomic
coordinates. To do so, we queried the mapped reads (in BAM format) of all accessions for matches
to significant \emph{k}-mers for a given trait. Because this is a
computationally demanding operation (there were typically tens of thousands of
significant \emph{k}-mers per trait), we developed a C program called katcher
(\url{https://github.com/malemay/katcher}) that
uses the htslib library \citep{bonfield2021} for efficient reading and
writing of BAM files. katcher and its associated utilities allow for efficient
retrieval and annotation of mapped reads containing any \emph{k}-mer from a set
of interest.

Once we retrieved all the reads containing significant \emph{k}-mers, we next
needed to gather the mapping information contained in those reads for the
purposes of identifying contiguous signals and comparing them to annotations. To
simplify the amount of information to process in downstream analyses, we
considered only the most significant \emph{k}-mer contained in a given read and
the position where this \emph{k}-mer was most often observed in
the dataset. Moreover, we limited the analysis to reads with a minimum mapping
quality of 20 and to \emph{k}-mers that were observed at least 10 times in at
least one accession. The latter was done to remove spurious \emph{k}-mers that may originate
from sequencing errors. All downstream analyses (Manhattan plots,
identification of signals, analysis of genes located near significant
\emph{k}-mers, etc.) used this processed set of significant \emph{k}-mers.

\subsection*{Downstream analyses of significant association signals}

We used contiguous regions of significant variants, which we called
``signals'', as a basis to compare our results with those obtained by previous
studies and to query the genome for candidate genes. We defined the boundaries
of those signals by grouping any significant markers or \emph{k}-mers that
occurred within 250 kb of one another.  We did consider the length of deletions
or \emph{k}-mers in defining the distance between significant variants.

For the GWAS analyses computed from SNP/indel genotype calls, we retrieved all
markers that had initially been pruned in a region ranging from 50 kb
downstream to 50 kb upstream of significant signals and computed their
\emph{p}-values by running GAPIT using the same parameters as noted above. The
\emph{p}-values thus computed on markers that were initially left out were not
used for initial signal discovery, but provided finer resolution when analyzing
the results of the SNP-based GWAS analyses. Indeed, some markers that had not
been included in the original GWAS may be the causal variant or simply yield
more significant \emph{p}-values than those initially used.

To help with the interpretation of results and the identification of candidate
genes, we defined regions of highly significant associations within each signal
by delimiting a range of coordinates containing the top 5\% most significant
associations (Paragraph) or 1\% most significant associations (SNPs/indels
and \emph{k}-mers) found within the signal. In cases where such a fraction
represented fewer than two markers or \emph{k}-mers, the highly significant
region was simply delimited by the two most significant associations.

Because previous studies \citep{bandillo2015, bandillo2017} used assembly version 1 of
Williams82, we converted the genomic coordinates of their signals to assembly version 4
using the same method that we used for converting
SoySNP50K positions (see Additional file 1). For the analysis of
qualitative traits, we only compared our results to signals at named classical
loci and ignored previously reported minor signals with \emph{p}-values that
just barely made it above the significance threshold.

\subsection*{Linking significant \emph{k}-mers to variants}

We used the reads retrieved by katcher and an assembly-based approach for
identifying the variants associated with a given set of \emph{k}-mers. In
addition to the reads themselves, we also retrieved their paired reads even
though they may not have matched a significant \emph{k}-mer in the first place.
These read pairs were used for \emph{de novo} assembly with SPAdes v. 3.15.4
\citep{spades} using the \texttt{--careful} parameter. The assemblies were done
individually for each accession and then aligned to the Williams82 reference
assembly using bwa-mem with default parameters.  Aligned assembled sequences
that overlapped loci of interest were realigned using multiple sequence
alignment with the MAFFT program v. 7.475 \citep{katoh2002} for identifying and
visualizing the haplotypes at these loci. The identification of variants
supported by significant \textit{k}-mers was based entirely on visual analysis
of haplotype alignments. We only considered haplotypes occurring at least five
times in the population in such analyses in order to leave out potential
assembly errors. In a single case (\emph{Dt1} locus), assembly at the locus
succeeded for a very low number of accessions (26). In this case, we instead used
consensus sequences from the reads aligned with bwa for comparing the
haplotypes across accessions. 

\subsection*{Computation of LD between \emph{k}-mers}

In some cases, the analysis of \emph{k}-mers revealed signals at previously
undocumented loci or found a very large number of associations throughout the
genome. To shed light on these situations, we computed the pairwise LD between
\emph{k}-mers in order to identify \emph{k}-mers that co-segregated and
likely corresponded to a single locus. Because of the sheer number of pairwise
comparisons that needed to be made for some traits (> 20,000 significant
\emph{k}-mers kept even after the filtering steps mentioned above), we limited
the number of \emph{k}-mers used for LD calculation to 1,500. For traits where
more than 1,500 \emph{k}-mers were found, we computed LD on a subsample of
these by selecting the 500 \emph{k}-mers with the most significant
\emph{p}-values and sampling the remaining 1000 \emph{k}-mers randomly. This
random sampling was done with a probability inversely proportional to the
number of \emph{k}-mers matching a given chromosome or scaffold such that
\emph{k}-mers were sampled from the whole genome and not simply from the most
significant locations.

\subsection*{Software used}
\label{sv-gwas-software-used}

Unless otherwise noted, all analyses were conducted using R v. 4.2.0
\citep{r2022} and Bioconductor v. 3.15 \citep{huber2015}. We used Bioconductor
packages Biostrings v. 2.64.0 \citep{pages2022}, GenomicRanges v. 1.48.0
\citep{lawrence2013}, Rsamtools v. 2.12.0 \citep{rsamtools2022}, rtracklayer v.
1.56.0 \citep{lawrence2009} and VariantAnnotation v. 1.42.1
\citep{obenchain2014}.  We gathered several functions used for the downstream
analysis of GWAS results into a package called gwask which is available on
GitHub (\url{https://github.com/malemay/gwask}).


%%%%%%%%%%%%%%%%%%%%%%%%%%%%%%%%%%%%%%%%%%%%%%
%%                                          %%
%% Backmatter begins here                   %%
%%                                          %%
%%%%%%%%%%%%%%%%%%%%%%%%%%%%%%%%%%%%%%%%%%%%%%

\begin{backmatter}

\section*{Acknowledgements}%% if any

We would like to thank Jonas A. Sibbesen for providing very helpful comments
on the methodology used and on a previous version of this manuscript.
We also thank Yoav Voichek for help regarding \textit{k}-mer GWAS
analysis as well as Brian Boyle and Martine Jean for valuable comments
regarding the methodology. We are thankful to the Digital Research Alliance of
Canada for using their high-performance computing servers.

\section*{Funding}%% if any

This work was supported by the SoyaGen grant (\url{https://www.soyagen.ca})
awarded to F. Belzile and R. Bélanger, and funded by Génome Québec, Genome
Canada, the government of Canada, the Ministère de l'Économie, Science et
Innovation du Québec, Semences Prograin Inc., Syngenta Canada Inc., Sevita
Genetics, Coop Fédérée, Grain Farmers of Ontario, Saskatchewan Pulse Growers,
Manitoba Pulse \& Soybean Growers, the Canadian Field Crop Research Alliance
and Producteurs de grains du Québec. M-A. Lemay has been supported by a NSERC
Canada Vanier Graduate Scholarship, a FRQNT doctoral B2X scholarship, a NSERC
Michael Smith Foreign Study Supplement, and a scholarship from the
AgroPhytoSciences NSERC CREATE Training Program. None of the funding bodies
were involved in study design, data acquisition, data analysis, interpretation
of the results, or manuscript writing.

\section*{Abbreviations}%% if any
BAM: Binary alignment map format;
CDW: corrected dry weight;
CHS: chalcone synthas;
CNV: copy number variant, copy number variation;
GRIN: Germplasm Resources Information Network;
GWAS: Genome-wide association study;
HPS: Hydrophobic protein from soybean;
LD: Linkage disequilibrium;
MAF: Minor allele frequency;
MATE: ultidrug and toxic compound extrusion;
MLM: Mixed linear model;
NCBI: National Center for Biotechnology Information;
SNP: Single-nucleotide polymorphism;
SRA: Sequence Read Archive;
SV: Structural variant, structural variation;
VCF: Variant call format;
WGS: Whole-genome (re)sequencing

\section*{Availability of data and materials}

Some of the datasets generated as part of this work are available on figshare
\citep{kmer-gwas-figshare}.

The Illumina sequencing data used for the analyses are listed in Additional file 2
and available through NCBI BioProjects \citeauthor{PRJNA257011}, \citeauthor{PRJNA289660}
and \citeauthor{PRJNA639876}.

The high-quality assemblies generated by \cite{liu2020-pangenome} are available
on the Genome Warehouse through Accession Number \citeauthor{PRJCA002030}.

The assemblies of ZH13, W05 and Lee are available on SoyBase
(\url{https://soybase.org/GlycineBlastPages/blast_descriptions.php}).

The \textit{I} locus contig assembled by \cite{tuteja2008} is available from
NCBI (\url{https://www.ncbi.nlm.nih.gov/nuccore/EF623854}).

SoySNP50K genotype calls are available from SoyBase (\url{https://soybase.org/snps/}).

The SVs identified from Oxford Nanopore data by \cite{lemay2022} are available
on figshare \citep{soybean-sv-figshare}.

The following software used in this work is available on GitHub:

\begin{itemize}
	\item Code used for the analyses:
		\url{https://github.com/malemay/soybean_kmer_gwas}
	\item katcher software for retrieving reads containing \textit{k}-mers:
		\url{https://github.com/malemay/katcher}
	\item gwask R package used for processing the output of GAPIT and
		\textit{k}-mer GWAS analyses, and for plotting:
		\url{https://github.com/malemay/gwask}
	\item Software for computing the LD between \textit{k}-mers
		based on presence/absence: \url{https://github.com/malemay/kmers_ld}
	\item Forked svmu version that was used for the analyses shown in this
		paper: \url{https://github.com/malemay/svmu}
	\item svmutools R package for converting the output of svmu to VCF
		format: \url{https://github.com/malemay/svmutools}
\end{itemize}

The code associated with the results presented in this manuscript is archived
on figshare \citep{kmer-gwas-code}.

\section*{Ethics approval and consent to participate}%% if any
Not applicable.

\section*{Competing interests}
The authors declare that they have no competing interests.

\section*{Consent for publication}%% if any
Not applicable.

\section*{Authors' contributions}
Conception and design of the study: MAL, MDR, RB, FB.
Phenotypic data acquisition (resistance to \textit{Phytophthora sojae}): MDR, RB.
Data analysis: MAL.
Data interpretation: MAL, FB.
Software writing: MAL.
Manuscript drafting: MAL, FB.
All authors have revised the manuscript and approved its submission.

%%%%%%%%%%%%%%%%%%%%%%%%%%%%%%%%%%%%%%%%%%%%%%%%%%%%%%%%%%%%%
%%                  The Bibliography                       %%
%%                                                         %%
%%  Bmc_mathpys.bst  will be used to                       %%
%%  create a .BBL file for submission.                     %%
%%  After submission of the .TEX file,                     %%
%%  you will be prompted to submit your .BBL file.         %%
%%                                                         %%
%%                                                         %%
%%  Note that the displayed Bibliography will not          %%
%%  necessarily be rendered by Latex exactly as specified  %%
%%  in the online Instructions for Authors.                %%
%%                                                         %%
%%%%%%%%%%%%%%%%%%%%%%%%%%%%%%%%%%%%%%%%%%%%%%%%%%%%%%%%%%%%%

% if your bibliography is in bibtex format, use those commands:
\bibliographystyle{vancouver} % Style BST file (bmc-mathphys, vancouver, spbasic).
\bibliography{references}      % Bibliography file (usually '*.bib' )
% for author-year bibliography (bmc-mathphys or spbasic)
% a) write to bib file (bmc-mathphys only)
% @settings{label, options="nameyear"}
% b) uncomment next line
%\nocite{label}

% or include bibliography directly:
% \begin{thebibliography}
% \bibitem{b1}
% \end{thebibliography}

%%%%%%%%%%%%%%%%%%%%%%%%%%%%%%%%%%%
%%                               %%
%% Figures                       %%
%%                               %%
%% NB: this is for captions and  %%
%% Titles. All graphics must be  %%
%% submitted separately and NOT  %%
%% included in the Tex document  %%
%%                               %%
%%%%%%%%%%%%%%%%%%%%%%%%%%%%%%%%%%%

%%
%% Do not use \listoffigures as most will included as separate files

\section*{Figures}

% Figure 1
\begin{figure}
	\caption{Results of SV- and \textit{k}-mer-based GWAS at the
	\textit{W1} locus for flower color. (a) Genome-wide Manhattan plots.
	Horizontal dotted lines indicate the significance
	threshold whereas vertical dotted lines indicate the location of the
	\textit{W1} locus. (b) Zoomed-in Manhattan plots at the \textit{W1} locus.
	The vertical dotted lines indicate the location of the
	Glyma.13g072100 gene associated with this locus. The shaded gray
	rectangles show the regions delimited by the top 5\% (SVs) and
	1\% (\textit{k}-mers) associations.  (c) Gene model of Glyma.13g072100.
	Rectangles represent exons, colored rectangles represent coding
	sequences, and arrows indicate the direction of transcription. The red
	rectangle highlights the region shown in panel (d). (d) Pairwise
	alignment of the two haplotype sequences in the population at the
	\textit{W1} locus. Nucleotides are
	colored according to the \textit{p}-value of the most significant
	\textit{k}-mer overlapping them. Vertical dotted lines between haplotypes
	indicate sequence differences, whereas dashes indicate gaps in the
	alignment. (e) Contingency table of the phenotypes and haplotypes found
	in the population at the \textit{W1} locus.}
	\label{flower-color-main-figure}
\end{figure}

% Figure 2
\begin{figure}
	\caption{Results of SNP- and \textit{k}-mer-based GWAS at the
	\textit{Td} locus for pubescence color. (a) Genome-wide Manhattan plots.
	Horizontal dotted lines indicate the significance
	threshold whereas vertical dotted lines indicate the location of the
	\textit{Td} locus. (b) Zoomed-in Manhattan plots at the \textit{Td} locus.
	The vertical dotted lines indicate the location of the
	Glyma.03g258700 gene associated with this locus. The shaded gray
	rectangles show the regions delimited by the top 1\% associations.
	In the case of SNPs, blue points denote markers used in the
	original analysis, whereas red points denote markers that had originally
	been pruned but whose \emph{p}-values were computed after signal
	discovery.
	(c) Gene model of Glyma.03g258700. Rectangles represent exons, colored
	rectangles represent coding sequences, and the arrow indicates the
	direction of transcription. The red rectangle highlights the region
	shown in panel (d). (d) Pairwise alignment of the two haplotype
	sequences in the population at the \textit{Td} locus. Nucleotides are
	colored according to the
	\textit{p}-value of the most significant \textit{k}-mer overlapping
	them. The vertical dotted line between haplotypes indicates the location of
	the causal SNP at this locus. (e) Contingency table of
	the phenotypes and haplotypes found in the population at the
	\textit{Td} locus.}
	\label{pubescence-color-main-figure}
\end{figure}

% Figure 3 with the results on pubescence color at locus G
\begin{figure}
	\caption{Results of SNP- and \textit{k}-mer-based GWAS at the
	\textit{G} locus for seed coat color. (a) Genome-wide Manhattan plots.
	Horizontal dotted lines indicate the significance
	threshold whereas vertical dotted lines indicate the location of the
	\textit{G} locus. (b) Zoomed-in Manhattan plots at the \textit{G} locus.
	The vertical dotted lines indicate the location of the
	Glyma.01g198500 gene associated with this locus. The shaded gray
	rectangles show the regions delimited by the top 1\% associations.
	In the case of SNPs, blue points denote markers used in the
	original analysis, whereas red points denote markers that had originally
	been pruned but whose \emph{p}-values were computed after signal
	discovery.
	(c) Gene model of Glyma.01g198500. Rectangles represent exons, colored
	rectangles represent coding sequences, and the arrows indicate the
	direction of transcription. The red rectangle highlights the region
	shown in panel (d). (d) Pairwise alignment of the two haplotype
	sequences in the population at the \textit{G} locus. Nucleotides are
	colored according to the
	\textit{p}-value of the most significant \textit{k}-mer overlapping
	them. The vertical dotted line between haplotypes indicates the location of
	the causal SNP at this locus. (e) Contingency table of
	the phenotypes and haplotypes found in the population at the
	\textit{G} locus.}
	\label{seed-coat-color-main-figure}
\end{figure}

% Figure 4 with the results on pubescence density at locus Ps
\begin{figure}
	\caption[Results of \textit{k}-mer-based GWAS at the
	\textit{Ps} locus for pubescence density.]{Results of \textit{k}-mer-based GWAS at the
	\textit{Ps} locus for pubescence density.
	(a) Genome-wide Manhattan plot showing the location of significant \textit{k}-mers. The
	horizontal dotted line indicates the significance threshold whereas the
	vertical dotted line indicates the location of the \textit{Ps} locus.
	(b) Zoomed-in Manhattan plot at the \textit{Ps} locus. The
	vertical dotted line indicates the location of the most significant
	\textit{k}-mer. The shaded gray area shows the location of the causal CNV.
	(c) Estimated copy number (sequencing depth at position divided by
	accession average sequencing depth) of accessions with normal and
	semi-sparse pubescence at the \textit{Ps} locus. Solid lines represent
	the median copy number across 73 (semi-sparse pubescence) and 180
	accessions (normal pubescence) computed using a 100-bp sliding window.
	Shaded areas outline the first and third quartiles. The start and end
	of the CNV region can be clearly inferred from the divergence between
	the copy number of accessions with contrasting pubescence.
	(d) Histogram showing the location of 1,169 mapped reads containing the
	most significant \textit{k}-mer in accessions with semi-sparse pubescence.
	Reads containing this \textit{k}-mer map to either end of the CNV
	region, suggesting that they are associated with this variant. Panels
	(b), (c) and (d) share the same x-axis.}
	\label{pubescence-density-main-figure}
\end{figure}

% Figure 5 with the results on pubescence form at locus Pa1
\begin{figure}
	\caption{Results of SNP- and \textit{k}-mer-based GWAS at the
	\textit{Pa1} locus for pubescence form. (a) Genome-wide Manhattan plots.
	Horizontal dotted lines indicate the significance
	threshold whereas vertical dotted lines indicate the location of the
	\textit{Pa1} and \textit{Pa2} loci. (b) Zoomed-in Manhattan plots at the \textit{Pa1} locus.
	The vertical dotted lines indicate the location of the
	Glyma.12g213900 gene suggested as a candidate for this locus. The shaded gray
	rectangles show the regions delimited by the top 1\% associations.
	In the case of SNPs, blue points denote markers used in the
	original analysis, whereas red points denote markers that had originally
	been pruned but whose \emph{p}-values were computed after signal
	discovery.
	(c) Gene model of Glyma.12g213900. Rectangles represent exons, colored
	rectangles represent coding sequences, and the arrows indicate the
	direction of transcription.
	(d) Manhattan plots at the location of the Glyma.12g213900 gene.
	Vertical lines indicate the location of two non-synonymous
	SNPs associated with appressed pubescence, which may represent causal variants
	at this locus. Panels (c) and (d) share the same x-axis.}
	\label{pubescence-form-main-figure}
\end{figure}

% Figure 6 with the results novel loci for stem termination type
\begin{figure}
	\caption[Detection of novel signals for stem termination type
	using a \textit{k}-mer GWAS analysis contrasting accessions with
	indeterminate and semi-determinate stems]{Detection of novel signals for stem termination type
	using a \textit{k}-mer-based GWAS analysis contrasting accessions with
	indeterminate and semi-determinate stems.
	(a) Genome-wide Manhattan plot showing the location of significant \textit{k}-mers.
	The horizontal dotted line indicates the significance threshold
	whereas vertical dotted lines indicate the location of the \textit{Dt1}
	classical locus as well as three new loci of interest (\textit{stGm11},
	\textit{stGm16}, and \textit{stGm18}).
	(b) Pairwise LD between the 162 significant \textit{k}-mers identified.
	\textit{k}-mers are sorted along the axes according to their
	putative position on the reference assembly version 4 of Williams82.
	\textit{k}-mers are
	represented in the same order along the x- and y-axis. The blue
	rectangles below the x-axis represent the
	-log\textsubscript{10}(\textit{p}) of each \textit{k}-mer.}
	\label{stem-termination-main-figure}
\end{figure}


%%%%%%%%%%%%%%%%%%%%%%%%%%%%%%%%%%%
%%                               %%
%% Tables                        %%
%%                               %%
%%%%%%%%%%%%%%%%%%%%%%%%%%%%%%%%%%%

%% Use of \listoftables is discouraged.
%%
\section*{Tables}

% Table 1 with results on all loci with cloned genes
\begin{table}
	\caption[Performance of three GWAS approaches at detecting causal genes/variants]%
	{Performance of three GWAS approaches at detecting causal genes or variants at cloned loci.}
	\label{loci-table}

	\footnotesize
	\centering
	\begin{threeparttable}

		\csvreader[head to column names,
		tabular = llllccc,
		table head = \toprule Trait & Locus & Gene & Causal variant & SNPs\tnote{a} & SVs\tnote{b} & \textit{k}-mers \\ \midrule,
		table foot = \bottomrule]%
		{../tables/loci_table.csv}%
		{}%
		{\Trait & \Locus & \Gene & \Causal & \platypus & \paragraph & \kmers}

		\begin{tablenotes}
		\item[a] SNPs and indels genotyped by Platypus
		\item[b] SVs genotyped by Paragraph
		\item[c] Locus not discussed in main text. See Additional file 1 for detailed results.
		\item[]- The approach detected no signal overlapping the gene.
		\item[]* The signal found at this locus overlaps the gene.
		\item[]** The region defined by the top 5\% (SVs) or 1\% (SNPs and \textit{k}-mers) variants overlaps the gene.
		\item[]*** The most significantly associated variant is the causal variant.
		\end{tablenotes}

	\end{threeparttable}
\end{table}

%%%%%%%%%%%%%%%%%%%%%%%%%%%%%%%%%%%
%%                               %%
%% Additional Files              %%
%%                               %%
%%%%%%%%%%%%%%%%%%%%%%%%%%%%%%%%%%%

\section*{Additional Files}

\subsection*{Additional file 1 --- Additional file 1.pdf}
Supplemental methods and results,
including Tables S1 to S11
and Figures S1 to S60 (PDF 12.9 MB)

\subsection*{Additional file 2 --- Additional file 2.csv}
Metadata on the sequencing runs retrieved
from the Sequence Read Archive (SRA) (CSV 71 KB)

\subsection*{Additional file 3 --- Additional file 3.csv}
Metadata on individual accessions following mapping
and comparison to SoySNP50K data (CSV 36KB)

\subsection*{Additional file 4 --- Additional file 4.csv}
Phenotypic dataset used for GWAS analyses (CSV 135 KB)

\end{backmatter}
\end{document}

